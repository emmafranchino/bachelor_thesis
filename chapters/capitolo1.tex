\chapter{Emozioni e linguaggio}
\section{Definire le emozioni}
Per arrivare a parlare del lessico emotivo è fondamentale, come prima cosa, dare una definizione di "emozione". 
Nonostante questo termine sia usato nella vita quotidiana di tutti noi, pochi saprebbero dare una definizione di esso; molto esaustiva è una citazione di Fehr e Russell che recita «tutti sanno cos'è un'emozione fino a che non si chiede loro di definirla» \parencite{russel_fehr}. Ciò è dovuto alla storia molto lunga e travagliata che interessa l'ambito emotivo.

L'interrogativo su cosa sia un'emozione viene esplicitato dal famoso articolo di W. James \textit{«What is emotion?»}, nel quale sostiene che «i cambiamenti corporei seguono direttamente la percezione del fatto eccitante, e che la nostra sensazione degli stessi cambiamenti nel momento in cui si verificano è l'emozione \parencite{james}».\\
Seppur la produzione di James sia assolutamente rivoluzionaria e ci dia grandi conoscenze sulle emozioni, dobbiamo ricordaci che rimane una tesi formulata per la prima volta da uno psicologo. 

Lo studio delle emozioni ha da sempre coinvolto molteplici settori di studio \parencite{storia_delle_emozioni}: dalla teologia, la filosofia, la retorica, alla medicina, per poi avere ancora più risonanza, a partire dal 1860, nella psicologia sperimentale e, negli ultimi decenni, nelle neuroscienze e scienze della vita.\\
Dunque, è anche importante capire cosa si intenda con il termine "emozione" nelle discipline che si sono occupate del loro studio: per esempio, secondo le teorie neuroscientifiche, sappiamo che l'emozione è considerata qualcosa di puramente corporeo, pre-verbale e inconscio; cosa che non si può di certo dire in una concettualizzazione delle emozioni dei primi stampi filosofici \footnote{Aristotele fu uno dei primi filosofi che parlò di emozioni, indicandole con il nome \textit{"pathos"} ovvero "passione". Egli le concettualizzava come risposte naturali e razionali ad eventi esterni e, dunque, credeva che fossero strettamente legate alla ragione e all'intelletto umano. Già da questa prima visione possiamo comprendere quanto le visioni delle emozioni possano differenziare in base alla disciplina di studio e al momento storico \parencite{aristotele}.}.

Un'altra questione fondamentale, che rende il compito di definizione sempre più difficile, è la lingua in cui sono state esplicitate: è interessante capire, per esempio, se il pensiero di uno psicologo francese e quello di uno scienziato britannico che si assomigliano molto, intendano effettivamente la stessa cosa.\\
Bisogna prestare attenzione non solo ai termini usati, ma ai significati ad essi attribuiti, per capire come, effettivamente, vengano concettualizzati; soprattutto attraverso delle traduzioni \footnote{Nei capitoli successivi verranno trattati più approfonditamente questi aspetti, concentrandoci in modo specifico sulle influenze culturali, che hanno condotto ad una vera e propria materia di studio da parte di antropologi e psicologi nel corso della storia.}.

Già da questa breve descrizione di alcuni dei principali aspetti che vanno considerati, possiamo capire quante definizioni differenti siano state postulate nel corso delle storia, secondo culture e discipline molto distanti. E del perché, ancora oggi, non si sia arrivati ad un accordo comune. \\
Dunque, nonostante non si abbia ancora una definizione univoca di emozione, è possibile comunque distinguerla da altri termini, con cui spesso viene confusa nella conoscenza comune.\\
Le caratteristiche che più contraddistinguono le emozioni da sentimenti, stati d'animo e affetti sono sicuramente la loro breve durata e la loro forte intensità.

Avendo specificato questo aspetto fondamentale, possiamo citare una definizione che troviamo sui dizionari, per avere un'idea più chiara di cosa si intenda nel complesso per emozione all'interno della disciplina psicologica, e indagare, poi,  altre proprietà essenziali: 
\begin{quote}
    «emozióne s. f. [dal fr. émotion, der. di émouvoir "mettere in movimento" sul modello dell’ant. motion] [...]. In psicologia, il termine indica genericamente una reazione complessa di cui entrano a far parte variazioni fisiologiche a partire da uno stato omeostatico di base ed esperienze soggettive variamente definibili (sentimenti), solitamente accompagnata da comportamenti mimici \parencite{definizione_emozioni}». 
\end{quote}

Leggendo attentamente si può capire che la definizione delle emozioni è strettamente legata al concetto di "attivazione fisiologica".\\
Quest'ultima, in gergo psicologico scientifico, viene chiamata \textit{arousal}, ed è una reazione corporea che sconvolge lo stato omeostatico, di equilibrio, del nostro organismo.\\
L'emozione viene, infatti, spiegata in termini di risposta complessa a degli stimoli (immaginari o reali) che può condurre a modificazioni corporee (come l'accelerazione del battito cardiaco, un'eccessiva sudorazione, ecc.). Queste portano poi a pattern d'azione specifici, che l'individuo metterà in atto, come, ad esempio, il principio \textit{fight or flight} ossia combattere o scappare davanti ad una minaccia \parencite{Lazarus}.\clearpage
Affinché l'individuo possa generare questa reazione è necessaria anche una fase di \textit{appraisal}, ossia una valutazione cognitiva delle modificazioni fisiologiche e della natura dello stimolo. 

Questi aspetti sono stati indagati profondamente nell'ambito scientifico psicologico, creando molteplici discussioni e mantenendo grande rilievo durante il corso dei decenni.\\
Le teorie principali generate da questi concetti vengono appunto denominate come: teorie dell'\textit{arousal} o teorie neurofisiologiche, per via del loro focus centrato prevalentemente sull'attivazione corporea, e teorie dell'\textit{appraisal} o teorie cognitive per la loro concentrazione sugli aspetti valutativi, che coinvolgono i processi cognitivi \parencite{come_funzionano_le_emozioni}.

\subsection{Teorie neurofisiologiche} 
\label{subsec:Teorie neurofisiologiche}
Le principali teorie dell’\textit{arousal} che, come sopra accennato, vengono chiamate “teorie neurofisiologiche” sono la teoria periferica di James-Lange e la teoria centrale di Cannon-Bard. Nella prima si afferma che le emozioni coincidono esattamente con le reazioni corporee associate ad esse: gli stati soggettivi che proviamo in risposta ad uno stimolo corrispondono meramente alla percezione cosciente dell’attivazione fisiologica \parencite{james}.\\
Le evidenze sperimentali hanno generato forti critiche verso questa teoria. Le principali sono state mosse da Cannon, il quale sosteneva che una stessa modificazione corporea può essere associata ad emozioni diverse e che, anche inducendo particolari reazioni fisiologiche, gli individui possono provare uno stato emotivo diverso da quello teoricamente associatogli \parencite{cannon}.\\
Bard riprende le critiche avanzate da Cannon proponendo quello che verrà denominato il 
"modello centrale" \parencite{cannon_bard}, chiamato così poiché localizzava i centri di attivazione emozionale a livello centrale, ovvero nel cervello e, più in specifico nell’area talamica.\\
In questo modello, seppure sia di tipo neurofisiologico, l’aspetto cognitivo acquisisce rilevanza e viene presa in considerazione l’importanza dell’interpretazione valutativa dell’individuo. Questa risulta, infatti, fondamentale per dare significato all’attivazione fisiologica e per generare uno stato emotivo soggettivo.

Secondo queste due teorie il sistema nervoso autonomo si attiverebbe in modo diverso a seconda degli stimoli ad esso presentati, come postulato da James-Lange; o, al contrario, per Cannon-Bard, verrebbe prodotto lo stesso pattern di attivazione simpatica, indipendentemente dagli stimoli.\\
Ad oggi, l'evidenza sperimentale, riporta che nessuna delle due teorie sia pienamente concorde con ciò che realmente succede a livello fisiologico. Si ritiene che la percezione dello stimolo generi le emozioni e che produca risposte automatiche e somatiche, ma che anche l'esperienza stessa delle emozioni sia in grado di influenzare questi due fattori. 

Il pensiero dei teorici dell'\textit{arousal} può essere riassunto secondo alcuni principi. Innanzitutto viene affermato che lo stato di attivazione fisiologica è soggetto all'interpretazione a seconda della situazione vissuta.\\
Inoltre viene specificato che l'attivazione emotiva avviene solamente se l'organismo è in uno stato di attivazione emotiva e se a questa non è possibile attribuire una spiegazione diversa \parencite{psicobiologia}.

\paragraph{Neurofisiologia}
\label{par: Neurofisiologia}
Si considera importante, a partire dall’input dato dal modello centrale Cannon-Bard, riportare una breve descrizione delle aree cerebrali coinvolte profondamente negli stati emotivi. 

Questo argomento è di fondamentale rilievo per la psicologia delle emozioni. A seconda del momento storico e dalla cultura in cui è stato sviluppato il pensiero per teorizzare le emozioni sono state messe in luce varie aree deputate alla gestione dei processi emotivi.\\
Un aspetto molto interessante è che a seconda della sede ritenuta centro dei processi emotivi da una determinata cultura, si possono individuare delle ricadute importanti anche sul campo linguistico \parencite{storia_delle_emozioni}): i termini utilizzati per descrivere determinate emozioni o espressioni corporee emotive si concentreranno infatti  su quel determinato organo.\\
Il passaggio della sede delle emozioni da un organo all’altro ha sempre avuto cause legate a questioni storiche e scientifiche.\\
Per fare alcuni esempi possiamo partire dagli antichi Egizi, i quali vedevano come centro delle emozioni il cuore che, di conseguenza, veniva conservato accuratamente all’interno di un vaso dopo la morte.\\
Oppure, secondo il pensiero del popolo di Tahiti, studiato dall’antropologo Foster, la sede principale delle emozioni come la rabbia, il desiderio e la paura era l’intestino \parencite{foster}: questa localizzazione si può ricondurre ad un’influenza Biblica oppure ai movimenti viscerali provocati dalle emozioni stesse.

Arrivando invece a modelli più recenti, a cui facciamo in parte affidamento ancora oggi, verranno di seguito spiegate le teorie più importanti nate nel XX secolo.\\
Tra il XIX e il XX secolo si imposero le ricerche dei primi neuroscienziati, che motivarono lo studio delle emozioni in un ambito cerebrale: il focus di studio delle emozioni a partire da quel momento diventa quindi il cervello, le sue aree specifiche e le connessioni tra esse.

In accordo con il modello centrale\footnote{Modello centrale descritto nelle teorie neurofisiologiche: \autoref{subsec:Teorie neurofisiologiche}.}, negli anni Trenta del Novecento, il neuroanatomista J. Papez, fu il primo ad avanzare l’ipotesi che ci fossero più strutture cerebrali, organizzate in un sistema, deputate al controllo del comportamento emozionale; in seguito nominato 'circuito di Pepez'.\\
Questo comprenderebbe diversi nuclei interconnessi (giro cingolato, ipotalamo, neuroni anteriori talamici e ippocampo), disposti ad anello intorno al talamo \parencite{papez}.\\
Successivamente questo insieme di strutture venne denominato da P. MacLean 'sistema limbico', all'interno del quale venne proposta l'aggiunta al circuito di Pepez di alcune aree ritenute fondamentali per le emozioni, come l'amigdala e la corteccia prefrontale \parencite{maclean}. 

L'amigdala ad oggi viene automaticamente associata al comportamento emotivo grazie all'importanza che ha acquisito nel tempo nell'ambito neuropsicologico.\\
La sua storia inizia a partire dagli anni Trenta del XX secolo, quando viene considerata come la struttura cerebrale responsabile del comportamento emotivo, data la sua funzione di attivare il sistema simpatico e rispondere a degli stimoli minacciosi, generando principalmente emozioni negative, in particolare la paura \parencite{goleman}.\\ 
Venne infatti anche dimostrato che alcune lesioni in quest'area provocavano cecità psichica (ossia la sindrome di Kluver-Bucy), la quale consiste nell’incapacità di riconoscere il significato emotivo di eventi, e quindi anche la tendenza ad avvicinarsi ad oggetti che normalmente evocherebbero paura \parencite{psicobiologia}.

La funzione fondamentale dell'amigdala viene poi ritenuta quella valutativa: essa, infatti, si occuperebbe di valutare gli stimoli captati dagli organi di senso, dando loro una connotazione di 'buono' o 'cattivo'.\\
La valutazione eseguita dall'amigdala viene successivamente inviata al giro del cingolo e alla corteccia prefrontale, che apporteranno una valutazione più accurata, e chiameranno in gioco le funzioni più cognitive.

Questo processo viene meglio descritto da J. LeDoux, nel 1996, con la 'doppia via' di elaborazione degli stimoli, secondo la quale il nostro cervello ha due vie per reagire ad una situazione potenzialmente pericolosa.
La prima è la via bassa: qui gli organi di senso traducono le percezioni in stimoli elettrici, i quali arrivano al talamo e, in seguito, direttamente all'amigdala. È la via più rapida, quindi con una generazione di risposte meno dettagliate.\\
Nella seconda, la via alta, gli stimoli elettrici dopo essere arrivati al talamo, vengono inviati prima alla corteccia e, solo successivamente, all'amigdala. Questo la rende la via più lenta, ma anche più precisa, data l'elaborazione aggiuntiva delle informazioni \parencite{ledoux}.

Nonostante questa teoria sia stata di estrema importanza nel campo delle neuroscienze, ad oggi non si ritiene più corretta, in quanto non sono presenti delle effettive differenze temporali nell'elaborazione degli stimoli da parte di due vie differenti.\\
In definitiva il ruolo dell'amigdala, secondo le ultime scoperte neuroscientifiche, sarebbe quello di trasmettitore tra le varie aree cerebrali implicate nei processi emotivi, grazie al quale i diversi impulsi vengono inoltrati al cervello ed ai centri di azione \parencite{psicobiologia}.

\subsection{Teorie cognitive}
\label{subsec: Teorie cognitive}
Le teorie neuroscientifiche hanno permesso di portare a galla elementi fondamentali, introdotti dalle più antiche formulazioni delle teorie psicologiche sulle emozioni. Ora verranno invece illustrate teorie delle emozioni più recenti, definite come "teorie cognitive", che si ritengono molto importanti per gli argomenti che verranno trattati nei  capitoli successivi.\\
Tali teorie vengono anche definite dell’\textit{appraisal} e partono dal presupposto che le emozioni derivino da processi cognitivi, interni all’individuo \parencite{appraisal}.

La prima che si vuole segnalare è la teoria cognitivo-attivazionale o bi-fattoriale di Schachter e Singer. Il suo nome conduce ad una teorizzazione delle emozioni a due fattori: quello cognitivo e quello biologico.
La teoria bi-fattoriale costituisce una grande innovazione grazie alla spiegazione della natura delle emozioni, a cui si giunge attraverso esperimenti formalizzati messi a punto per indagare i processi fisiologici \footnote{Schachter e Signer nel loro importantissimo esperimento utilizzarono un'iniezione di adrenalina per capire se l'attivazione fisiologica acquisisse significato nel contesto in cui si crea. Nonostante critiche dal punto di vista etico, i risultati confermarono che è la percezione-cognizione dello stimolo contesto o la sua rappresentazione cognitiva a costituire la genesi dell’emozione e a determinarne l'intensità \parencite{schachter_singer}.}, ma soprattutto quelli cognitivi.\\
Questa teoria, infatti, dà una grandissima importanza al processo cognitivo di valutazione: viene specificato che il soggetto attribuisce un valore di attivazione allo stimolo presentatogli e, in secondo luogo, assegnerà ad esso un particolare significato emotivo.
Secondo questa teoria sarà la rappresentazione cognitiva prodotta in base allo stimolo e al contesto che darà vita all'emozione stessa, con intensità diversa \parencite{schachter_singer}.

Un'altra teoria cognitiva delle emozioni, che può avvicinarci sempre di più alla comprensione del tema del lessico emotivo, obbiettivo della presente trattazione, è quella postulata da Johnson-Laird e Oatley.\\
Nella loro teoria viene messo in rilievo il piano comunicativo e, dunque, l'importanza del linguaggio nei processi emotivi. L'approccio degli autori è quello di costruire una teoria emotiva collegata, oltre che all'elaborazione cognitiva, alle teorie computazionali del linguaggio \footnote{Le teorie computazionali fanno parte di una branchia della linguistica che si occupa di adattare il linguaggio naturale ai sistemi tecnologici, quali i computer \parencite{teorie_computazionali}.}.

Per descrivere le emozioni, nel loro articolo, gli autori iniziano dividendo le loro funzioni su due livelli comunicativi.\\
In primo luogo, le emozioni sono viste come una forma di comunicazione interna, definita non propositiva, la quale ha la funzione di priorizzare gli obbiettivi di azione e mantenerne tali priorità.\\
Sul secondo livello, invece, le emozioni sono viste come una forma di comunicazione esterna: qui nascono le emozioni complesse, generate in corrispondenza di piani sociali.\\
Questo tipo di comunicazione è proposizionale, ciò significa che è composta da segnali proposizionali, i quali corrispondono a schemi di chiamata che riescono ad invocare funzioni situate al livello inferiore (non proposizionale), diverse rappresentazioni del mondo, messaggi e funzioni che possono portare alla costruzione di nuove procedure.\\
Questo tipo di comunicazione si distingue da quello non proposizionale, proprio del primo livello descritto, che è molto più semplice, grezzo e antico dal punto di vista evolutivo.\\
I segnali non proposizionali, infatti, non posseggono una struttura simbolica e significativa per il sistema cognitivo. Il loro funzionamento è puramente casuale e immediato e servono per agire velocemente portando il sistema in una particolare "modalità di emozione" \parencite{Keith_JohnsonLaird}.\\
Tali modalità di emozioni, considerate primitive ed innate, vengono interpretate grazie alle valutazioni propositive, generando così specifiche emozioni. 

Questa teoria si differenzia da altre apparentemente molto simili poiché le emozioni complesse non vengono generate dall'insieme di più emozioni di base, in questo caso di modalità di emozioni, ma nascono dalla valutazione cognitiva, grazie ai segnali proposizionali.\clearpage
Le emozioni, quindi, nascono grazie al sistema comunicativo generato dai segnali delle stesse emozioni. Un primo livello, funzionante senza dati proposizionali, può dare vita immediatamente a modalità di emozione, successivamente valutate cognitivamente, su un livello più alto.\\
Per concludere le emozioni cono definite come «stati cognitivi che coordinano processi quasi autonomi nel sistema nervoso \parencite{Keith_JohnsonLaird}».

Questa teoria ci permette di introdurre la linguistica, disciplina, i cui argomenti ci serviranno per comprendere al meglio il lessico emotivo: protagonista assoluto di questo elaborato. 

\section{Introduzione alla linguistica}
Prima di porre il focus sul lessico emotivo e sull'approfondimento delle correnti psicologiche che danno voce ad esso, si ritiene importante spiegare brevemente le caratteristiche principali del linguaggio in senso stretto.

La disciplina che ci permette di analizzare accuratamente il linguaggio e i suoi risvolti nell'ambito psicologico e antropologico è la linguistica. Essa, infatti, è lo studio scientifico del linguaggio verbale umano e delle strutture che lo compongono; ma anche delle lingue in relazione ad un momento storico preciso ed alla loro cultura di riferimento \parencite{introduzione_linguistica}.\\
Data questa esplicitazione relativa alla linguistica, è bene anche chiarire la distinzione tra linguaggio e lingua. 

Il linguaggio è definito come «un sistema di comunicazione che consiste in suoni, parole e grammatica \parencite{langauge}».\\
Rappresenta, dunque, l'insieme di fenomeni della comunicazione che troviamo all'interno delle interazioni umane e al di fuori di esse (possiamo trovare forme di linguaggio anche negli animali o nelle macchine). \\
La lingua, invece, è un modo in cui si manifesta il linguaggio, determinato dal punto di vista storico e spaziale; un sistema attraverso il quale gli individui di una determinata comunità riescono a comunicare tra loro \parencite{lingua}.

Il linguaggio viene quindi considerato come un prerequisito per la lingua: è ciò che permette di creare i sistemi comunicativi, ovvero le lingue. É costituito anche da abilità e processi cognitivi molto complessi e difficilmente localizzabili nei circuiti cerebrali, che rendono particolarmente difficile il suo studio.\\
Ogni lingua, inoltre, ha una propria storia ed evoluzione e si distingue nettamente dai linguaggi artificiali (come, ad esempio, i segnali stradali).

La relazione tra lingua e linguaggio ci permette anche di fare chiarezza sulle componenti naturali e culturali della linguistica.\\
La parte considerata di natura biologica nell'essere umano è quella del linguaggio, mentre la lingua è considerata la componente culturale. Le lingue vengono studiate in relazione all'ambiente sociale e culturale in cui si trovano.

La predisposizione al linguaggio verrebbe dunque intesa come innata ed universale, trasmessa geneticamente, mentre la lingua come apprendibile, trasmessa per contatto, a seconda degli elementi ambientali e sociali che ci circondano \parencite{chomsky_predisposizione_linguaggio}. \\
In altre parole potremmo dire che il linguaggio è ciò che accomuna gli individui, mentre la lingua è ciò che li differenzia. Essendo la lingua relativa al contesto in cui viene utilizzata, essa presente diverse caratteristiche peculiari. \\
Innanzitutto possiede un ciclo di vita proprio: nasce, muta e muore. Nasce in una cultura specifica, attraverso le differenti influenze sociali. Si sviluppa come una proprietà mutevole: cambia a seconda dello spazio, del tempo, della formalità della situazione, dell’argomento, del rapporto che si ha con l’altro. Infine, se non viene più parlata, quindi se non viene più mantenuta in vita mediante un suo utilizzo, muore \parencite{lingue_e_linguaggio}.

\subsection{Funzioni linguistiche}
La linguistica, nel suo ampio ambito di studio, prende in considerazione tutti gli aspetti del linguaggio, che si possono dividere in interni ed esterni.\\
Quelli interni si riferiscono allo studio della morfologia, della fonologia, del lessico e della sintassi. Per quanto riguarda quelli esterni, invece, intendiamo lo studio degli aspetti comunicativi del linguaggio. 

La funzionalità principale della lingua, è proprio quella comunicativa, che permette agli individui di interagire con il mondo esterno e produrre, a livello interno, pensieri e ragionamenti.\\
Le funzioni della lingua, dunque, sono: comunicare con gli altri, parlare a noi stessi, pensare, formulare ragionamenti, produrre nuove idee e raccontarle \parencite{fondamenti_linguistica}.

Le funzioni linguistiche vengono definite con precisione dal linguista russo Roman Jakobson. Egli ne distingue sei: funzione emotiva o espressiva, conativa, referenziale, fàtica, metalinguistica e poetica.\\
Quella più rilevante per la comprensione del lessico emotivo è sicuramente la funzione emotiva o espressiva che si preoccupa di esprimere le emozioni, gli stati d'animo, gli atteggiamenti del mittente. Centrale è proprio il ruolo di quest'ultimo, che utilizza tutti gli elementi grammaticali declinati in prima persona, diventando il protagonista assoluto del racconto. In questo caso risulta quindi fondamentale la capacità di sapere esprimere le proprie emozioni e riuscire a parlare di sé \parencite{Jakobson}.

Ognuna delle funzioni linguistiche appena citate, risultano quindi fondamentali per la creazione di processi comunicativi e, infatti, ad ognuna di essere corrisponde ad una variabile comunicativa \parencite{Jakobson}. \\
Le variabili comunicative sono le parti di cui si compongono i processi comunicativi, che vanno analizzate per comprendere il funzionamento globale della comunicazione stessa \parencite{linguistica_comunicazione}.\\
I fattori comunicativi individuati da sono: il codice, per il quale si intende un «sistema di segni teso a trasmettere informazione tra un mittente e un ricevente, per il tramite di un messaggio \parencite{fondamenti_linguistica}», è ciò che  che permette effettivamente di formulare il messaggio, e può corrispondere, ad esempio, ad una lingua specifica. Il messaggio, ovvero ciò che si vuole comunicare all'altro attraverso il codice, è quindi il contenuto proprio dell'atto comunicativo. L'emittente, cioè colui che vuole comunicare il messaggio e il ricevente che, oppostamente, è colui a cui è rivolto il messaggio. Il canale, che rappresenta il mezzo che mette in relazione l'apparato utilizzato dell'emittente e quello del ricevente. Infine, il contesto, con il quale si intende l'insieme degli elementi di un testo messi in correlazione fra loro. Il contesto può anche essere considerato come lo sfondo della situazione di cui si sta parlando \parencite{fondamenti_linguistica}.

\paragraph{Semantica} 
\label{par: Semantica}
Una delle componenti principali che interessa lo studio della comunicazione, sulla quale è bene soffermarsi per comprendere molti aspetti del lessico emotivo, è la semantica.\\
Con semantica si intende «l'analisi e studio del linguaggio dal punto di vista del significato \parencite{semantica}». L'argomento di studio che concerne la semantica si occuperebbe quindi del rapporto tra significante e significato; considerando proprio la definizione di significante come: «elemento formale, fonico o grafico, del segno linguistico, a cui corrisponde l'elemento concettuale, detto significato \parencite{significante}».\\
La semantica, dunque, può riferirsi alla rappresentazione mentale che abbiamo di un’entità, oppure a quella stessa entità nella realtà.\\
L'insieme di significati che costituiscono le rappresentazioni mentali di ogni individuo si differenziano in base alla lingua parlata, a causa di diverse variabili, quali: la cultura in cui ci troviamo, il livello intellettuale degli interlocutori a cui ci rivolgiamo, l'ambito concettuale entro il quale ci stiamo esprimendo (per esempio se ci troviamo in ambito cinematografico, giuridico, ecc.) \parencite{lessico_semnatica}. 

Lo studio della semantica risulta quindi fondamentale per comprendere l'altro e rispondere adeguatamente in un processo di interazione.\\
Ci permette anche di andare ad indagare attentamente le differenze semantiche che troviamo nelle diverse lingue. Una stessa parola, infatti, può avere significati differenti in lingue diverse a causa dei fattori contestuali, ed è proprio per questo motivo che bisogna porre moltissima attenzione al processo di traduzione.  

Per riassumere e sottolineare nuovamente l'importanza del linguaggio e della semantica possiamo riflettere sul fatto che è proprio questo che permette all'individuo di costruire pensieri e ragionamenti: a seconda della vastità del vocabolario che una persona possiede, potrà generare pensieri più o meno articolati e avrà la possibilità di farsi capire in differente misura dalle persone che lo circondano. È dunque il linguaggio ciò che rende possibile comunicare con l'altro e creare legami interpersonali.\\
Il linguaggio ha il grande potere di unire o separare le persone: un significato diverso che si attribuisce ad una parola o frase può portare ad incomprensioni e divergenze, così come comprendere il linguaggio altrui può instaurare coesione ed intesa.

\subsection{Famiglie linguistiche}
Un altro elemento che è interessante da prendere in considerazione per comprendere successivamente le differenze culturali del lessico emotivo, è l'analisi delle lingue in base alla loro locazione spaziale e le relazioni tra esse.\\
Le molteplici analogie e differenze che troviamo tra le miriadi di lingue utilizzate nel mondo sono proprio dovute a vicinanza o lontananza rispetto al luogo in cui hanno avuto origine, o a cause storiche come, ad esempio, i processi di colonizzazione, che hanno portato allo spostamento di popoli da una parte all'altra del mondo, come pure all'espansione della propria lingua e cultura. 

Elementi comuni a diverse lingue hanno permesso di raggrupparle in diverse famiglie linguistiche, che ci permettono di comprendere al meglio le loro similitudini.\\
Le famiglie linguistiche sono un insieme di lingue che hanno un 'antenato comune', ovvero che sono nate da una lingua comune (ad esempio dal latino), chiamata 'protolingua' \parencite{Family_of_language}.\\
Ad oggi vengono identificate circa 150 famiglie linguistiche, alcune più grandi di altre, all'interno delle quali si localizzano molteplici lingue. Data la grande vastità di lingue che possiamo trovare, è possibile anche dedurre che alcune di esse saranno più simili di altre.\\
Bisogna tenere conto che esistono anche i gruppi, i sottogruppi, i rami: una grandissima vastità di suddivisioni. \\
La famiglia linguistica più diffusa di tutte è quella Indo-europea, che si compone di nove rami tra cui possiamo citare, tra quelli più diffusi in Europa, le lingue romanze, celtiche, germaniche e baltiche: i rami più comuni in Europa \parencite{fam_linguistiche}. 

Queste suddivisioni sono un argomento di indubbio interesse, che si può indagare nei minimi dettagli, ma  che, in questo contesto non sarà possibile approfondire, tuttavia risulta necessario fare riferimento ai ceppi linguistici per avere un’idea più chiara delle ragioni per cui alcune lingue si somigliano così tanto, del perché esista un lessico emotivo, uno stile di  vita, delle norme culturali che influiscono sulle similarità o peculiarità specifiche nel confronto tra le diverse lingue.\\
Per esempio, sapendo che dalla famiglia delle lingue romanze (o di matrice latina) si originano l’italiano, il francese, lo spagnolo, il portoghese, il rumeno e il catalano, possiamo fare caso ai molteplici termini simili, che andranno studiati più approfonditamente, considerando molti altri fattori sociali per capire il nesso tra lessico e cultura. 

\subsection{Rilevanza emotiva del linguaggio}
\paragraph{Suicidi per amore della lingua}
Un primo evento che ci può far comprendere quanto l'aspetto linguistico sia di grandissimo rilievo nella vita degli individui è la storia della questione linguistica indiana, dove si parla addirittura di "morte per amore della lingua" \parencite{language_emotion_politics}. 

Attualmente la Costituzione indiana riconosce l'inglese come seconda lingua ufficiale del Paese, ma le vicissitudini storiche che hanno portato a questo riconoscimento sono state tortuose e drammatiche.\\
Tutto nacque con l'affermarsi del colonialismo britannico in India, che impose, per legge, l'utilizzo dell'inglese in tutto il Paese.\\
Successivamente, negli anni Cinquanta, con l'indipendenza indiana si cercò di istituire una sola lingua nazionale che rappresentasse la cultura del Paese: l'Hindi. Nonostante questa lingua fosse parlata da quasi un terzo della popolazione, questa rimaneva propria solamente degli abitanti di regioni nordiche della nazione, trascurando tutta la popolazione del Sud.\\
Una soluzione proposta per non avvantaggiare solamente una parte del Paese ed escluderne un'altra fu l'utilizzo della lingua coloniale (l'inglese), ritenuta più neutra poiché non in grado di identificare e distinguere l'etnia, la religione o il rango sociale.\\
Questo conflitto sulla lingua più consona da utilizzare per una maggiore coesione del popolo si concluse poi con un compromesso tra le due lingue: pur mantenendo l'Hindi come lingua ufficiale, il governo indiano permise a ciascuno degli Stati indiani di cui è composta, di adottare l'inglese come lingua principale.\\
L'inglese ha visto in questi casi un suo uso pervasivo in tutti gli ambiti della vita quotidiana e, soprattutto nell'ambito scolastico: molti indiani cercano, infatti, di inserire i propri figli nelle scuole inglesi. Questo principalmente perché l'inglese è sempre stato un elemento discriminatorio delle classi sociali più abbienti, e questa concezione permane ancora oggi \parencite{indiamodi}. 

Prima di arrivare a questa soluzione di compromesso tra l'inglese e l'Hindi, ancora oggi messa in discussione, ci furono incredibili proteste da parte degli abitanti meridionali dello Stato, che portarono a quelli che vennero chiamatati «suicidi per amore della lingua \parencite{language_emotion_politics}».\\
Il primo fu quello di Potti Sriramulu, un leader politico indiano, che ha lottato affinché i Telugu di Andhra avessero un proprio stato e una propria lingua, il \textit{Telugu}. Per combattere per i suoi ideali intraprese uno sciopero della fame che lo condusse alla morte, da cui prese il via una protesta più estesa grazie alle moltissime persone dell'India meridionale che lo presero come esempio. 

Tante furono le spiegazioni che vennero date a questo movimento di ribellione, come la ricerca di fama e di importanza, nessuna però sufficientemente convincente per gli antropologi che studiarono questo caso.\\
Una di questi fu Lisa Mitchell che, nel suo libro \textit{"Language, Emotion and Politics in South India"}, illustra il pensiero del XIX secolo, sviluppato da J. G. Herder \footnote{J. G. Herder (1774-1803) fu un importantissimo filosofo tedesco, uno degli intellettuali più rilevanti nella corrente dello \textit{Sturm und Drang}, dove si celebra il sentimento, la natura e l'estetica. Nella sua filosofia del linguaggio egli afferma che «nella parola è l'anima stessa che si esprime e, viceversa, l'anima esiste solo in quanto si esprime nella parola \parencite{Herder}», portando alla luce un nuovo legame tra lingua e spiritualità.}, il quale si interroga sull'origine naturale del linguaggio, giungendo alla concezione della lingua come un essere vivente, attribuendogli le fasi dello sviluppo e la componente relazionale propria del rapporto tra individui.\\
Un altro punto fondamentale per interpretare al meglio questo pensiero è la concezione che l'uomo è in grado di svilupparsi solamente ricreando continuamente il suo linguaggio.\\
Partendo da questa corrente di pensiero si può dunque concepire anche la morte di una lingua, così come era stato fatto per le altre fasi vitali. 

In aggiunta, la lingua fu vista come divinità indiana, e dunque fu spostata anche sul piano spirituale, di culto, oltre che su quello biologico.\\
Infine, l'elemento che funse da "cassa di risonanza" furono i mass media, che permisero che l'amore che si era sviluppato per lingua si diffondesse in tutto il Paese \parencite{language_emotion_politics}. 

Da questa intensa e violenta rappresentazione di amore per la lingua possiamo comprendere quanto abbia rilevanza nella vita di un popolo la propria lingua, in quanto rappresenta ciò che permette di mantenere viva la cultura specifica di un popolo e che, se denigrata, porta alla cancellazione dell'identità etnica di una comunità.\\
La grande presenza di emozioni che legano l'uomo al suo linguaggio e la risonanza emotiva che hanno avuto le manifestazioni di protesta per proteggere la propria identità culturale mette in rilievo la vicinanza del piano emotivo a quello lessicale.

\paragraph{Svolta linguistica}
\textit{The linguistic turn}, o in italiano "svolta linguistica", è l'espressione che si usa per far riferimento ad un fenomeno culturale e intellettuale che ha avuto luogo durante il XX secolo.\\
Questa corrente si sviluppò principalmente nell'ambito della filosofia occidentale, dando successivamente vita alla filosofia analitica e linguistica \parencite{Rorty_Linguistic_Turn}. 

Il termine fu coniato dal filosofo Gustav Bergmann negli anni Cinquanta, e reso celebre da Richard Rorty con la sua antologia \textit{"The linguistic turn"} (1967), il quale si dissociò, in seguito, dal pensiero della filosofia linguistica.\\
La caratteristica più importante della svolta linguistica fu proprio l'attenzione della filosofia e delle altre discipline umanistiche principalmente alle relazioni tra il linguaggio, coloro che lo usano e il mondo esterno \parencite{linguistic_turn}. 

Un obbiettivo che si voleva perseguire era quello di superare il dualismo della dimensione soggettiva e oggettiva della mente, nata dal dualismo genitore di tutti: quello cartesiano, che divise mente e corpo.\\
Ciò che si intende con il raggiungimento di questo obbiettivo è quello di vedere l'oggettività della realtà a partire dal recupero della dimensione soggettiva dell'individuo, mettendo in luce la sua coscienza.\\
Il \textit{linguistic turn} costituisce una vera e propria svolta poiché, attraverso l'analisi della logica e della filosofia del linguaggio, sposta il focus dalla dimensione soggettiva della mente sul linguaggio stesso.\\
Si cercherà, quindi, di fare un'analisi molto approfondita del linguaggio utilizzato per dare un significato obbiettivo, partendo da un punto di vista soggettivo. 

La svolta linguistica, con il suo cambio di paradigma, ha interessato molte discipline.\\
Un effetto evidente è stato quello avvenuto nell'ambito della storia, più precisamente nella storiografia \parencite{Toews_linguistic} \footnote{Per storiografia si intende una «scienza e pratica dello scrivere opere relative a eventi storici del passato, in quanto si possano riconoscere in essa un’indagine critica e dei principi metodologici: i metodi della s.; storia della s., che ha per oggetto l’evolversi del metodo storico \parencite{storiografia}».}.\\
In questo caso viene posta al centro l'analisi del linguaggio utilizzato dagli storici, dal punto di vista della logica e della filosofia, che permetterebbe di sorpassare la dimensione soggettiva dell'autore, riuscendo a raggiungere l'oggettività del contenuto.\\
Con lo studio dei testi storici come artefatti linguistici, proprio come se fossero opere narrative e letterarie, è possibile comprendere la costruzione della realtà soggettiva dell'autore: ma è proprio la forma linguistica analizzata che determina l'"oggettività" del contenuto.\\
Questo concetto viene introdotto da H. White, con la sua pubblicazione \textit{"Metahistory"} \parencite{white}, nella quale viene sottolineata l'importanza di analizzare il discorso degli storici per comprendere la costruzione delle identità individuali e collettive. 

La presunta oggettività della storicità viene quindi messa in discussione, facendo presente che è sempre mediata dal pensiero etico e politico dello storico, che manipola la descrizione storica. É proprio dall’analisi semantica del linguaggio utilizzato che si può arrivare a riconoscere i limiti della realtà oggettiva del passato e l’enorme influenza che la percezione soggettiva dello storico ha esercitato sulla ricostruzione degli eventi storici.\\
Fino a quel momento, infatti, la soggettività dello storico era stata totalmente trascurata lasciando credere che fosse possibile ad una descrizione oggettiva dei fatti.\\
In sintesi dunque, con l’affermarsi del \textit{linguistic turn} gli studiosi cominciano a prendere coscienza del fatto che il passato non esiste al di fuori delle rappresentazioni testuali e che queste rappresentazioni, che sono rappresentazioni linguistiche, non possono essere separate dal bagaglio ideologico che gli storici portano con sé. 

Prendere coscienza di questi aspetti ha permesso innanzitutto di indagare alcuni aspetti fino ad allora trascurati nell’ambito storico come, ad esempio, la storia di genere, che si occupa di esaminare  il ruolo del genere e delle dinamiche di potere nella storia ed ha avuto una grande rilevanza  nella lotta per una maggiore parità tra generi \parencite{storia_di_genere}; o la memorialistica, ovvero la scritture delle memorie personali ed autobiografiche degli individui, che pur lasciando ampio spazio alla soggettività dell’autore permette di capire in modo esplicito come e in che misura la visione emotivo/soggettiva di chi descrive gli eventi permette di comprenderne i significati senza trascurare gli aspetti culturali, sociali e del contesto inteso sia come spazio che come tempo \parencite{memorialistica}.\\
In questa prospettiva diviene possibile afferrare quanto l’analisi semantica del linguaggio possa aprire nuovi mondi alla conoscenza dei fatti e permetta di sradicare concezioni assodate nel corso del tempo come oggettive, che invece vanno contestualizzate e relativizzate al contesto socio-culturale, al momento storico e alle ideologie e dimensioni emotive dell'autore. 

Lo studio analitico del linguaggio e della dimensione soggettiva degli autori ha permesso anche di dare più spazio all'ambito emotivo. Attraverso il linguaggio si possono individuare le emozioni provate, dando così la possibilità di ricostruire le norme sociali e culturali che veicolano quelle particolari emozioni intercettate.\\
\textit{The linguistic turn} fu quindi una grande "cassa di risonanza" per la dimensione emotiva, il linguaggio e, quindi, per lo studio del lessico emotivo.  

\paragraph{\textit{Affect labeling}}
\label{par: Affect labeling}
Raggiungendo tempi più recenti, possiamo trovare alcune indagini interessanti sull'\textit{affect labeling}: altro argomento che evidenzia la stretta correlazione tra emozioni e linguaggio.

Ciò che viene identificato come \textit{affect labeling}, letteralmente "etichettatura degli affetti", può essere spiegato semplicemente con l'espressione \textit{"putting feelings into words"} cioè "inserire i sentimenti all'interno delle parole".\\
Quello che vuole esplicitare quest'espressione è un metodo che consiste nell'etichettare esplicitamente le emozioni che si stanno provando \parencite{naural_basis_labeling}.\\
Alcune delle tecniche utilizzate per l'\textit{affect labeling} potrebbero essere, ad esempio: parlare con un terapeuta, scrivere le proprie esperienze interiori su un diario, sfogarsi con un amico; insomma, prendere coscienza dei propri sentimenti negativi attraverso l'esplicitazione a parole di essi \parencite{modalità_labeling}. 

Gli studi condotti sull'\textit{affect labeling} attraverso metodologie neuroscientifiche, si sono concertati in particolar modo sull'etichettatura delle emozioni negative, poiché è proprio durante questo processo che sono stati riscontrati alcuni effetti benefici psichici per l'individuo.\\
Tali ricerche hanno infatti dimostrato che parlare delle proprie emozioni negative, attraverso i vari modi sopracitati, aiuterebbe ad interfacciarci con esse, riuscendo a reagire meglio sia a livello interiore, mentale, prendendone coscienza; sia a livello esteriore, comportamentale.\\
Il processo che si mette in atto può essere descritto come una tecnica di regolazione emotiva molto potente: quando esprimiamo ciò che sentiamo a parole regoliamo automaticamente le nostre emozioni, ottenendo effetti positivi anche senza volerlo.

Grazie alla risonanza magnetica, una delle tecniche neuroscientifiche utilizzate per questi studi, è stato anche possibile comprendere i principali processi neurofisiologici messi in atto durante l'\textit{affect labeling} \parencite{fmri_affect_labeling}; riuscendo ad approfondire lo studio delle regioni cerebrali che coinvolgono il linguaggio e le esperienze emotive. 

Ciò che gli autori prendono in considerazione per rendere ragione dell’effetto benefico del dare parola alle emozioni sono gli aspetti di codifica delle informazioni e le aree cerebrali coinvolte \footnote{Possiamo partire dalle informazioni descritte nel \autoref{par: Neurofisiologia} e ampliarle.}.\\
Quello che è emerso da questo studio è che nell’etichettatura degli affetti, rispetto ad altri processi di codifica, si assiste ad una diminuzione significativa dell’attivazione dell’amigdala e di altre regioni limbiche quando vengono presentati ai soggetti sperimentali immagini di emozioni negative.\\
L'unica regione cerebrale che aumenta la sua attività nell'elaborazione linguistica delle emozioni negative è la corteccia prefrontale ventrolaterale destra (RVLPFC): questa è associata all'elaborazione simbolica delle informazioni emotive e ai processi inibitori top-down.

Questi risultati ci dicono che l'esplicitazione delle emozioni negative a parole, attraverso i processi linguistici, provocano un'attivazione della RVLPFC e una conseguente diminuzione responsiva dell'amigdala, che conduce ad un calo sostanziale dello stress emotivo \parencite{affect_labeling}.
Più nello specifico la conclusione a cui giungono i ricercatori è che ci sia una riduzione della reattività emotiva lungo il percorso che va dalla corteccia prefrontale ventrolaterale destra alla corteccia prefrontale mediale all’amigdala \parencite{affect_labeling}.\\
Nonostante i risultati ottenuti, rimangono molti altri processi cerebrali legati a questo fenomeno che necessitano di ulteriori indagini. 

L'\textit{affect labeling} viene ritenuto di grande importanza per l'impatto benefico che può avere in situazioni terapeutiche: invitando il paziente ad esprimere le proprie emozioni, infatti, possiamo aiutarlo a ridurre il suo malessere psichico. Inoltre questa tecnica può essere utilizzata da ciascuno di noi anche in situazioni quotidiane: si può prendere l'abitudine a scrivere un diario o a parlare più soventemente con qualcuno delle proprie emozioni negative.

Gli studi condotti su quest'argomento, inoltre, portano alla luce la grande connessione tra l'ambito linguistico e comunicativo e quello emotivo e psicologico.\\
L'aspetto del lessico emotivo acquisisce ampia rilevanza: è proprio grazie al lessico emotivo che si possiede che è possibile esprimere adeguatamente i propri stati emotivi, riconoscendone quelli positivi e negativi in base al contesto socio-culturale in cui ci si trova.