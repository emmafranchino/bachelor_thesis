\chapter{Lessico emotivo} 
\label{chap: Lessico emotivo}

\section{Definire il lessico emotivo}
Seppure il senso comune non trovi una grande rilevanza nella relazione tra emozioni e linguaggio, anzi vi è più la concezione delle emozioni come qualcosa di fisico, distaccato dalle etichette linguistiche ad esse associate, la psicologia vede il linguaggio come fondamentale per i processi emotivi: sia per la percezione delle emozioni che per la loro sperimentazione concreta\footnote{Tema affrontato in modo approfondito dalla corrente di pensiero psicologico di tipo costruzionista, che verrà indagato nella \autoref{subsec: Socio-costruttivismo}.}.\\
Attualmente infatti, l'indagine delle emozioni umane richiede in maniera imprescindibile il contributo dello studio delle lingue, portando la psicologia ad interagire con altri ambiti di ricerca come l'antropologia e la sociologia, ma anche la linguistica e, più precisamente la psicolinguistica, disciplina nata negli anni Cinquanta, che studia i processi psicologici e neurobiologici che permettono l'acquisizione, la comprensione e l'utilizzo del linguaggio \parencite{psicolinguistica}. 

Il ruolo centrale del linguaggio nello studio delle emozioni risulta particolarmente rilevante quando si prendono in considerazione gli aspetti transculturali: a volte, ad esempio, un gruppo sembra concentrare maggiormente l'attenzione su stati emotivi per i quali altri gruppi non hanno nemmeno un nome \parencite{Wierzbicka}.\\
L'analisi approfondita del lessico emotivo, vale a dire dei significati degli stati emotivi nei diversi gruppi linguistici e culturali ha permesso di apportare elementi conoscitivi molto importanti sulle emozioni e sul contesto culturale all'interno del quale tali emozioni vengono espresse.

L'obbiettivo del presente elaborato è analizzare e definire il lessico emotivo e capire la sua funzione nella vita quotidiana degli individui.\\
Il lessico emotivo viene definito come: 

\begin{quote}
    «l'insieme di parole con le quali vengono identificate, isolate e distinte le differenti forme di vissuti emotivi e affettivi, nei diversi linguaggi umani \parencite{development_of_emotioinal_competencde}».
\end{quote}

In altri termini, per lessico emozionale si può intendere quell'insieme che raggruppa tutte le parole che usiamo nella vita quotidiana per parlare delle nostre emozioni e di quelle altrui: nello studio del lessico emotivo si prendono in considerazione gli aspetti riguardanti le parole di una lingua e le loro interazioni con le emozioni.\\
Il lessico emotivo ci dà la possibilità di trasmettere varie informazioni sull'emozione che stiamo provando, specificando di che emozione si tratta oppure spiegando l'evento che l'ha causata.

Nello studio del lessico emotivo, oltre ai contenuti semantici che possono essere considerati di tipo emotivo, bisogna esplorare anche le risorse utilizzate per trasmettere tali significati e riuscire a comunicare tutti i dati relativi ad uno stato uno emotivo.\\
Le risorse in questione, che si distinguono in diverse tipologie, vengono definite segmentali. Questo significa che sono considerate come segmenti, ovvero elementi linguistici, disposti in successione temporale, che in una frase formano una sequenza lineare \parencite{segmentale}; le risorse sono dunque segmenti del processo comunicativo.\\
Verranno di seguito elencati i tipi di risorse segmentali prese in considerazione nello studio della comunicazione emotiva attraverso il linguaggio.\\
Risorse lessicali: sono le più facilmente riconoscibili e costituiscono l'insieme dei nomi, aggettivi, verbi avverbi ed esclamazioni (ad esempio: felicità, rallegrarsi, gioiosamente, ecc.) che usiamo per denominare e descrivere approfonditamente gli stati emotivi, le particolari sensazioni provate da noi o da altri in un certo conteso, riuscendo a comunicarle facilmente. Risorse sintattiche: riguardano la struttura della frase, si riferiscono all'ordine specifico di ogni parola all'interno della frase, in modo che sia possibile esprimere le proprie idee. Tale risorsa permette di produrre uno specifico significato emotivo, riuscendo a portare l'enfasi si determinati elementi connotati emotivamente: ad esempio cambiando l'ordine delle parole o ripetendo la stessa più volte a inizio frase (anafora) è possibile enfatizzare quell'elemento specifico. 
Risorse morfologiche: sono costituite da diminutivi, vezzeggiativi o dispregiativi. Queste risorse permettono di dare un particolare connotato (negativo o positivo) alle parole che stiamo esprimendo, riuscendo così ad invocare emozioni coerenti con il connotato che si è dato nel ricevente del messaggio. 
Infine le risorse fonologiche sono riconducibili ad esempio ai fonosimboli, ossia delle manifestazioni foniche o onomatopeiche, che quindi riconducono a suoni o rumori (per esempio, nell'ambiente linguistico italiano, uffa, mah, bah) \parencite{fonosimbolo}.

Tali risorse, utilizzate congiuntamente in modo adeguato e coerente all'interno di un atto comunicativo, ci permettono di esprimere gli stati emotivi personali e, al contempo, capire quelli altrui.\\ 
Un altro mezzo che risulta fondamentale per adempire questo compito è, sicuramente, il contesto in cui ci si trova: analizzare attentamente la situazione in cui si è immersi ci permette di dare un senso a ciò che viene comunicato attraverso le risorse segmentali \parencite{parlato_emotivo}.  

Lo studio del lessico emotivo, inoltre, è considerato una via di accesso importante per studiare le emozioni stesse: vengono analizzati i vari modi in cui le emozioni sono identificate, isolate e distinte in base alla lingua parlata.

Ciò che viene particolarmenete indagato è la struttura semantica del lessico emotivo, ovvero «una sorta di linea guida con cui la nostra conoscenza dell'esperienza emotiva è strutturata e organizzata nella nostra mente \parencite{Sini}».\\
Questa si può mettere in luce attraverso l'analisi approfondita della considerazione di ogni individuo dei significati delle parole associate alle emozioni e, nello specifico, del modo in cui ognuno di noi mette in relazione tra loro queste parole. \\
La struttura semantica rappresenta quindi l'insieme dei significati che ognuno di noi esprime attraverso il linguaggio.\\
Un concetto basilare da comprendere è che il modo in cui gli individui utilizzano il linguaggio permette di accedere indirettamente alla modalità in cui sono organizzati i concetti internamente. \\
La grande rilevanza che viene data al linguaggio riguarda proprio questo punto cruciale: la disponibilità di vocaboli che abbiamo, la modalità in cui li organizziamo tra loro per formare dei significati, si traduce poi nei nostri pensieri e nella forma in cui riusciamo ad esprimerci e comunicare con gli altri. La struttura semantica  è la responsabile dei significati che ognuno di noi dà al mondo esteriore ed interiore. 

Il lessico emotivo sarà quindi un potentissimo strumento da utilizzare per studiare le emozioni e le differenze che possiamo trovare tra le diverse culture.\\
Per prima cosa, però, è importante comprendere quali siano le prospettive teoriche che lo riguardano e che ci permettono di comprendere al meglio i concetti che lo definiscono.

\section{Prospettive teoriche del lessico emotivo}
La concettualizzazione delle emozioni secondo una prospettiva universalista o socio-costruttivista ha acceso sempre un grande dibattito tra filosofi, psicologi, scienziati e antropologi nel corso della storia. L'ambito delle emozioni e, conseguentemente quello del lessico emotivo, sono stati indagati sotto l'influenza di epoche storiche ed intellettuali diverse, giungendo a prove a favore di entrambe le prospettive teoriche.

Per quanto riguarda la branca della linguistica \footnote{Come spiegato nel \autoref{par: Semantica}, questo ambito della linguistica si riferisce all'insieme di significati espressi attraverso il linguaggio, che permettono l'accesso alla nostra organizzazione mentale.}, si possono distinguere teorie che vengono suddivise in base a come vengono concepite le categorie linguistiche di ciascuna lingua.\\
Esse possono essere considerate come universali, comuni a tutte lingue: in questo caso i linguisti cercano di  osservare il comportamento e definire le caratteristiche universali che permetto di individuare una categoria universale per tutte le lingue prese in considerazione.\\
Se, al contrario, si pensa che ogni lingua abbia sue categorie specifiche, si parla di particolarismo categoriale e si ritiene che non sia possibile equiparare una categoria di una determinata lingua con la stessa di un altra \parencite{haspelmath}. 

Ciò che è stato messo in discussione nell'ambito del lessico emotivo è lo studio della semantica, all'interno del quale ci si chiede se l'organizzazione mentale del nostro lessico emotivo sia universale e innata, oppure dipendente dalla cultura specifica in cui viviamo \parencite{universal_structure}.

\subsection{Universalismo}
La prospettiva universalista sostiene che le emozioni abbiano un substrato innato, comune a tutti gli individui, indipendentemente dalla cultura a cui appartengono o al contesto sociale in cui sono immersi \parencite{izard_article}.\\
Le emozioni, quindi, sarebbero qualcosa di costante e biologicamente universale, e l'influenza culturale permetterebbe solamente di modificare la loro diversa manifestazione e concettualizzazione, senza intaccare il substrato universale che le compone.\\
Alcuni aspetti universali dell'esperienza emotiva possono essere individuati facilmente e vengono 
infatti ritenuti \textit{trivially true}, ossia banalmente veri: si tratta di automatismi fisiologici in risposta a stimoli emotivi che sono comuni a tutti gli esseri umani. Ad esempio, la paura può essere associata a un aumento del battito cardiaco, dell'afflusso di sangue ai muscoli e ad altre risposte fisiologiche simili in tutte persone, indipendentemente dalla loro cultura \parencite{storia_delle_emozioni}.

L'universalismo emotivo trova le sue origini per la prima volta nel  "darwinismo delle emozioni". Darwin, infatti, sosteneva che le emozioni umane avessero una base biologica comune e fossero il frutto dell'evoluzione, che potessero contribuire al successo riproduttivo e alla sopravvivenza della specie umana \parencite{darwin}.\\
Il pensiero Darwiniano fu ripreso poi da grandi psicologi che diedero vita alla prospettiva universalista delle emozioni. 

Gli studi di maggiore rilevanza, che hanno dato voce al pensiero universalistico delle emozioni sono sicuramente quelli di Paul Ekamn. Egli, attraverso molteplici esperimenti, afferma un principio di universalità delle espressioni facciali e l'esistenza di sei emozioni di base (gioia, tristezza, sorpresa, disgusto, paura e rabbia) che accomunano tutti gli individui, indipendentemente dal contesto sociale, dalla cultura d'origine o dalla lingua parlata.\\ Ekman voleva dimostrare che le differenze culturali non sono rilevanti per quanto riguarda le emozioni primarie di base: incondizionatamente dall'area geografica di provenienza, dallo status sociale o dal livello di sviluppo della popolazione presa in considerazione, tutti gli individui riuscivano a riconoscerle \parencite{ekman}.\\
Seppure il lavoro di Ekman fu successivamente  criticato da psicologi ed antropologi, il suo apporto teorico sulle espressioni facciali e l'esistenza delle emozioni di base ebbe un'incredibile rilevanza nell'ambito della psicologia delle emozioni. 

Molti autori del Ventesimo e Ventunesimo secolo hanno seguito le orme di Ekman, dando maggiore risonanza alla prospettiva universalista, rilevando diverse emozioni di base per spiegare l'esperienza emotiva umana.\\
Alcuni dei più conosciuti sono: Tomkins, il quale rileva la presenza di otto emozioni innate e universali, presenti fin dalla nascita, che fornirebbero una base comune per forgiare il comportamento e l'esperienza umana \parencite{tomkins}; così come Johnson Laird e Oatley, che affermano l'esistenza di cinque emozioni fondamentali, generate dal primo livello di risposta agli stimoli emotivi, predisposto all'adattamento dell'organismo all'ambiente \parencite{Keith_JohnsonLaird} \footnote{Tali livelli sono stati approfonditi nella \autoref{subsec: Teorie cognitive}.}.

Questa prospettiva viene approfondita anche nell'ambito delle neuroscienze, in quanto sono state individuate risposte cerebrali e fisiologiche agli stimoli emotivi comuni a tutti gli individui.\\
Uno degli neuroscienziati più rilevanti per la prospettiva universalista è stato sicuramente Panksepp\footnote{È bene sottolineare che nonostante alcuni suoi apporti teorici siano stati molto importanti per lo studio dell'universalità delle emozioni, le sue ricerche si sono concentrata principalmente sulle emozioni animali e la loro continuità con quelle umane \parencite{panksepp_libro}.}, che, attraverso i suoi studi nell'ambito delle "neuroscienze affettive" \footnote{Questo termine fu coniato dallo stesso Panksepp per indicare un'area di ricerca della scienza riguardante lo studio del cervello interspecie \parencite{panksepp_articolo}.}, ha individuato sette sistemi emozionali primari comuni (ricerca, cura, gioco e piacere, e i corrispettivi negativi di paura, malattia e angoscia). Varie ricerche oltre a dimostrare l'effettiva esistenza di tali emozioni, hanno confermato anche l'ipotesi secondo la quale i loro squilibri sarebbero una causa di disturbi psichiatrici \parencite{panksepp_articolo}.\\
Il sistema emotivo, ritenuto il più basilare, attraverso il quale tutti gli altri si sarebbero sviluppati, è quello della ricerca, ovvero la "voglia di fare", che corrisponderebbe a quella che Izard chiama "interesse" \parencite{panksepp_izard}. 

Carroll Izard è considerato un altro importante pioniere della prospettiva universalista delle emozioni, in quanto sostiene che le emozioni siano innate e che si sviluppino nel corso della crescita a partire dalle prime settimane di vita e per alcuni anni dello sviluppo.\\
Per far comprendere al meglio la non apprendibilità delle emozioni, le paragona a quattro gusti (salato, dolce, aspro e amaro) che siamo in grado di riconoscere in modo spontaneo, istintivo.\\
Un'importante aggiunta che egli propone rispetto ad altri autori, è però la funzione del linguaggio: questo viene ritenuto come qualcosa che può essere appreso, a differenza delle emozioni, ed è ciò che gli individui utilizzano per riferirsi agli stati emotivi. Egli, quindi, afferma che è possibile apprendere a nominare e gestire gli stati emotivi, ma questi ultimi rimangono di per sé un concetto universale ed innato \parencite{izard_intro}. 

Con questa teoria Izard differenzia quindi le emozioni di base, definite come strutture innate, e gli schemi emotivi, più complessi e appresi attraverso l'esperienza e il linguaggio \parencite{izard_schemi_emotivi}. Gli schemi emotivi, infatti, subiscono l'influenza della cultura in cui l'individuo è inserito, e vedono come elemento fondamentale quello della parola: il linguaggio appreso permette di comprendere a fondo l'emozione provata, esplicitandone le cause e le conseguenze, dando agli stati emotivi diversi significati in base ai costrutti culturali che influenzano il nostro modo di sentire.

Izard ci dà la possibilità di iniziare ad esplorare più approfonditamente il ruolo del linguaggio utilizzato nell'ambito emotivo, quindi del lessico emotivo, in relazione alle diverse prospettive teoriche che definiscono le emozioni.

\subsection{Verso il costruzionismo sociale}
Quando il linguaggio guadagna più importanza all'interno dello studio delle emozioni, bisogna iniziare a prendere in considerazione anche la prospettiva sociocostruttivista.\\
Di seguito verranno riportati alcuni autori che, nonostante rimangano legati ad alcuni elementi della prospettiva universalista delle emozioni, iniziano ad introdurre aspetti relativi al costruzionismo sociale, mettendo in evidenza le componenti culturali e sociali che influenzano l'ambito delle emozioni e del lessico emotivo.

Klaus R. Scherer, legandosi ai concetti delle emozioni di base proposti da Ekman, Tomnkins e Izard, introduce il suo pensiero riguardante le emozioni di base, definite da lui come "utilitaristiche", in quanto utili per l'adattamento individuale in risposta a situazioni frequenti \parencite{scherer_ekman}. Considerando gli aspetti di valutazione degli stimoli emotivi e, in particolare la prototipicità \footnote{La prototipicità è un concetto derivato dalla teoria dei prototipi di Eleanor Rosch e si riferisce alla misura della somiglianza di un oggetto ad un determinato prototipo di una categoria. Gli esempi che sono più simili al prototipo sono considerati più prototipici, mentre quelli che sono meno simili sono considerati meno prototipici \parencite{prototipicità}.} di queste emozioni, decide di nominarle "modali" e non "di base". 

Partendo da questa considerazione, Scherer, mette in evidenza la vasta gamma di emozioni presenti negli esseri viventi, che va al di là delle sole emozioni modali (facilmente rilevabili poiché più frequenti). Le molteplici emozioni che possono originare dai modelli di risposta agli stimoli emotivi \footnote {Scherer è di estrema rilevanza nell'ambito della psicologia per la sua teoria dell'\textit{appraisal}: egli considerava fondamentale la valutazione cognitiva degli stimoli emotivi per generare l'emozione. Nella sua teoria del "Modello dei Processi Componenti" vengono descritti i cinque componenti del processo di valutazione che, interagendo tra di loro, costituiscono l'emozione \parencite{teoria_componenti_scherer}.}, però, presentano un grande limite: sono difficilmente misurabili.\\
Una metodologia che lo psicologo svizzero propone, dunque, è quella di indagare le condotte popolari e il lessico emotivo utilizzato. In questo compito risulta fondamentale concentrarsi sulle distinzioni emozionali che il lessico emotivo descrive, cogliendo i processi a cui fanno riferimento i vari termini.\\
Per lo studio di questi aspetti, Scherer formula un esperimento in cui veniva chiesto ai partecipanti di cogliere le modalità in cui una persona tendenzialmente valuterebbe e risponderebbe ad uno stimolo emotivo relativo ad una determinata emozione.\\
Grazie ai riscontri ottenuti si riescono a formulare delle griglie semantiche per i diversi termini emozionali, le quali definiscono il loro campo di significato in ciascuna lingua analizzata. Questi dati permettono di studiare le sottili differenze di significato nei diversi termini emotivi in base alla lingua parlata e, conseguentemente, forniscono informazioni sulle potenziali differenze culturali e linguistiche nella codifica delle emozioni \parencite{scherer}.

Un libro fondamentale, che indaga ancora in maniera più approfondita il tema del lessico emotivo in relazione all'ambito transculturale e alla prospettiva universalista è sicuramente \textit{«Emotions in Crosslinguistic Perspective»} (2001) di Harkins e Wierzbicka. \\
Qui gli autori, parlando dell'universalità delle emozioni affermano che «è molto probabile che il numero di modelli fisiologici [legati alle emozioni] sia limitato e universale, ma che non ci sia universalità nell'esperienza soggettiva corrispondente \parencite{Wierzbicka}».\\
Le emozioni vengono quindi descritte come "universali esperienziali": esse sono legate al principio secondo cui tutti gli individui possono provare un insieme di esperienze soggettive simili tra loro. 

A differenza delle teorie universaliste, però, viene evidenziato che un'emozione non può essere semplicemente considerata universale, poiché la parola associata ad essa ha un significato specifico a seconda della cultura in cui è inserita e a seconda della lingua parlata. Le emozioni non possono dunque riferirsi a significati universali.\\
In sintesi le emozioni umane variano parecchio in base alla cultura e alla lingua, ma hanno anche molti aspetti esperienziali comuni tra loro. 

Per studiare tali somiglianze e differenze del lessico emotivo viene ritenuto necessario un metalinguaggio che possa arginare la grande variabilità data dalle lingue naturali, le quali posseggono una propria "immagine ingenua del mondo", che porta a una visione della realtà differente per ciascuna lingua.\\
Il metalinguaggio definito dagli autori è chiamato \textit{"Natural Semantic Metalanguage (NSM)"} \parencite{Wierzbicka}, attraverso il quale si cercano di analizzare i concetti emotivi universali, collegandoli ad una loro grammatica universale \footnote{Per grammatica universale si intendono le regole innate di combinazione universale dei concetti comuni a tutti gli individui \parencite{Chomsky}.}, partendo da studi linguistici realizzati nel corso degli anni. 

Lo studio degli aspetti universali del lessico emotivo secondo l'approccio del NSM si distanzia dal metodo precedente, proposto da Van Geert, il quale ritiene che solamente un esperto, grazie al linguaggio tecnico riesca a decifrare tutti i componenti dell'esperienza soggettiva emotiva \parencite{Van_Geertz}.

Secondo Harkins e Wierzbicka questa prospettiva sarebbe una forma di etnocentrismo \footnote{Per etnocentrismo si intende la «tendenza a giudicare i membri, la struttura, la cultura, la storia e il comportamento di altri gruppi etnici con riferimento ai valori, alle norme e ai costumi del gruppo a cui si appartiene, per acritica presunzione di una propria superiorità culturale \parencite{etnocentrismo}».} e di scientismo sviante \footnote{Scientismo è un termine che indica la «tendenza a considerare solo la scienza come unica fonte di conoscenza valida e a svalutare altri campi del sapere» e che, come in questo caso, può portare all'errore \parencite{scientismo}».}.
L'utilizzo di un metalinguaggio tecnico da parte di esperti, infatti, non può mettere in luce l'esperienza umana ordinaria e la sua concettualizzazione soggettiva.\\
Al contrario, per fare ciò, gli studiosi devono comprendere come gli individui, considerati "ordinari", pensano e parlano, provando a trovare elementi comuni tra il linguaggio utilizzato dagli studiosi e quello che si intende prendere in analisi. 

Con questo libro, gli autori mettono in discussione la prospettiva universalista, dando voce agli aspetti di differenziazione culturale, introducendone la loro grande importanza nello studio delle emozioni.\\
Viene anche profondamente contestata l'idea dell'universalista Ekman, secondo cui le parole emotive non hanno rilevanza per lo studio delle emozioni in sé, dato che queste ultime sarebbero frutto di processi biologici innati \parencite{Wierzbicka}.

Levy è un altro psicologo nel cui pensiero possiamo trovare un passaggio tra la prospettiva universalista e quella sociocostruttivista delle emozioni.\\
Nella sua teoria le emozioni sarebbero il risultato di processi cognitivi relativi all'elaborazione dello stimolo emotivo, l'interpretazione cognitiva di tale stimolo, la risposta fisiologica e l'espressione comportamentale. \\
Tra le varie emozioni generate dall'interazione di tutti questi processi, l'autore distingue una serie di emozioni, definite primarie e un'altra costituita da emozioni complesse \parencite{levy_culture}.\\
Le prime sarebbero emozioni universali ed innate, come la paura, la rabbia, la tristezza e la gioia; mentre le emozioni complesse risulterebbero da un'elaborazione cognitiva e culturale delle emozioni primarie. 

Levy si è poi principalmente occupato della regolazione emotiva, specialmente in relazione alla psicopatologia. In particolare, ha messo in evidenza come le emozioni complesse, generate attraverso le relazioni sociali, possano contribuire allo sviluppo di problemi psicopatologici \parencite{levy_emotion_regulation}. Egli, quindi, prende in considerazione le relazioni sociali e il contesto culturale per costruire elementi psicoterapeutici che possano aiutare nella regolazione delle emozioni complesse. \\
Nonostante la sua influenza sia stata più nell'ambito clinico e psicoterapeutico, nella sua teoria delle emozioni, possiamo osservare la compresenza di aspetti universalisti e sociocostruttivisti utili allo studio delle emozioni. 

La prospettiva universalista mette da parte gli aspetti culturali e sociali ma, come si è potuto già evincere dagli ultimi autori citati, questi sono fondamentali per studiare al meglio i processi emotivi e, soprattutto, il ruolo del lessico emotivo.\\
Le emozioni differiscono così tanto in base alla  cultura presa in considerazione, che il principio secondo cui tutti gli esseri umani sentono alla stessa maniera, viene fortemente messo in discussione. I principi culturali e sociali verranno infatti presi in considerazioni da molti psicologi, antropologi e linguisti, e costituiranno il nucleo della prospettiva socio-costruttivista.

\subsection{Socio-costruttivismo}
\label{subsec: Socio-costruttivismo}
La prospettiva socio-costruttivista delle emozioni viene concepita come opposta a quella universalista, poiché concettualizza le emozioni come un prodotto del contesto sociale e culturale e non come un qualcosa di innato e comune a tutti gli esseri umani. 

Nella prospettiva socio-costruttivista le emozioni sono completamente influenzate dal contesto in cui l'individuo vive e, dunque, dalle norme, aspettative e pratiche sociali di tale contesto.\\
Le emozioni vengono viste come qualcosa di socialmente appreso, costruite attraverso l'interazione sociale, seguendo regole ben precise definite socialmente e culturalmente \parencite{gergen}.\\
Un altro elemento fondamentale che evidenzia il socio-costruttivismo, che non veniva preso in considerazione nella prospettiva universalista, è la dinamicità delle emozioni: queste possono essere influenzate dalle risposte emotive degli altri e dalle dinamiche relazionali e, quindi, mutare nel tempo.\\
L'enfatizzazione delle dinamiche relazionali e delle interazioni tra individui ci porta a sottolineare il ruolo fondamentale del linguaggio nel socio-costruttivismo e quindi, per quanto riguarderà le emozioni, del lessico emotivo, il quale permette di etichettare e descrivere l'esperienza emotiva, dandole un significato preciso.\\
Per comprendere al meglio la prospettiva socio-costruttivista è bene conoscere il pensiero dei principali autori che descrivono la realtà come frutto della costruzione sociale.

Lo psicologico statunitense Kenneth J. Gergen è considerato uno dei padri di questa corrente di pensiero e mette in luce i suoi aspetti chiave.
Afferma che le esperienze individuali, comprese le emozioni, siano costruite attraverso le interazioni sociali e culturali \parencite{gergen}. \\
Le emozioni, che nell'universalismo erano concepite come interiori ed innate, ora vengono viste come una costruzione attraverso la comunicazione e le relazioni sociali. Esse sono influenzate dai significati condivisi dalla comunità a cui l'individuo appartiene: vengono, dunque, costruite dal linguaggio in base ai significati socialmente condivisi.\\
Di conseguenza le emozioni variano parecchio tra le diverse culture e i contesti sociali in base  all'insieme di tali significati.\\
Inoltre, proprio per questo motivo, le emozioni sarebbero costruite collettivamente (non sono più individuali come nell'universalismo) attraverso le interazioni sociali e le pratiche culturali che influenzano la gamma di emozioni che riconosciamo e le modalità attraverso cui le esprimiamo.

Più o meno negli stessi anni anche il filosofo e psicologo britannico Rom Harrè diede un grosso contributo alla prospettiva del costruzionismo sociale delle emozioni, soprattutto in relazione al linguaggio.\\
Harrè afferma che l'analisi dei differenti vocaboli, utilizzati in un determinato contesto per descrivere le emozioni, sia centrale \parencite{harre}. Studiare come viene usata una determinata parola associata ad un'emozione ci permette di indagare l'emozione stessa nella cultura presa in considerazione.\\  
Harré critica la visione psicologica del lessico emotivo come mera rappresentazione dell'emozione: esso ha la funzione di costruire le emozioni stesse.\\
La visione costruzionista, infatti, vede la realtà come il risultato di un lavoro di organizzazione, ordine ed interpretazione degli eventi esterni, che conduce alla possibilità di agire solamente in base alle risorse linguistiche e alle abilità sociali individuali.
 
Anche la psicologa e neuroscienziata britannica Lisa Felman Barrett sostiene che le emozioni non siano innate ed universali, ma costruite culturalmente.\\
Attraverso le sue ricerche sostiene che le emozioni non vengono innescate da stimoli esterni, ma che siano il risultato dell'interpretazione personale della propria risposta corporea, influenzata dal contesto sociale in cui ci si trova.\\
La neuroscienziata studia approfonditamente la relazione tra i processi cognitivi e le emozioni, dimostrando che le emozioni non sembrano essere localizzate in un area precisa del cervello, ma che smebrano essere generate dall'interazione di molteplici regioni cerebrali \parencite{barrett_neuroscienze}. \\
I processi cognitivi che si sviluppano in queste regioni, però, sono profondamente influenzati dalle norme sociali e culturali, dal contesto in cui ci si trova e dalle relazioni interpersonali presenti tra gli interlocutori \parencite{barrett_costruzionismo}.

Barrett, oltre a cogliere gli aspetti culturali che costituiscono l'esperienza emotiva, mette in luce anche l'importanza dell'etichettatura delle emozioni, ovvero l'assegnazione di determinate parole per riferirsi ad un'esperienza emotiva: l'importanza del lessico emotivo. L'utilizzo di un particolare lessico per descrivere uno stato emotivo influenza la percezione e la regolazione dell'emozione stessa. \\
La psicologa riesce quindi, attraverso le sue ricerche neuroscientifiche, a mettere in evidenza la rilevanza degli aspetti cognitivi relativi alle emozioni, l'influenza culturale sui processi interpretativi e l'importanza del lessico emotivo.

Il campo del costruzionismo sociale ha interessato oltre che psicologi e filosofi anche molti antropologi che, grazie ai loro studi sul campo, hanno apportato valido sostegno a questa prospettiva teorica. 

Verso la metà del Ventesimo secolo l'antropologo Clifford Geertz inizia a pubblicare i suoi studi in cui esprime il suo supporto alla prospettiva costruzionista della realtà. Il suo pensiero si centrava sull'antropologia simbolica, secondo la quale i simboli costruirebbero i significati socialmente condivisi dalla cultura.\\
La cultura, infatti, viene definita come «un sistema di concezioni ereditate, espresse in forme simboliche, attraverso le quali gli uomini comunicano, perpetuano e sviluppano le loro conoscenze e i loro atteggiamenti nei confronti della vita \parencite{geertz}».\\
Geerzt, riprendendo Max Weber, afferma che la cultura sia essenzialmente semiotica \footnote{La semiotica fa parte della scienza della comunicazione e viene definita come «scienza generale dei segni, della loro produzione, trasmissione e interpretazione, o dei modi in cui si comunica e si significa qualcosa, o si produce un oggetto comunque simbolico \parencite{semiotica}».}, e che quindi l'uomo si trovi all'interno dell'insieme di significati che egli stesso ha creato.\\
Questo concetto porta alla convinzione che l'analisi della cultura sia una scienza interpretativa alla ricerca del significato.\\
Conseguente al fatto che la cultura sia costituita da un insieme di simboli e di significati, infatti, viene messo in evidenza il principio guida dell'interpretazione, che permette di comprendere la vita degli individui in una particolare cultura \parencite{geertz}. 

Nella sua concettualizzazione della cultura egli sottolinea che anche le emozioni umane sono manufatti culturali, costituiti da significati condivisi; ma lo studio delle emozioni, in realtà, viene approfondito principalmente dalla moglie Hildred.\\
H. Geertz condusse vari studi antropologici nell'isola di Java, studiandone i processi di socializzazione. Nelle sue produzioni scientifiche riporta la concezione secondo la quale negli esseri umani sono presenti alcune emozioni culturalmente universali \parencite{hildred_geertz}.\\
Queste, però, sono soggette a diversi influssi e condizionamenti nel corso della socializzazione infantile, la quale è variabile a seconda delle diverse culture in cui si è inseriti. A seconda di tale cultura e dalle norme socialmente condivise in essa, infatti, verrebbero promosse alcune emozioni e smorzate altre e, proprio per questo motivo, le emozioni risulterebbero culturalmente diverse. 

Un'altra antropologa che ha trattato il tema delle emozioni a livello culturale e si è preoccupata di mettere in evidenza l'importanza del linguaggio in quest'ambito, è Catherine Lutz.\\
Nei suoi studi concernenti il lessico emotivo, Lutz afferma che le parole emotive non sono universali, ma vengono influenzate dalla cultura e dal contesto sociale in cui ci si trova. Essendo costruite socialmente e culturalmente, le parole che compongono il lessico emotivo, riflettono le norme e i valori condivisi della società in cui vengono utilizzate \parencite{lutz_politics_emotions}.\\
Il contesto culturale, quindi, assume una grandissima importanza, soprattutto per quanto riguarda la comprensione dei processi emotivi attraverso il lessico di riferimento. Quest'ultimo, infatti, può includere parole emotive che descrivono emozioni specifiche che non esistono in altre lingue, o che descrivono le emozioni in modo diverso a seconda del contesto culturale e situazionale differente.

L'antropologa mette in luce la grande importanza del lessico emotivo attraverso le sue molteplici funzioni che può avere nei confronti dei processi emotivi.\\
Il linguaggio utilizzato per descrivere una determinata esperienza emotiva, può anche influenzare la stessa: impiegare specifiche parole in un contesto può condizionare l'esperienza emotiva di chi partecipa a tale contesto sociale.\\
Un'altra funzione fondamentale del lessico emotivo che si riscontra è la regolazione delle emozioni, che può essere raggiunta attraverso l'etichettatura e la condivisione degli stati emotivi con gli altri, attraverso il linguaggio e la comunicazione\footnote{Questa argomentazione di Lutz può essere confermata dagli esperimenti condotti sull'etichettuatura delle emozioni, viste approfonditamente nel \autoref{par: Affect labeling}} \parencite{lutz_cultural_category}.\\
Tutti gli apporti di Lutz descritti finora hanno avuto un impatto significativo sulla comprensione delle emozioni e della loro relazione con il linguaggio e la cultura; fornendo una base teorica solida per lo studio delle parole emotive come un fenomeno culturale e sociale complesso.

L'ultimo ambito di studio che ha preso in carico il lessico emotivo in relazione alla prospettiva socio-costruttivista delle emozioni è sicuramente quello della linguistica. 

Un linguista e filosofo che ha dato un grande apporto allo studio del lessico emotivo è Reddy, il quale si concentra sulla semantica e sulla relazione di questa con la cultura, sviluppando la teoria degli \textit{"emotives"}.\\
Con questo termine, Reddy, rappresenta tutte le parole facenti parte del linguaggio emotivo che, attraverso la costruzione, indicano emozioni o valutazioni soggettive: vengono utilizzate per descrivere ciò che viene definito "indescrivibile", ovvero "come ci si sente" \parencite{reddy_no_costruzionismo}.\\
Importante, inoltre, è la specificazione della differenza tra un uso emotivo e uno descrittivo del linguaggio: si ha un'intenzionalità ben diversa. Quest'ultimo viene utilizzato con l'intenzione di comunicare significati descrittivi, mentre il linguaggio emotivo, proprio delle \textit{emotives}, vuole tradurre a parole un'emozione, con l'intenzione specifica di suscitare nell'altro uno stato emotivo preciso \parencite{reddy_articolo}. \\
Le \textit{emotives}, attraverso l'atto comunicativo, hanno anche la possibilità di costruire, modificare, nascondere o intensificare le emozioni stesse.\\
Da quest'affermazione si può comprendere che il lessico emotivo, attraverso la teoria di Reddy, acquisisce il ruolo di costruttore della realtà sociale.\\
Essendo il lessico costruito sulle basi di una specifica lingua socialmente e culturalmente influenzata, la sua costruzione delle emozioni varierà da cultura a cultura e genererà dunque emozioni non universali, ma socio-costruite. \\
Reddy, grazie ai suoi apporti teorici, ha contribuito allo studio approfondito del lessico emotivo, legato alle variabili culturali, sociali e contestuali che lo influenzano; permettendo uno studio approfondito delle emozioni attraverso il linguaggio. 

Un altro linguista del Ventesimo secolo che tratta il tema del lessico emotivo e, in particolar modo, delle metafore utilizzate nell'ambito emotivo è Zoltan Kovecses.\\
Il lessico emotivo, secondo la sua prospettiva, è un sistema complesso, costituito da metafore, schemi concettuali e processi cognitivi specifici che influenzano l'espressione delle emozioni; si differenzia quindi dagli studiosi secondo cui il lessico emotivo è definito solamente come l'insieme di parole che descrivono gli stati emotivi \parencite{kovecses_articolo}. \\
Egli si concentra molto sullo studio delle metafore, che portano alla comprensione delle emozioni. Per descrivere gli stati emotivi si possono utilizzare più metafore, che vengono identificate come subordinate di una metafora principale, la quale indica i concetti principali di quella determinata emozione.\\
Un esempio che potrebbe aiutare la comprensione di tale concetto sono le metafore legate all'emozione di rabbia. Si possono utilizzare molte metafore per descrivere il concetto di rabbia, come “sbuffare, esplodere, far ribollire il sangue, schiumare”. Tutte le immagini evocate da tali metafore possono ricondurre a sensazioni di calore, pericolo, forte intensità e perdita di controllo. Come si può osservare vi è una metafora principale che è quella della forza, energia\clearpage e violenza, dalle quale possono nascere diverse metafore ad esse subordinate \parencite{kovecses_articolo}. 

Kovecses, partendo dallo studio delle metafore, si interroga sulla dicotomia tra universalismo e socio-costruttivismo delle emozioni \footnote{Dicotomia presa in considerazione proprio in questi capitoli.}, ritenendola poco produttivo.\\
Proprio per questo motivo propone una mediazione tra i due poli opposti, prendendo in considerazione sia gli elementi universali delle emozioni, che quelli costruiti socialmente \parencite{kovecses_libro_metafore}. Secondo la sua teoria gli elementi universali riguarderebbero l'attivazione fisiologica scaturita dagli stimoli emotivi, dunque la dimensione corporea, sulla quale poi verrebbero formulate delle metafore.
La rabbia, ad esempio, viene associata ad un aumento della temperatura corporea, un aumento del battito cardiaco, eccetera. \\
Gli elementi fisiologici determinano l'ambito entro il quale vengono formulate le metafore centrali e universali di tale emozioni, nel caso quella rabbia questa sarebbe della del “contenitore sotto pressione”.

Questi concetti sono stati già menzionati nella descrizione del pensiero di Wierzbicka. Anche la linguista polacca, infatti, cerca di includere sia gli aspetti legati all'universalismo che quelli di matrice socio-costruttivista nella descrizione delle emozioni.\\
Nel suo libro afferma che tutti gli individui possono provare esperienze soggettive simili a livello esperienziale, corporeo, denominando le emozioni proprio "universali esperienziali". Dall'altra parte, però, proprio come Kovecses, dà all'ambito linguistico e semantico il ruolo socio-costruttivista, andando così a contraddire la concezione universalista delle emozioni.\\
I pensieri dei due linguisti trovano grandi analogie generali, ma si può notare come Kovecses si concentri molto più profondamente sugli aspetti corporei, legandoli anche alle metafore linguistiche, mentre Wierzbicka lasci più spazio allo studio semantico e grammaticale del linguaggio trovandone somiglianza e differenze nelle varie culture. 

Tornando allo studio del lessico emotivo di Kovecses, possiamo incontrare molto facilmente l'aspetto del costruzionismo sociale. \\
Secondo il linguista ungherese, infatti, il lessico emotivo viene  utilizzato principalmente in modo figurato e costruito culturalmente: le parole utilizzate per descrivere le emozioni sono sottoposte all'influenza delle esperienze culturali personali e delle norme socialmente condivise, che possono condurre ad una diversa quantità di vocaboli per descrivere la stessa emozione in base alla cultura in cui ci si trova \parencite{kovecses_libro_metafore}.\\
Di grande importanze viene anche ritenuto il contesto situazionale e relazionale in cui le parole vengono utilizzate, che può cambiare il loro significato.

L'importanza delle metafore come strumento per comunicare gli stati emotivi veniva già citata dai linguisti Lakoff e Johnson, negli anni Ottanta. 
Data la grande difficoltà nella descrizione degli stati emotivi, gli autori affermano che l'individuo si serve delle metafore per riuscire ad esprimersi  più facilmente ed in modo indiretto, facendo riferimento ad analogie con altre esperienze.\\
Identificano diversi tipi di metafore tra cui: corporee, di azione, di sostanza e di spazio, facendo riferimento proprio ai fenomeni a cui l'individuo fa riferimento per esprimere le emozioni attraverso le analogie \footnote{Ad esempio, le metafore corporee si riferiscono alle emozioni attraverso i termini relativi alle sensazioni fisiche, come "sentirsi pesanti" \parencite{lakoff_johnson}.} \parencite{lakoff_johnson}.\\
Le metafore sono profondamente influenzate dalla cultura e dal contesto sociale; pertanto si può ricorrere all'utilizzo di diverse metafore per descrivere le stesse emozioni. 

Sempre nell'ambito della linguistica, il tedesco Martin Haspelmath, si occupa dello studio della semantica del lessico emotivo e delle differenze tra le diverse lingue.\\
Il suo pensiero si pone a metà tra la prospettiva universalista e quella sociocostruttivista.\\
Egli, infatti, ritiene che esistano parole emotive universali, presenti in tutte le lingue, ma che la semantica di tali parole e la loro distribuzione (ovvero la quantità di vocaboli disponibili per parlare delle emozioni) subiscano l'influenza culturale che porta ad una loro grande differenziazione in base alla lingua parlata.\\
Viene anche evidenziata la presenza di alcune parole specifiche esistenti solo in determinate lingue, e non in altre \parencite{haspelmath}.\\
Per riuscire a comprendere al meglio l'espressione delle emozioni attraverso il linguaggio, risulta quindi fondamentale analizzare il contesto culturale e situazionale in cui si è inseriti. \\
Seguendo questa corrente di pensiero, Haspelmath, studia la semantica del lessico emotivo in relazione al lessico generale della lingua in questione \parencite{Haspelmath_articolo}.\\
Egli sostiene che le parole utilizzate nella descrizione emotiva possiedono una base semantica simile ad altre parole nella lingua. Tale relazione semantica tra le parole emotive e il lessico generale può portare alla comprensione ed espressione delle emozioni in quella lingua. 

Haspelmath ha dunque contribuito in modo considerevole alla comprensione delle emozioni attraverso il linguaggio, alla semantica di esso e alle differenze del lessico emotivo in relazione al contesto sociale e culturale. 

Attraverso l’esposizione delle diverse teorie sulle emozioni e sul linguaggio emotivo sistematizzate dai diversi psicologi, linguisti, antropologi e filosofi che hanno focalizzato l’attenzione su tale oggetto di indagine, si possono trarre alcune considerazioni generali. 

Le due prospettive teoriche prese in considerazione vengono facilmente viste come due poli teorici opposti: le emozioni possono nascere in concomitanza con l'individuo, come meccanismi cerebrali innati; oppure in un ambiente sociale e culturale, divenendo un prodotto della costruzione socio-culturale.\\
Ciò che si è visto prendendo in considerazione il pensiero dei vari autori, però, è che spesso tali prospettive sono complementari e non contrapposte, risultando così entrambe necessarie per comprendere le emozioni e la loro genesi.\\
La dicotomia tra universalismo e socio-costruttivismo è infatti ad oggi superata, ma ciò che è importante comprendere è in che maniera e misura gli elementi universali e quelli socio-costruiti influenzano i processi emotivi. Si è visto, ad esempio, che gli aspetti legati all'innatismo e comuni a tutti gli individui sembrano appartenere più alla sfera di attivazione fisiologica, mentre quelli che si differenziano da cultura a cultura sembrano essere legati alla sfera semantica delle emozioni. Più approfonditamente il processo di costruzione sociale e culturale è stato fatto proprio del lessico emotivo.\\
Quest'ultimo pare essere di fondamentale rilevanza nell'ambito semantico dei processi emotivi, risultando esso stesso costruito culturalmente e socialmente, ma anche in grado di modificare e costruire le emozioni attraverso il suo utilizzo. \\
Questo è uno dei focus principali che danno origine a vari dibattiti in questo campo: quanto le emozioni siano in grado di essere costruite dal lessico, e quindi dalla cultura e dalla società e quanto siano invece innate ed universali.

Ciò che si è potuto sicuramente comprendere è che le emozioni sono caratterizzate da un'estrema complessità, data anche dalla loro influenza da innumerevoli variabili di tipo biologico, culturale e sociale e, soprattutto, dalla loro difficile misurazione, essendo processi non direttamente osservabili e difficilmente descrivibili verbalmente. 

Partendo da tali considerazioni, il prossimo capitolo cercherà di approfondire gli aspetti socio-culturali legati al lessico emotivo, andando ad indagarne le differenze nelle varie lingue, e analizzandolo come strumento per esplorare per i processi emotivi.

