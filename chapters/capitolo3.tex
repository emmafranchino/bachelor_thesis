\chapter{Differenze culturali del lessico emotivo} 
\label{chap: Capitolo 3}
\section{Studi antropologici}
Ciò che è stato osservato negli studi di matrice socio-costruttivista sul lessico emotivo è che questo si sviluppa all'interno di un contesto sociale e culturale specifico. Di conseguenza, le parole utilizzate per descrivere le emozioni sarebbero legate alle credenze, i valori, le norme e le pratiche di  una specifica comunità e differirebbero notevolmente in base alla lingua parlata.\\
Ad esempio, alcune culture potrebbero avere parole specifiche per descrivere emozioni non presenti in altre, oppure possedere vocaboli che si riferiscono a certe emozioni in modo più preciso e dettagliato o, ancora, le differenze culturali possono influire anche la frequenza con cui le persone esprimono determinate emozioni e sulla loro disposizione a mostrare apertamente le loro emozioni in pubblico.\\
Le norme socio-culturali, quindi, possono influenzare sia il tipo di emozioni che vengono promosse, sia l'intensità con la quale vengono espresse pubblicamente.

Le norme sociali e le differenze culturali che condizionano gli stati emotivi sono state studiate approfonditamente da parte dell'antropologia, che ha dato un grande apporto anche agli studi sul lessico emotivo.\\
Molti antropologi, alcuni dei quali già menzionati nel \autoref{chap: Lessico emotivo}, hanno condotto studi sul campo, immergendosi nella realtà di una specifica comunità per svariato tempo hanno cercato di comprendere al meglio come gli individui vivessero le proprie emozioni, quali fossero quelle prevalenti e i termini utilizzati per descriverle. 

Una di queste è sicuramente l'antropologa statunitense Catherine Lutz, la quale nel 1977 condusse uno studio sui popoli della Micronesia, in particolare sul popolo Ifaluk, abitanti le isole Caroline nel Pacifico. \\
Nella cultura di questo popolo le emozioni risultano essere centrali e vengono integrate alla sfera spirituale e religiosa, quindi sono parte di un sistema di valori più ampio. \\
Per quanto riguarda le emozioni specifiche riscontrate, l'antropologa afferma la forte rilevanza delle emozioni negative, che non vengono controllate o represse come nella cultura occidentale, ma espresse ed accettate. \\
Le emozioni principali del popolo Ifaluk sembrano essere due, denominate \textit{fago} e \textit{song}: la prima corrisponde ad una combinazione di amore, compassione e tristezza e si prova principalmente nel confronti di un individuo che manifesta uno stato di bisogno, è quindi legata al prendersi cura dell'altro, un interesse altruistico verso la comunità. L'emozione \textit{song}, invece, rappresenta una rabbia ingiustificata o indignazione morale contro qualcuno che trasgredisce valori fondamentali condivisi nella comunità. \\
Tali emozioni vengono considerati complementari l'una dell'altra: è necessario provare compassione e interesse nei confronti dell'altro per poi provare indignazione quando tali atteggiamenti di cura altrui vengono trasgrediti o non condivisi dall’intera comunità \parencite{lutz_micronesia}. \\
Attraverso questa ricerca si può notare, oltre che la presenza di diverse parole emotive che rappresentano le emozioni assenti nel mondo occidentale, anche il fatto di trovarvi riflesse le norme di una società all'interno di questi vocaboli, che divengono espressione della diversa esperienza emotiva espressa dal popolo in questione. 

Abu-Lughod, antropologa che ha lavorato con Lutz, ha condotto uno studio sulle donne beduine e sulla loro esperienza emotiva. \\
Il primo elemento che pone in evidenza è che in presenza di uomini, le donne beduine dovevano attenersi a rigide regole di pudore e sottomissione impartite dal codice morale \textit{Hasham}. Questo implicava il divieto di esprimere o la necessità di camuffare le proprie emozioni come ad esempio la gelosia, che però potevano poi esprimere nell'ambito privato, tramite canti carichi emotivamente denominati \textit{Ghinnawa} \parencite{abu_lughod}. \\
Questo aspetto di camuffamento delle emozioni dovuto a norme imposte dalla società in una specifica cultura si rifà al concetto di "lavoro emotivo", ovvero il mascheramento delle proprie emozioni in pubblico per adeguarsi alle regole emanate dal contesto culturale, e la possibilità di esprimerle nel privato. Il lavoro emotivo può essere superficiale, quindi una modificazione a livello espressivo delle emozioni, oppure profondo, che si riferisce ad una modificazione e gestione interna delle emozioni che vengono provate \parencite{hochschild}.\\
Questo ci fa apprendere quanto le norme culturali siano in grado di influenzare sia l'espressione esterna sia la capacità di modificare gli stati emotivi che si provano internamente.

Oltre alla diversa manifestazione delle emozioni dovuta alle norme sociali e culturali si possono notare differenze culturali interlinguistiche per quanto riguarda vocaboli che indicano emozioni specifiche di una comunità, non esistenti in nessun'altra. \\
Per fare alcuni esempi possiamo citare un'emozione identificata in studi antropologici della Papua Nuova Guinea chiamata \textit{awumbuk}, che descrive un senso di stanchezza ed esaurimento dovuto al commiato di un ospite dopo aver trascorso una notte nella propria casa \parencite{fajans}. \clearpage
Levy, nel suo studio sul popolo tahitiano, notò la presenza di alcune parole specifiche per descrivere emozioni di rabbia e odio, che sembrano emozioni molto rilevanti nella loro cultura, ma anche l'assenza di vocaboli per descrivere altre emozioni come quelle di affetto, sentimento o passione \parencite{levy_thaitians}. \\
Oppure, ancora, nella lingua tedesca possiamo trovare la parola \textit{Weltschmerz}, spesso usata in ambito letterario, che significa "dolore del mondo" e sta ad indicare un senso profondo di tristezza e insoddisfazione esistenziale nei confronti della condizione umana e delle sofferenze condivise all'interno di un contesto \parencite{goethe}. \\
L'elenco di questi vocaboli è ampissimo e la loro esistenza sembra dimostrare quanto le emozioni si differenzino in base alla cultura in cui si vive e quanto le specificità culturali e sociali si riflettano nella lingua parlata, quindi nel lessico emotivo. 

Il linguaggio, quindi, diventa un importantissimo strumento per andare ad indagare come gli stati emotivi interni vengano concettualizzati nelle varie culture e per metterli a confronto tra loro. \\
Lo studio del lessico emotivo risulta assai complesso e, per poterlo analizzare al meglio e permettere il confronto tra più lingue diverse, sono state individuate dimensioni affettive specifiche in ciascuna lingua presa in considerazione. 

\section{Modelli dimensionali}
Linguisti e psicologi cercano di analizzare le somiglianze e le differenze delle parole utilizzate per parlare delle emozioni, andando ad indagare se fattori come la vicinanza geografica o la famiglia linguistica possano influenzare tali relazioni. \\
Un metodo utilizzato che permette una dettagliata indagine del lessico emotivo e il confronto tra più lingue risulta essere quello di analizzare le varie dimensioni affettive delle emozioni, ovvero proprietà che forniscono significato e che vengono ritrovate nei termini usati per descrivere le emozioni. Risultano quindi identificabili nell'organizzazione semantica del lessico emotivo e potrebbero essere indice della manifestazione e dell'esperienza stessa delle emozioni.\\
Tali dimensioni, proprie della valutazione affettiva, hanno dato vita a diversi modelli, tra cui i principali definiti tridimensionali o bidimensionali, in base al numero di dimensioni rilevate nell'analisi del lessico emotivo delle lingue prese in esame \parencite{model}. 

Uno dei modelli principali è quello di Russell, noto come teoria circomplessa delle emozioni, che si concentra sulla descrizione organizzativa delle emozioni e sulla relazione tra esse.\\
Il modello circomplesso è bidimensionale: in esso le emozioni prese in considerazione si possono rappresentare in uno spazio circolare, caratterizzato da due assi, che rappresentano due dimensioni bipolari indipendenti. L'asse orizzontale si riferisce alla dimensione edonica, ovvero un continuum tra piacere e dispiacere; mentre l'asse verticale delinea il grado di attivazione, quindi un continuum che va dalla calma all'eccitazione fisiologica o cognitiva.\\
Attraverso questi due assi vengono introdotti due concetti importanti: quello di valenza, che coincide con l'asse orizzontale proprio del grado di piacevolezza e quello di \textit{arousal}, corrispondente all'asse verticale dell'attivazione fisica o mentale \parencite{russell_circumplex}. \\
Secondo Russell gran parte delle emozioni possono essere descritte utilizzando solamente questi due fattori, in quanto la combinazione tra essi sembra dare luogo alle diverse sfumature di significato dell’esperienza emozionale, rappresentabile mediante parole che si collocano nello spazio cartesiano che si struttura attorno a tali assi dimensionali. 

Il modello circomplesso può essere un valido strumento per studiare l'esperienza emotiva umana, analizzata attraverso \textit{self-report} \footnote{Per riuscire a studiare le emozioni umane, lo strumento più facile che possiamo utilizzare è l'intervista diretta. Viene chiesto ai partecipanti come ci si sente e la dichiarazioni fatte, ossia i \textit{self-report}, vengono trattate come comportamenti verbali che possono essere studiati e osservati. Questo processo, però, è molto complesso a causa delle diverse difficoltà comunicative: una domanda chiave è se le differenze nei \textit{self-report} riflettano i sentimenti reali o semplicemente differenze nel modo in cui le persone comprendono le parole usate nel processo di valutazione \parencite{barrett}.}. Nell'analisi di questi ultimi si va ad osservare quali sono le dimensioni che più vengono enfatizzate dall'individuo, si parla quindi di \textit{valence focus} per quanto riguarda la concentrazione sulla dimensione valutativa o \textit{arousal focus} relativa alla dimensione di eccitazione \parencite{barrett}.

Vari autori hanno proposto modelli bidimensionali simili, che includono sempre dimensioni relative all'\textit{arousal} e alla valenza \parencite{Larsen}. Altri, invece, pur prendendo in considerazione due dimensioni, si distinguono da quello di Russell.\\
Per esempio, il modello bidimensionale di Watson e Tellegen individua come dimensioni l'affetto positivo e l'affetto negativo. Tale modello, però, descrive la presenza di due scale unipolari, che sono costituite da una "miscela" degli aspetti di valenza e attivazione \footnote{I punteggi elevati rilevati nella scala dell'affetto positivo rappresentavano stati emotivi positivi con elevato \textit{arousal}; quelli elevati nell'asse dell'affetto negativo invece, corrispondevano a stati emotivi negativi, sempre con elevata attivazione fisiologica \parencite{Watson}.}.\\
Qui, infatti, l'elemento distintivo e piuttosto rilevante è la presenza di due dimensioni ortogonali e non di un'unica scala in cui l'affetto negativo e positivo ne rappresentano i poli opposti. Questo dà la possibilità di rappresentare stati emotivi con una valenza mista, sia negativa che positiva \parencite{Watson}.

Ciascuno di questi modelli bidimensionali porta con sé argomentazioni valide, le quali mantengono vivo il dibattito sul grado di applicazione di ciascun modello alle diverse lingue e culture, che riesce ad ampliarsi ancora di più con i modelli tridimensionali, ovvero quelli che prendono in considerazione tre dimensioni diverse.

La loro storia inizia con Wundt, il quale descrive le dimensioni individuate con lo studio dei termini emotivi: parla di "piacere-dispiacere", che si traduce nella valenza per i modelli bidimensionali; "eccitazione-calma" e "tensione-rilassamento", che rappresentano l'attivazione fisiologica \parencite{wundt}.\\
Come fa notare Thayer, però, c'è una grande differenza tra i due tipi di attivazione fisiologica poiché mentre l'eccitazione energetica può scaturire da uno stimolo molto positivo, quella tensiva può essere frutto di una situazione di stress e tensione molto forte \parencite{Thayer}.\\
Successivamente, uno dei modelli tridimensionali di maggiore rilevanza è stato quello proposto da Schimmack e Grob, che vede come dimensioni principali per descrivere le emozioni: piacere-dispiacere, veglia-stanchezza e tensione-rilassamento, che lo fanno risultare molto simile al modello originale di Wundt \parencite{Schimmack}. 

Modelli storicamente anche precedenti da quello descritto, ma che si concentrano più approfonditamente sull'aspetto della semantica degli stimoli emotivi, vengono presi in considerazione soprattutto nello studio del lessico emotivo.\\
Il modello principale è stato quello di Mehrabian, che presenta tre dimensioni: quella di piacere, di eccitazione e di dominanza \parencite{Mehrabian}. Tale modello, però, deriva da quello proposto da Osgood, che descriveva precisamente il significato affettivo delle parole.

Il modello tridimensionale di Osgood rileva l'esistenza di tre diverse dimensioni che le parole possono esprimere implicitamente o esplicitamente: la valutazione, ovvero il senso di piacere o dispiacere, l'attività, che si riferisce al livello di attivazione fisiologica, e la potenza, che corrisponde alla capacità di affrontare le sfide ambientali da parte dell'individuo \parencite{Osgood}. \\
Tali dimensioni vengono riprese anche da altri autori, tra cui Davitz, il quale parla di tono edonico, corrispondente alla dimensione di piacere o dispiacere \parencite{Davitz}; \textit{arousal} o attivazione per quanto riguarda l'attivazione fisiologica e competenza, che corrisponde alla potenza di Osgood. Egli, inoltre, aggiunge una quarta dimensione: quella della relazionalità, che si riferisce alla relazione con le persone e con l'ambiente, la quale influenza fortemente l'esperienza emotiva. 

Un altro modello utilizzato in ricerche psicologiche più recenti è quello di \textit{scaling} multidimensionale (MDS). In alcuni studi condotti sulla lingua inglese però, si è notato che questo modello ha praticamente replicato le tre dimensioni del modello di Osgood, non aggiungendone altre \parencite{russell_multidimensional}. \\
Attraverso diversi studi si è quindi compreso che le dimensioni strutturali del lessico emotivo che si possono incontrare attraverso la sua analisi sembrano essere principalmente due: la valenza e l'attivazione fisiologica. 

Per concludere, i modelli principali strutturano gli stati emotivi e la loro descrizione attraverso il lessico emotivo includono principalmente due o tre dimensioni. Si è vista inoltre la tendenza per gli autori americani alla creazione di modelli bidimensionali, mentre per gli autori europei di quelli tridimensionali \parencite{model}.\\
Tali modelli vengono realizzati solitamente attraverso self-report, auto dichiarazioni che utilizzano tendenzialmente la lingua inglese e, quindi, fanno riferimento solo a termini inglesi o termini traducibili in inglese per descrivere le emozioni, escludendo l'enorme possibilità di variabilità linguistica legata all'ambito emotivo. Questo conduce ad un dubbio attuale sulla possibilità di un'applicazione di tali modelli a diverse lingue e culture per studiarne le emozioni.\\
Questo dovuto anche alla supposizione di un'influenza da parte di fattori culturali e sociali molto ampia nella struttura dimensionale degli affetti, che deve essere esaminata al meglio.

\section{Variazioni interlinguistiche}
Partendo da queste premesse teoriche si possono quindi andare ad analizzare i risultati di diversi studi che vogliono indagare le differenze dell'ambito emotivo in diverse culture, attraverso l'analisi semantica del lessico emotivo.\\
Verranno prese in considerazione le differenti dimensioni esplicitate dai modelli sopracitati e verranno comparate nelle varie lingue, cercando di comprendere quali siano le analogie che accomunano certi gruppi linguistici e quali le differenze che li distanziano.

Come già accennato, gran parte delle ricerche empiriche condotte in questo campo si riferiscono alla lingua angloamericana o su termini traducibili in inglese, senza prendere in considerazione termini peculiari propri di ciascuna lingua \parencite{russell_multidimensional}. In questi casi i gruppi semantici identificati maggiormente come categorie concettuali discrete, comuni a gran parte delle lingue studiate, si riferiscono a quelle che Ekman chiama "emozioni di base" quindi: rabbia, gioia, sorpresa, paura, disgusto e tristezza \parencite{ekman}. Data la forte influenza della lingua inglese, però, tali studi non possono essere considerati pienamente validi a livello interlinguistico, si ritiene pertanto necessario approfondire la questione mediante  studi transculturali.  

Gli studi transculturali del lessico emotivo danno vita a diversi modelli dimensionali, mettendo in luce le dimensioni più rilevanti nelle diverse lingue prese in esame; quindi le dimensioni che ciascuna lingua permette di identificare. \\
I riscontri ottenuti attraverso diversi studi hanno portato a galla interessanti considerazioni. \\
Innanzitutto si è potuto notare che non prendendo in considerazione la lingua inglese le dimensioni individuate differiscono dalle sole valenza ed attivazione fisiologica, considerate generalmente le più rilevanti \parencite{Galati}. \\ 
Ciò che si è visto, infatti, è che la dimensione edonica è l'unica ad essere presente in tutte le lingue studiate: piacere e dispiacere sembrano essere imprescindibili per la descrizione delle emozioni, in alcuni casi è presente solamente quest'asse, ad esempio nella lingua ebraica \parencite{Fillenbaum}. 
La dimensione di potenza, invece, è stata identificata come dimensione rilevante solamente in alcuni casi, per esempio, studiando la lingua italiana si è visto che questa dimensione risulta più importante rispetto a quella dell'eccitazione fisiologica \parencite{Sini}. \\
Ciò non toglie la possibile compresenza di tutte e tre le dimensioni di potenza, attivazione e valenza, come è stato riscontrato in alcuni studi della lingua spagnola.

Uno studio specifico sulla famiglia linguistica delle lingue neolatine, ovvero: italiano, francese, castigliano, catalano, portoghese e rumeno \parencite{Sini}, ci permette di approfondire ancora di più l'argomento, dandoci così la possibilità di afferrare meglio la misura delle variazioni del lessico emotivo. \\
Lo studio di trentadue termini emotivi per ciascuna lingua neolatina ha permesso di rilevare diverse dimensioni emotive. Le dimensioni riscontrate come comuni a tutte le lingue analizzate sono la valenza edonica, quindi la presenza di termini linguistici che indicano una distinzione tra emozioni negative e positive, che danno dunque un senso di piacere o dispiacere; e l'attivazione fisiologica, data la presenza di parole che indicano una bassa o alta eccitazione comportamentale. \\
Alcuni esempi, per quanto riguarda la valenza possono essere i termini italiani felice/infelice, quelli in castigliano \textit{euforico/triste}, in catalano \textit{alegre/frustrat} e così via. Per quanto riguarda la dimensione dell'attivazione, invece, si possono incontrare diversi vocaboli che indicano stati di agitazione, ossia una forte attivazione fisiologica e, in contrapposizione, altri che si riferiscono a stati di calma, quindi a basse attivazioni fisiologiche, ad esempio i termini francesi \textit{agité/ désolé}, oppure quelli portoghesi \textit{agitado/acanhado} o ancora in rumeno \textit{infuriat/melancolic}. \\
Sugli assi delle dimensioni della valenza edonica e dell'attivazione fisiologica i termini sopracitati si pongono ai rispettivi poli opposti: per esempio, in castigliano \textit{euforico} si colloca al polo che rappresenta il punteggio più alto della valenza edonica, mentre \textit{triste} all'esatto polo opposto di tale asse.

L'ultima dimensione del modello tridimensionale esaminata è quella che si riferisce alla potenza, che, a differenza delle due precedenti, non è stata ritrovata in tutte e sei le lingue neolatine \parencite{Sini}.\\
La dimensione di potenza, infatti, può essere chiaramente individuata in italiano, rumeno e francese attraverso la presenza di termini che indicano irritazione, rabbia o disgusto, come il francese \textit{fâché}, l'italiano "irritato" o il rumeno \textit{aprins}. Questi termini si trovano in opposizione alle emozioni di significato contrario, che possono indicare spavento o paura come \textit{effrayé} in francese oppure "pauroso" in italiano.\\
In castigliano e catalano questa dimensione viene riconosciuta come un insieme delle dimensioni di potenza e attivazione fisiologica: mentre in castigliano troviamo parole come \textit{abochornado} o \textit{asustado}, in catalano incontriamo \textit{atemorit} o \textit{horrorit}, termini che indicano stati emotivi di spavento e terrore. Infine, in portoghese, la dimensione di potenza sembra ricondurre alla valutazione dell'«adeguatezza del comportamento alle proprie norme interne \parencite{Sini}», ad esempio, attraverso il vocabolo \textit{orgulhoso} o \textit{envergonhado}, che indicano orgoglio o vergogna. 

Questa dimensione e quella dell'attivazione fisiologica variano parecchio nelle diverse lingue, con alcune di esse in cui risulta più rilevante la dimensione dell'eccitazione rispetto a quella delle potenza, e viceversa. Tali dimensioni, inoltre, sembrano fare riferimento a diverse strategie di adattamento dell'individuo all'ambiente: modalità con cui vengono affrontate le sfide ambientali e i cambiamenti fisiologici e psichici. \\
Partendo dalle differenze sopraelencate riguardanti queste differenze, quindi, è possibile fare delle inferenze su come gli individui che utilizzano una specifica lingua affrontino le sfide ambientali a loro sottoposte. \\
Considerando il catalano e il castigliano, ad esempio, si può osservare che i termini che si riferiscono alle dimensioni di attivazione fisiologica indicano i meccanismi di \textit{coping}, andando a distinguere reazioni aggressive e comportamenti di fuga, contrapposte ai comportamenti di riposo \parencite{Sini}. 

Attraverso quest'analisi dettagliata delle sei lingue neolatine e delle dimensioni emotive presenti in esse, si è concluso che per tale studio il modello tridimensionale di Osgood risulta il più adeguato. In questo caso, infatti, sono fondamentali le dimensioni di valenza ed attivazione, ma anche quella riferita alla potenza, che gioca un ruolo di rilievo. \\
Ciò che si è anche notato è che mentre la dimensione di valutazione edonica risulta sempre la più importante, le altre due presentano diversa salienza nelle sei lingue. \\
In italiano, francese, catalano e castigliano, la seconda dimensione più rilevante è la potenza, seguita dall'attivazione; mentre in rumeno e in portoghese la situazione è opposta, vedendo come seconda l'attivazione fisiologica \parencite{Sini}. 

Ciò che è interessante indagare sono le cause di tali somiglianze nelle lingue indicate: una probabile spiegazione è quella della distanza geografica, vedendo che le lingue più simili sono anche le più vicine geograficamente. \\
Inoltre, le lingue neolatine che presentano una predilezione per la dimensione della potenza hanno un'origine latina e greca, ovvero lingue nate da culture che pongono al centro l'attività cognitiva e quindi meno l'attivazione fisiologica. Quest'ultima considerazione andrebbe quindi a sostenere una causa di tipo storico, sociale e culturale: secondo questa spiegazione sarebbero la cultura greca e latina con le loro norme sociali ad influenzare il lessico emotivo. 

Dopo aver visto un'indagine su una famiglia linguistica specifica, si andrà ora ad approfondire uno degli studi interculturali più ambiziosi, che analizza il lessico emotivo di 20 famiglie linguistiche.\\
Si tratta dello studio è di Jackson e altri psicologici, i quali presentano una ricerca interlinguistica che prende in considerazione le reti semantiche di oltre un terzo delle lingue del mondo, mostrando la grande diversità con cui i concetti di emozione vengono espressi nelle diverse culture \parencite{majid_jackson}.\\
Per analizzare il modo in cui i concetti di emozione sono collegati tra loro, gli autori utilizzano quelle che vengono chiamate "colessificazioni", ovvero i casi in cui una singola parola si riferisce a più concetti: nella lingua persiana, ad esempio, non esistono due parole distinte per indicare "dolore" e "rimpianto", ma una soltanto \textit{ænduh}, che li indica entrambi \parencite{jackson_joshua}. Il metodo che implica lo studio delle colessificazioni risulta parecchio favorevole per ricavare informazioni sulla struttura semantica del lessico emotivo, poiché queste vengono spesso riscontrare quando due parole vengono percepite come concettualmente simili. 

Andando ad analizzare le reti semantiche di tali colessificazioni in 2.474 lingue, gli autori hanno riportato la presenza di una grandissima variazione semantica nei concetti di emozione. I vocaboli riferiti alle emozioni sembrano avere diversi modelli di associazione a seconda delle famiglie linguistiche indagate. Ad esempio, prendendo in considerazione le lingue austroasiatiche \footnote{La famiglia delle lingue austroasiatiche comprende 169 lingue, parlate nel sud-est asiatico e India \parencite{austroasiatiche}.} e quelle tai-kadai \footnote{Si tratta di una famiglia linguistica, composta da circa 90 lingue utilizzate nell'indocina, ovvero nell'Asia sud-orientale.}, viene riportato che mentre nelle prime il concetto di "ansia" è strettamente correlato al "dolore" e al "rimpianto"; nelle seconde questo viene maggiormente legato alla "paura". 

Le variazioni riscontrare riguardanti le dimensioni delle emozioni appartengono soprattutto a due: la valenza e l'eccitazione; queste due dimensioni, infatti, sono considerate «i vincoli universali alla variabilità della semantica delle emozioni \parencite{jackson_joshua}», ciò che più differenzia il lessico emotivo nelle diverse lingue parlate. \\
Ciò che viene osservato in questo studio è la presenza di una maggiore eccitazione fisiologica e mentale nelle culture individualiste, quindi principalmente nel mondo occidentale; in contrapposizione, le culture orientali, culturalmente collettiviste, prediligono una bassa eccitazione. \\ 
Prendendo come esempio la felicità si è potuto rilevare che, mentre nelle culture occidentali questa viene associata all'essere ottimisti, presentando la dimensione dell'eccitazione in grande misura; nelle culture orientali alla felicità viene assegnato il significato di solennità e riservatezza, che è associata ad una minore eccitazione fisiologica. \\
La dimensione della valenza presenta i due poli opposti di valenza positiva e negativa: ciò che si è riscontrato nello studio di colessificazione è la quasi totale assenza di emozioni con valenza positiva e negativa nella stessa comunità di colessificazione; con alcune eccezioni peculiari, che possono far comprendere ancora meglio la vastità di differenze del lessico emotivo che si possono trovare. \\
Ad esempio, in alcune lingue austronesiane vengono uniti i concetti di "pietà" e "amore": questo significa che, rispetto ad altre lingue, queste possono presentare una concettualizzazione di "amore" più negativa oppure di "pietà" più positiva \parencite{jackson_joshua}.

La globale differenza delle dimensioni di attivazione e valenza nella maggior parte delle culture occidentali ed orientali, confermerebbe la presenza di pattern culturali simili in base alla vicinanza geografica delle lingue.\\ Famiglie linguistiche più vicine, infatti, tendono a colessificare i concetti di emozione in modi più simili rispetto a quelle geograficamente più lontane.

Questa deduzione apre diverse questioni sulle cause di tale somiglianza basata sulla posizione geografica. Alcune ipotesi potrebbero essere la presa in prestito di diversi concetti tra lingue geograficamente vicine, o la presenza di un antenato linguistico comune. Questa seconda ipotesi interpretativa parte dall'assunto che lingue molto simili tra loro, probabilmente posseggono una lingua primordiale comune, che condurrebbe alla presenza di un lessico emotivo affine. Ad esempio, l'inglese e il tedesco sono entrambi lingue germaniche, quindi tutte e due originate dalla lingua proto-germanica; è possibile, quindi, trovare molte somiglianze nel lessico utilizzato per descrivere concetti emotivi e, conseguentemente, nella rilevanza delle diverse dimensioni. 

Altri fattori che possono aver influenzato tali analogie linguistiche potrebbero anche riguardare il commercio, le conquiste territoriali, i flussi migratori. Gli aspetti storici e sociali avrebbero quindi influito parecchio sulla concettualizzazione delle emozioni: questi aspetti incoraggiano la ricerca sull'analisi dei processi di trasmissione verticale ed orizzontale, che conducono alle variazioni nella semantica delle emozioni \parencite{majid_jackson}. \\
Tale spiegazione sosterrebbe il pensiero costruzionista, che propende per una spiegazione della natura delle emozioni come qualcosa di appreso, che conduce alla loro differenziazione interculturale. 

Lo studio di Jackson et al., oltre a mettere in luce l'aspetto dell'influenza geografica sul lessico emotivo utilizzato, pone al centro della discussione la presenza di concetti preesistenti a cui le parole fanno riferimento a cui si giunge mediante studi linguistici che analizzano termini polisemici o monosemici\footnote{In linguistica è presente un forte dibattito in cui alcuni linguisti si trovano a favore dell'analisi di significato in termini di polisemia, quindi prendendo in considerazione concetti multipli, oppure di monosemia, quindi un'analisi effettuata solamente attraverso concetti unitari \parencite{majid_jackson}.}. \\
Ciò che viene messo in discussione è l'esistenza di concetti universali ed innati data la grande variazione culturale che si è riscontrata studiando le molteplici lingue e la presenza di molte colessificazioni.\\
Un punto cruciale di questo dibattito riguarda la presenza di singole parole che si riferiscono a più concetti: ci si domanda, infatti, se nelle lingue che possiedono un solo vocabolo per indicare due concetti esistano effettivamente due concetti distinti o uno soltanto. Per riprendere l'esempio fatto in precedenza, sul vocabolo persiano \textit{ænduh}, ci si domanda se effettivamente gli individui che parlano questa lingua conoscano i concetti di dolore e rimpianto a cui la parola fa riferimento o se, al contrario, per loro esista solo l'insieme di questi due concetti, racchiusi in \textit{ænduh} \parencite{jackson_joshua}. \\
Ci si domanda, quindi, se diversi modi di parlare di emozioni cambino effettivamente il modo in cui le persone vivono tali emozioni.\\
Lo studio di Jackson et al., attraverso l'analisi di colessificazione, ha potuto creare delle reti che descrivono come gli individui utilizzano il lessico emotivo, cercando di esplicitare i meccanismi evolutivi, biologici e culturali che permettono di attribuire significati alle parole che si riferiscono alle emozioni in ciascuna lingua. 

Ciò che bisogna andare ad indagare sono proprio i fattori sociali, culturali, evolutivi, per andare a comprendere quanto effivamente influiscano sulle emozioni e sulla creazione del lessico emotivo. Inoltre, bisogna comprendere quanto quest'ultimo sia in grado di modificare il sentire umano interno. \\
La domanda più difficile che gli studiosi in questi ambiti si pongono è la possibilità dell'esistenza di un'emozione che non possiede un vocabolo che la rappresenta. \\
Insomma, la semantica delle emozioni, il significato che posseggono le parole che le descrivono, e la loro correlazione con l'esperienza emotiva, è un dibattito antico, che rimane ancora oggi attuale nella letteratura scientifica. 

Grazie a questi importantissime ricerche, però, si può afferrare al meglio la vastità di variazioni interlinguistche per quanto riguarda il lessico emotivo e approfondire la relazione tra i vocaboli utilizzati per descrivere gli stati emotivi, la manifestazione delle emozioni e il sentire emotivo interno a ciascun individuo.

