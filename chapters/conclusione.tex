\clearpage
\chapter{Conclusione}
Il presente elaborato permette di andare a scoprire ed esaminare un ambito molto importante della psicologia delle emozioni, che per svariato tempo è stato trascurato: il lessico emotivo. Questo argomento si ritiene molto interessante per diversi motivi: la descrizione delle emozioni attraverso le parole dà loro significato e permette la loro comunicazione; inoltre risulta fondamentale per un'indagine approfondita delle emozioni e la possibilità di portare a termine ricerche interculturali, che mettono a confronto le emozioni in diverse culture. 

Per addentrarsi negli aspetti culturali del lessico emotivo è essenziale conoscere le prospettive teoriche che lo circondano, andandone a cogliere gli elementi fondamentali. \\
L'universalismo permette di individuare emozioni di base, comuni in tutte le culture, che ogni individuo sembra essere in grado di riconoscere, indipendentemente dal luogo in cui vive, la lingua che parla e la società in cui è inserito. \\
La prospettiva socio-costruttivista, invece, ha idee completamente diverse e sostiene che le emozioni vengano socialmente costruite, attraverso norme e valori della cultura in cui si nasce e cresce, rendendole un qualcosa che può essere appreso e normato dalla società. \\
Un elemento che aggiungono alcuni autori appartenenti alla corrente socio-costruttivista è il ruolo del lessico emotivo: le parole che si usano per descrivere le emozioni sembrano essere il mezzo capace di dare loro significato, e quindi di costruirle. Data la costruzione e l'apprendimento delle emozioni in uno specifico contesto culturale, le norme socio-culturali presenterebbero un ruolo importantissimo: influenzerebbero completamente la costruzione delle emozioni in ciascuna cultura.\\
Dopo aver visto alcuni dei principali autori che contribuiscono alle teorie universaliste e socio-costruttiviste, si sono cercate di analizzare varie differenze e analogie del lessico emotivo in diverse culture, che possiamo trovare attraverso il confronto di varie lingue e famiglie linguistiche. Per fare ciò va indagato l'aspetto semantico del lessico emotivo, che dà significato alle parole, andando a rappresentare stati emotivi specifici. \\
Attraverso i risultati degli studi condotti, si sono andati a delineare dei modelli dimensionali, a seconda del numero di dimensioni riscontrate nei termini emotivi di ciascuna lingua. 

Le conclusioni ottenute dalle molteplici ricerche interlinguistiche sono assai varie e possono dare luogo ad interessanti discussioni. \\
Sono state riscontrate moltissime variazioni sia per quanto riguarda le dimensioni del lessico emotivo, che per la presenza di termini emotivi unici, esistenti in una sola lingua; ma anche parecchie somiglianze. Si è visto, infatti, che nella maggior parte delle lingue analizzate è possibile identificare le dimensioni di attivazione fisiologica e valenza: questo indicherebbe la possibile presenza di elementi innati delle emozioni, presenti in tutte le culture e rappresentati in tutte le lingue. \\
Si è notato, inoltre, che le molte lingue risultanti simili tra loro sono geograficamente vicine: tali analogie, quindi, potrebbero essere date da antenati linguistici comuni, eventi storici e quindi norme sociali e culturali simili. 

Analizzando, invece, le differenze emerse nel lessico di culture distanti si possono fare diverse ipotesi: tali diversità linguistiche potrebbero essere la rappresentazione delle regole e valori di una determinata cultura, le quali influenzerebbero fortemente la manifestazione emotiva. Ad esempio, se in una determinata lingua si hanno molti termini per descrivere emozioni di rabbia si può dedurre che in quella cultura la rabbia sia un'emozione importante, la cui manifestazione è accettata e promossa; al contrario, la mancanza di termini che indicano altre emozioni potrebbero rappresentare una non tolleranza sociale di esse. \\
Le regole e i valori che normerebbero la manifestazione o la soppressione di certe emozioni potrebbero essere dovuti a fattori storici, geografici o religiosi.

Un'altra ipotesi che è possibile effettuare riguardo le variazioni interlinguistiche sarebbe quella proposta dai socio-costruttivisti, quindi l'idea secondo la quale è proprio il lessico emotivo il mezzo con il quale si costruiscono le emozioni. \\
In questo caso, quindi, sarebbero proprio le parole, con la loro funzione semantica, a far nascere emozioni diverse e a condizionare anche il modo di sentire interno agli individui. Ricorrendo all'esempio precedente, secondo questa prospettiva, sarebbe proprio una parola specifica rappresentante un sentimento di rabbia a farla provare all'individuo. 

Il dibattito, dunque, è molto vasto e può prendere direzioni diverse in base ai dati conseguiti dalle molteplici ricerche. Ciò che viene dimostrato è sia la presenza di elementi universali, che di un'incredibile variazione interlinguistica: sono le cause di tali fattori ad essere messe in continua discussione. \\
Ricerche future saranno utili per mantenere tale dibattito aperto: con una società come quella odierna in continuo cambiamento, gli aspetti culturali, linguistici ed emotivi andranno indagati sempre di più. Elementi sociali rivoluzionari come la digitalizzazione potrebbero portare a risultati nuovi, che potranno favorire una certa prospettiva piuttosto che un'altra. 
\clearpage