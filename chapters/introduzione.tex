\chapter{Introduzione}

L'elaborato di tesi che si andrà ad esporre avrà come tema centrale il lessico emotivo e le differenze culturali che si possono riscontrare attraverso un suo studio approfondito. \\
L'intento è stato quello di unire un tema psicologico così complesso e vasto come quello delle emozioni con quello della linguistica, esplorando i principi comunicativi fondamentali e gli aspetti socio-culturali del linguaggio delle emozioni. 

Prima di affrontare il tema del lessico emotivo nello specifico, però, si sono volute dare le basi teoriche per comprendere i processi emotivi. \\
Nel primo capitolo, infatti, si andranno a definire le emozioni, proponendo un piccolo excursus storico, per poi esporre le principali teorie che spiegano al meglio i processi cognitivi e fisiologici che le compongono. Tali teorie, infatti, saranno suddivise in due sottosezioni: da una parte si avranno le teorie neurofisiologiche, attraverso le quali si riuscirà ad approfondire l'aspetto più neuroscientifico, ovvero le regioni cerebrali che coinvolgono i processi emotivi. Dall'altro lato, invece, verranno esposte le teorie cognitive, che accentuano l'interesse per i processi cognitivi ed interpretativi degli stimoli emotivi. 

Nella seconda parte del primo capitolo si farà una divagazione sui principi della linguistica, andando a differenziare il linguaggio e la lingua, per poi, successivamente, esplicare le principali funzioni linguistiche, utili a comprendere al meglio i processi comunicativi. In questa sezione, inoltre, ci si soffermerà particolarmente sulla semantica, fondamentale per comprendere gli studi sul lessico emotivo.\\
Per addentrarsi ancora di più nel mondo della linguistica, soprattutto per quanto riguarda le differenze culturali tra le diverse lingue che poi si andranno ad indagare, si propone una breve spiegazione delle famiglie linguistiche. \\
Infine, il primo capitolo si chiuderà con alcuni eventi storici, culturali e di ricerca che vanno ad enfatizzare il legame tra emozioni e linguaggio, iniziando a far comprendere la profonda connessione tra questi due macro argomenti. 

Nel secondo capitolo, ci si addentrerà nel vivo del tema, andando a definire il lessico emotivo e esponendo le principali prospettive teoriche che lo caratterizzano. \\
Quindi, dopo aver spiegato che cosa effettivamente sia il lessico emotivo, verranno indagati i principali pensieri degli autori che si incontrano nella prospettiva universalista e socio-costruttivista delle emozioni e di coloro che si situano a metà tra le due. 

Infine, nell'ultimo capitolo, verranno prese in considerazione le differenze culturali delle emozioni, che si possono individuare analizzando il lessico emotivo nelle diverse lingue. \\
Innanzitutto verranno esposti alcuni studi antropologici che permettono di comprendere l'enorme varietà delle emozioni e del lessico emotivo che si può riscontrare nelle diverse parti del mondo. \\
Successivamente, verranno spiegati i modelli dimensionali del lessico emotivo, esaminando quelli secondo i quali la struttura del lessico emozionale risulta tridimensionale e quelli che ritengono sia bidimensionale, modelli che provengono da metodologie di ricerca che hanno tenuto conto delle differenze culturali. \\
Il capitolo si concluderà con l'esposizione di alcuni studi che hanno esaminato il lessico emotivo in diverse famiglie linguistiche trovando vastissime differenze culturali, proponendo interessanti spunti per dibattiti futuri. 


