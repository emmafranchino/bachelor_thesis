% Tipo di documento. L'uso di twoside implica che i capitoli inizino sempre con la prima pagina a sinistra, eventualmente lasciando una pagina vuota nel capitolo precedente. Se questa cosa è fastidiosa, è possibile rimuoverlo.
\documentclass[a4paper, twoside, openright]{book}

% Margini
\usepackage[a4paper, top=2.5cm, bottom=2.5cm, left=2.5cm, right=2.5cm, centering]{geometry}
% head=21.75 pt
% head=15.75 pt

% Lingua del testo
\usepackage[italian]{babel}
% Codifica del testo
\usepackage[utf8]{inputenc} 
% Encoding del testo
\usepackage[T1]{fontenc}
% Dimensione del font
\usepackage[fontsize=12pt]{scrextend}
% Typeface
% Times New Roman (simile)
\usepackage{mathptmx}
% \usepackage{tgtermes}
% Interlinea
\linespread{1.5}
% Uso delle immagini
\usepackage{graphicx}
% Posizionamento delle immagini
\usepackage{float}
% Ruotare le immagini
\usepackage{rotating}
% Formattazione titoli delle sezioni, capitoli etc.
\usepackage{titlesec}
% Inserirmento hyperlinks tra i vari elementi del testo 
\usepackage{hyperref}
% Introduzione
\usepackage[nottoc]{tocbibind}
% Citazioni "in blocco"
\usepackage{csquotes}
% Uso dei colori
\usepackage[dvipsnames]{xcolor}
% subsubsections appaiono in Index
\setcounter{tocdepth}{3}
% subsubsection numerata
% paragraph non numerato
\setcounter{secnumdepth}{3}
% dimensione rientro paragrafo
\setlength{\parindent}{19pt}

% Bibliografia con biblatex (https://ctan.org/pkg/biblatex?lang=en)
\usepackage[bibstyle=authoryear, citestyle=apa, dashed=false, backend=biber, block=space, language=italian]{biblatex}
% Path file delle entries
\addbibresource{chapters/archive.bib}

% A capo con gli url
\appto{\bibsetup}{\raggedright}
% Rimuove In: nella bibliografia
\renewbibmacro{in:}{}
% Definisce \urlseen
\newcommand{\urlseen}[1]{\normalfont{#1}}
% \urlseen da URL: a In:
\DeclareFieldFormat{url}{%
  \iffieldundef{doi}{%
    \iffieldundef{url}{#1
    }{%
      {\urlseen{In:}}\addspace%
      \href{#1}{\nolinkurl{\thefield{url}}}%
    }%
  }{%
    \href{http://dx.doi.org/\thefield{doi}}{\urlseen{DOI: \thefield{doi}}}%
  }%
}
% Parentesi dopo il link
\DefineBibliographyStrings{italian} {urlseen = {Consultato il}}

% Modificare l'header delle pagine 
\usepackage{fancyhdr}

% Librerie matematiche
\usepackage{amssymb}
\usepackage{amsmath} % AMS Math Package
\usepackage{amsthm} % Theorem Formatting
% Altre librerie utili
\usepackage{subfig}
\usepackage{booktabs}
\usepackage{caption}
\usepackage{titling}

% ------------------------------------------------------------
% Stile link
% Quando usato con hyperref porta all'errore dell'anchor
% Formato delle intestazioni
% Inizio all'inizio della pagina
\titleformat{\chapter}
{\normalfont\huge\bfseries}{\thechapter.}{0.5em}{\huge}
% cmd, ls, before s, after s, r
% \titlespacing*{\chapter}{0pt}{40pt}{30pt}
\titlespacing*{\chapter}{0pt}{-40pt}{25pt}

% Paragrafi
\titleformat{\paragraph}
{\normalfont\normalsize\bfseries}{\theparagraph}{1em}{\normalsize}
\titlespacing*{\paragraph}
{0pt}{3.25ex plus 1ex minus .2ex}{1.5ex plus .2ex}

% Stile riferimenti
% Off per stampa
% \hypersetup{
%     colorlinks,
%    linkcolor=CornflowerBlue,
%    citecolor=CornflowerBlue
%}
\hypersetup{
    colorlinks,
    linkcolor=black,
    filecolor=black,
    urlcolor=black,
    citecolor=black
}

% Stile header
\pagestyle{fancy}
\fancyhf{}
\lhead{\rightmark}
\rhead{\textbf{\thepage}}
\fancyfoot{}
% \setlength{\headheight}{12.5pt}
\setlength{\headheight}{15pt}
% No numero pagina a inizio capitolo
\fancypagestyle{plain}{
  \fancyfoot{}
  \fancyhead{}
  \renewcommand{\headrulewidth}{0pt}
}

% ------------------------------------------------------------
\begin{document}
\frontmatter 

\selectlanguage{italian}

\begin{titlepage}
\begin{figure}[H]
    \centering
    \includegraphics[keepaspectratio=true,scale=0.2]{images/logo_univda.png}
\end{figure}

\begin{center}
  {\LARGE{UNIVERSITÀ DELLA VALLE D’AOSTA}}
  \vspace{3mm}
  \\{\LARGE{UNIVERSITÉ DE LA VALLÉE D’AOSTE}}
  \vspace{9mm}
  \\ {\large{DIPARTIMENTO DI SCIENZE UMANE E SOCIALI}}
  \vspace{9mm}
  \\ {\Large{Laurea Triennale in Teorie e Tecniche psicologiche}}
  \vspace{2mm}
  \\ {\large{Classe L-24}}
\end{center}

\vspace{15mm}
\begin{center}
  {\Large{\bf{IL LESSICO EMOTIVO E LE DIFFERENZE IN RELAZIONE ALLA LINGUA PARLATA}}}
\end{center}
\vspace{40mm}

\begin{minipage}[t]{0.47\textwidth}
  {\large{\bf{Relatrice:}} \\ Prof.ssa Barbara SINI}
  \vspace{0.5cm}
  %{\large{\bf \\Correlatore:}\\ Dott.ssa Frida MILELLA}
\end{minipage}\hfill\begin{minipage}[t]{0.47\textwidth}\raggedleft
  {\large{\bf Candidata:} \\Emma FRANCHINO\\}
    {\large{Matr. 20 D03 207}}
\end{minipage}

\vspace{25mm}
\hrulefill
\\\centering{\large{ANNO ACCADEMICO 2022-2023}}
\end{titlepage}


\vspace*{\fill}
\begin{flushright}
    \textit{A chi viaggia con me \\
    A chi mi abbraccia quando torno \\
    A chi non mi fa mai sentire sola \\}
\end{flushright}  
\vspace*{\fill}

    

   





% \include{chapters/abstract.tex}

\tableofcontents

\chapter{Introduzione}

L'elaborato di tesi che si andrà ad esporre avrà come tema centrale il lessico emotivo e le differenze culturali che si possono riscontrare attraverso un suo studio approfondito. \\
L'intento è stato quello di unire un tema psicologico così complesso e vasto come quello delle emozioni con quello della linguistica, esplorando i principi comunicativi fondamentali e gli aspetti socio-culturali del linguaggio delle emozioni. 

Prima di affrontare il tema del lessico emotivo nello specifico, però, si sono volute dare le basi teoriche per comprendere i processi emotivi. \\
Nel primo capitolo, infatti, si andranno a definire le emozioni, proponendo un piccolo excursus storico, per poi esporre le principali teorie che spiegano al meglio i processi cognitivi e fisiologici che le compongono. Tali teorie, infatti, saranno suddivise in due sottosezioni: da una parte si avranno le teorie neurofisiologiche, attraverso le quali si riuscirà ad approfondire l'aspetto più neuroscientifico, ovvero le regioni cerebrali che coinvolgono i processi emotivi. Dall'altro lato, invece, verranno esposte le teorie cognitive, che accentuano l'interesse per i processi cognitivi ed interpretativi degli stimoli emotivi. 

Nella seconda parte del primo capitolo si farà una divagazione sui principi della linguistica, andando a differenziare il linguaggio e la lingua, per poi, successivamente, esplicare le principali funzioni linguistiche, utili a comprendere al meglio i processi comunicativi. In questa sezione, inoltre, ci si soffermerà particolarmente sulla semantica, fondamentale per comprendere gli studi sul lessico emotivo.\\
Per addentrarsi ancora di più nel mondo della linguistica, soprattutto per quanto riguarda le differenze culturali tra le diverse lingue che poi si andranno ad indagare, si propone una breve spiegazione delle famiglie linguistiche. \\
Infine, il primo capitolo si chiuderà con alcuni eventi storici, culturali e di ricerca che vanno ad enfatizzare il legame tra emozioni e linguaggio, iniziando a far comprendere la profonda connessione tra questi due macro argomenti. 

Nel secondo capitolo, ci si addentrerà nel vivo del tema, andando a definire il lessico emotivo e esponendo le principali prospettive teoriche che lo caratterizzano. \\
Quindi, dopo aver spiegato che cosa effettivamente sia il lessico emotivo, verranno indagati i principali pensieri degli autori che si incontrano nella prospettiva universalista e socio-costruttivista delle emozioni e di coloro che si situano a metà tra le due. 

Infine, nell'ultimo capitolo, verranno prese in considerazione le differenze culturali delle emozioni, che si possono individuare analizzando il lessico emotivo nelle diverse lingue. \\
Innanzitutto verranno esposti alcuni studi antropologici che permettono di comprendere l'enorme varietà delle emozioni e del lessico emotivo che si può riscontrare nelle diverse parti del mondo. \\
Successivamente, verranno spiegati i modelli dimensionali del lessico emotivo, esaminando quelli secondo i quali la struttura del lessico emozionale risulta tridimensionale e quelli che ritengono sia bidimensionale, modelli che provengono da metodologie di ricerca che hanno tenuto conto delle differenze culturali. \\
Il capitolo si concluderà con l'esposizione di alcuni studi che hanno esaminato il lessico emotivo in diverse famiglie linguistiche trovando vastissime differenze culturali, proponendo interessanti spunti per dibattiti futuri. 




\mainmatter

\chapter{Emozioni e linguaggio}
\section{Definire le emozioni}
Per arrivare a parlare del lessico emotivo è fondamentale, come prima cosa, dare una definizione di "emozione". 
Nonostante questo termine sia usato nella vita quotidiana di tutti noi, pochi saprebbero dare una definizione di esso; molto esaustiva è una citazione di Fehr e Russell che recita «tutti sanno cos'è un'emozione fino a che non si chiede loro di definirla» \parencite{russel_fehr}. Ciò è dovuto alla storia molto lunga e travagliata che interessa l'ambito emotivo.

L'interrogativo su cosa sia un'emozione viene esplicitato dal famoso articolo di W. James \textit{«What is emotion?»}, nel quale sostiene che «i cambiamenti corporei seguono direttamente la percezione del fatto eccitante, e che la nostra sensazione degli stessi cambiamenti nel momento in cui si verificano è l'emozione \parencite{james}».\\
Seppur la produzione di James sia assolutamente rivoluzionaria e ci dia grandi conoscenze sulle emozioni, dobbiamo ricordaci che rimane una tesi formulata per la prima volta da uno psicologo. 

Lo studio delle emozioni ha da sempre coinvolto molteplici settori di studio \parencite{storia_delle_emozioni}: dalla teologia, la filosofia, la retorica, alla medicina, per poi avere ancora più risonanza, a partire dal 1860, nella psicologia sperimentale e, negli ultimi decenni, nelle neuroscienze e scienze della vita.\\
Dunque, è anche importante capire cosa si intenda con il termine "emozione" nelle discipline che si sono occupate del loro studio: per esempio, secondo le teorie neuroscientifiche, sappiamo che l'emozione è considerata qualcosa di puramente corporeo, pre-verbale e inconscio; cosa che non si può di certo dire in una concettualizzazione delle emozioni dei primi stampi filosofici \footnote{Aristotele fu uno dei primi filosofi che parlò di emozioni, indicandole con il nome \textit{"pathos"} ovvero "passione". Egli le concettualizzava come risposte naturali e razionali ad eventi esterni e, dunque, credeva che fossero strettamente legate alla ragione e all'intelletto umano. Già da questa prima visione possiamo comprendere quanto le visioni delle emozioni possano differenziare in base alla disciplina di studio e al momento storico \parencite{aristotele}.}.

Un'altra questione fondamentale, che rende il compito di definizione sempre più difficile, è la lingua in cui sono state esplicitate: è interessante capire, per esempio, se il pensiero di uno psicologo francese e quello di uno scienziato britannico che si assomigliano molto, intendano effettivamente la stessa cosa.\\
Bisogna prestare attenzione non solo ai termini usati, ma ai significati ad essi attribuiti, per capire come, effettivamente, vengano concettualizzati; soprattutto attraverso delle traduzioni \footnote{Nei capitoli successivi verranno trattati più approfonditamente questi aspetti, concentrandoci in modo specifico sulle influenze culturali, che hanno condotto ad una vera e propria materia di studio da parte di antropologi e psicologi nel corso della storia.}.

Già da questa breve descrizione di alcuni dei principali aspetti che vanno considerati, possiamo capire quante definizioni differenti siano state postulate nel corso delle storia, secondo culture e discipline molto distanti. E del perché, ancora oggi, non si sia arrivati ad un accordo comune. \\
Dunque, nonostante non si abbia ancora una definizione univoca di emozione, è possibile comunque distinguerla da altri termini, con cui spesso viene confusa nella conoscenza comune.\\
Le caratteristiche che più contraddistinguono le emozioni da sentimenti, stati d'animo e affetti sono sicuramente la loro breve durata e la loro forte intensità.

Avendo specificato questo aspetto fondamentale, possiamo citare una definizione che troviamo sui dizionari, per avere un'idea più chiara di cosa si intenda nel complesso per emozione all'interno della disciplina psicologica, e indagare, poi,  altre proprietà essenziali: 
\begin{quote}
    «emozióne s. f. [dal fr. émotion, der. di émouvoir "mettere in movimento" sul modello dell’ant. motion] [...]. In psicologia, il termine indica genericamente una reazione complessa di cui entrano a far parte variazioni fisiologiche a partire da uno stato omeostatico di base ed esperienze soggettive variamente definibili (sentimenti), solitamente accompagnata da comportamenti mimici \parencite{definizione_emozioni}». 
\end{quote}

Leggendo attentamente si può capire che la definizione delle emozioni è strettamente legata al concetto di "attivazione fisiologica".\\
Quest'ultima, in gergo psicologico scientifico, viene chiamata \textit{arousal}, ed è una reazione corporea che sconvolge lo stato omeostatico, di equilibrio, del nostro organismo.\\
L'emozione viene, infatti, spiegata in termini di risposta complessa a degli stimoli (immaginari o reali) che può condurre a modificazioni corporee (come l'accelerazione del battito cardiaco, un'eccessiva sudorazione, ecc.). Queste portano poi a pattern d'azione specifici, che l'individuo metterà in atto, come, ad esempio, il principio \textit{fight or flight} ossia combattere o scappare davanti ad una minaccia \parencite{Lazarus}.\clearpage
Affinché l'individuo possa generare questa reazione è necessaria anche una fase di \textit{appraisal}, ossia una valutazione cognitiva delle modificazioni fisiologiche e della natura dello stimolo. 

Questi aspetti sono stati indagati profondamente nell'ambito scientifico psicologico, creando molteplici discussioni e mantenendo grande rilievo durante il corso dei decenni.\\
Le teorie principali generate da questi concetti vengono appunto denominate come: teorie dell'\textit{arousal} o teorie neurofisiologiche, per via del loro focus centrato prevalentemente sull'attivazione corporea, e teorie dell'\textit{appraisal} o teorie cognitive per la loro concentrazione sugli aspetti valutativi, che coinvolgono i processi cognitivi \parencite{come_funzionano_le_emozioni}.

\subsection{Teorie neurofisiologiche} 
\label{subsec:Teorie neurofisiologiche}
Le principali teorie dell’\textit{arousal} che, come sopra accennato, vengono chiamate “teorie neurofisiologiche” sono la teoria periferica di James-Lange e la teoria centrale di Cannon-Bard. Nella prima si afferma che le emozioni coincidono esattamente con le reazioni corporee associate ad esse: gli stati soggettivi che proviamo in risposta ad uno stimolo corrispondono meramente alla percezione cosciente dell’attivazione fisiologica \parencite{james}.\\
Le evidenze sperimentali hanno generato forti critiche verso questa teoria. Le principali sono state mosse da Cannon, il quale sosteneva che una stessa modificazione corporea può essere associata ad emozioni diverse e che, anche inducendo particolari reazioni fisiologiche, gli individui possono provare uno stato emotivo diverso da quello teoricamente associatogli \parencite{cannon}.\\
Bard riprende le critiche avanzate da Cannon proponendo quello che verrà denominato il 
"modello centrale" \parencite{cannon_bard}, chiamato così poiché localizzava i centri di attivazione emozionale a livello centrale, ovvero nel cervello e, più in specifico nell’area talamica.\\
In questo modello, seppure sia di tipo neurofisiologico, l’aspetto cognitivo acquisisce rilevanza e viene presa in considerazione l’importanza dell’interpretazione valutativa dell’individuo. Questa risulta, infatti, fondamentale per dare significato all’attivazione fisiologica e per generare uno stato emotivo soggettivo.

Secondo queste due teorie il sistema nervoso autonomo si attiverebbe in modo diverso a seconda degli stimoli ad esso presentati, come postulato da James-Lange; o, al contrario, per Cannon-Bard, verrebbe prodotto lo stesso pattern di attivazione simpatica, indipendentemente dagli stimoli.\\
Ad oggi, l'evidenza sperimentale, riporta che nessuna delle due teorie sia pienamente concorde con ciò che realmente succede a livello fisiologico. Si ritiene che la percezione dello stimolo generi le emozioni e che produca risposte automatiche e somatiche, ma che anche l'esperienza stessa delle emozioni sia in grado di influenzare questi due fattori. 

Il pensiero dei teorici dell'\textit{arousal} può essere riassunto secondo alcuni principi. Innanzitutto viene affermato che lo stato di attivazione fisiologica è soggetto all'interpretazione a seconda della situazione vissuta.\\
Inoltre viene specificato che l'attivazione emotiva avviene solamente se l'organismo è in uno stato di attivazione emotiva e se a questa non è possibile attribuire una spiegazione diversa \parencite{psicobiologia}.

\paragraph{Neurofisiologia}
\label{par: Neurofisiologia}
Si considera importante, a partire dall’input dato dal modello centrale Cannon-Bard, riportare una breve descrizione delle aree cerebrali coinvolte profondamente negli stati emotivi. 

Questo argomento è di fondamentale rilievo per la psicologia delle emozioni. A seconda del momento storico e dalla cultura in cui è stato sviluppato il pensiero per teorizzare le emozioni sono state messe in luce varie aree deputate alla gestione dei processi emotivi.\\
Un aspetto molto interessante è che a seconda della sede ritenuta centro dei processi emotivi da una determinata cultura, si possono individuare delle ricadute importanti anche sul campo linguistico \parencite{storia_delle_emozioni}): i termini utilizzati per descrivere determinate emozioni o espressioni corporee emotive si concentreranno infatti  su quel determinato organo.\\
Il passaggio della sede delle emozioni da un organo all’altro ha sempre avuto cause legate a questioni storiche e scientifiche.\\
Per fare alcuni esempi possiamo partire dagli antichi Egizi, i quali vedevano come centro delle emozioni il cuore che, di conseguenza, veniva conservato accuratamente all’interno di un vaso dopo la morte.\\
Oppure, secondo il pensiero del popolo di Tahiti, studiato dall’antropologo Foster, la sede principale delle emozioni come la rabbia, il desiderio e la paura era l’intestino \parencite{foster}: questa localizzazione si può ricondurre ad un’influenza Biblica oppure ai movimenti viscerali provocati dalle emozioni stesse.

Arrivando invece a modelli più recenti, a cui facciamo in parte affidamento ancora oggi, verranno di seguito spiegate le teorie più importanti nate nel XX secolo.\\
Tra il XIX e il XX secolo si imposero le ricerche dei primi neuroscienziati, che motivarono lo studio delle emozioni in un ambito cerebrale: il focus di studio delle emozioni a partire da quel momento diventa quindi il cervello, le sue aree specifiche e le connessioni tra esse.

In accordo con il modello centrale\footnote{Modello centrale descritto nelle teorie neurofisiologiche: \autoref{subsec:Teorie neurofisiologiche}.}, negli anni Trenta del Novecento, il neuroanatomista J. Papez, fu il primo ad avanzare l’ipotesi che ci fossero più strutture cerebrali, organizzate in un sistema, deputate al controllo del comportamento emozionale; in seguito nominato 'circuito di Pepez'.\\
Questo comprenderebbe diversi nuclei interconnessi (giro cingolato, ipotalamo, neuroni anteriori talamici e ippocampo), disposti ad anello intorno al talamo \parencite{papez}.\\
Successivamente questo insieme di strutture venne denominato da P. MacLean 'sistema limbico', all'interno del quale venne proposta l'aggiunta al circuito di Pepez di alcune aree ritenute fondamentali per le emozioni, come l'amigdala e la corteccia prefrontale \parencite{maclean}. 

L'amigdala ad oggi viene automaticamente associata al comportamento emotivo grazie all'importanza che ha acquisito nel tempo nell'ambito neuropsicologico.\\
La sua storia inizia a partire dagli anni Trenta del XX secolo, quando viene considerata come la struttura cerebrale responsabile del comportamento emotivo, data la sua funzione di attivare il sistema simpatico e rispondere a degli stimoli minacciosi, generando principalmente emozioni negative, in particolare la paura \parencite{goleman}.\\ 
Venne infatti anche dimostrato che alcune lesioni in quest'area provocavano cecità psichica (ossia la sindrome di Kluver-Bucy), la quale consiste nell’incapacità di riconoscere il significato emotivo di eventi, e quindi anche la tendenza ad avvicinarsi ad oggetti che normalmente evocherebbero paura \parencite{psicobiologia}.

La funzione fondamentale dell'amigdala viene poi ritenuta quella valutativa: essa, infatti, si occuperebbe di valutare gli stimoli captati dagli organi di senso, dando loro una connotazione di 'buono' o 'cattivo'.\\
La valutazione eseguita dall'amigdala viene successivamente inviata al giro del cingolo e alla corteccia prefrontale, che apporteranno una valutazione più accurata, e chiameranno in gioco le funzioni più cognitive.

Questo processo viene meglio descritto da J. LeDoux, nel 1996, con la 'doppia via' di elaborazione degli stimoli, secondo la quale il nostro cervello ha due vie per reagire ad una situazione potenzialmente pericolosa.
La prima è la via bassa: qui gli organi di senso traducono le percezioni in stimoli elettrici, i quali arrivano al talamo e, in seguito, direttamente all'amigdala. È la via più rapida, quindi con una generazione di risposte meno dettagliate.\\
Nella seconda, la via alta, gli stimoli elettrici dopo essere arrivati al talamo, vengono inviati prima alla corteccia e, solo successivamente, all'amigdala. Questo la rende la via più lenta, ma anche più precisa, data l'elaborazione aggiuntiva delle informazioni \parencite{ledoux}.

Nonostante questa teoria sia stata di estrema importanza nel campo delle neuroscienze, ad oggi non si ritiene più corretta, in quanto non sono presenti delle effettive differenze temporali nell'elaborazione degli stimoli da parte di due vie differenti.\\
In definitiva il ruolo dell'amigdala, secondo le ultime scoperte neuroscientifiche, sarebbe quello di trasmettitore tra le varie aree cerebrali implicate nei processi emotivi, grazie al quale i diversi impulsi vengono inoltrati al cervello ed ai centri di azione \parencite{psicobiologia}.

\subsection{Teorie cognitive}
\label{subsec: Teorie cognitive}
Le teorie neuroscientifiche hanno permesso di portare a galla elementi fondamentali, introdotti dalle più antiche formulazioni delle teorie psicologiche sulle emozioni. Ora verranno invece illustrate teorie delle emozioni più recenti, definite come "teorie cognitive", che si ritengono molto importanti per gli argomenti che verranno trattati nei  capitoli successivi.\\
Tali teorie vengono anche definite dell’\textit{appraisal} e partono dal presupposto che le emozioni derivino da processi cognitivi, interni all’individuo \parencite{appraisal}.

La prima che si vuole segnalare è la teoria cognitivo-attivazionale o bi-fattoriale di Schachter e Singer. Il suo nome conduce ad una teorizzazione delle emozioni a due fattori: quello cognitivo e quello biologico.
La teoria bi-fattoriale costituisce una grande innovazione grazie alla spiegazione della natura delle emozioni, a cui si giunge attraverso esperimenti formalizzati messi a punto per indagare i processi fisiologici \footnote{Schachter e Signer nel loro importantissimo esperimento utilizzarono un'iniezione di adrenalina per capire se l'attivazione fisiologica acquisisse significato nel contesto in cui si crea. Nonostante critiche dal punto di vista etico, i risultati confermarono che è la percezione-cognizione dello stimolo contesto o la sua rappresentazione cognitiva a costituire la genesi dell’emozione e a determinarne l'intensità \parencite{schachter_singer}.}, ma soprattutto quelli cognitivi.\\
Questa teoria, infatti, dà una grandissima importanza al processo cognitivo di valutazione: viene specificato che il soggetto attribuisce un valore di attivazione allo stimolo presentatogli e, in secondo luogo, assegnerà ad esso un particolare significato emotivo.
Secondo questa teoria sarà la rappresentazione cognitiva prodotta in base allo stimolo e al contesto che darà vita all'emozione stessa, con intensità diversa \parencite{schachter_singer}.

Un'altra teoria cognitiva delle emozioni, che può avvicinarci sempre di più alla comprensione del tema del lessico emotivo, obbiettivo della presente trattazione, è quella postulata da Johnson-Laird e Oatley.\\
Nella loro teoria viene messo in rilievo il piano comunicativo e, dunque, l'importanza del linguaggio nei processi emotivi. L'approccio degli autori è quello di costruire una teoria emotiva collegata, oltre che all'elaborazione cognitiva, alle teorie computazionali del linguaggio \footnote{Le teorie computazionali fanno parte di una branchia della linguistica che si occupa di adattare il linguaggio naturale ai sistemi tecnologici, quali i computer \parencite{teorie_computazionali}.}.

Per descrivere le emozioni, nel loro articolo, gli autori iniziano dividendo le loro funzioni su due livelli comunicativi.\\
In primo luogo, le emozioni sono viste come una forma di comunicazione interna, definita non propositiva, la quale ha la funzione di priorizzare gli obbiettivi di azione e mantenerne tali priorità.\\
Sul secondo livello, invece, le emozioni sono viste come una forma di comunicazione esterna: qui nascono le emozioni complesse, generate in corrispondenza di piani sociali.\\
Questo tipo di comunicazione è proposizionale, ciò significa che è composta da segnali proposizionali, i quali corrispondono a schemi di chiamata che riescono ad invocare funzioni situate al livello inferiore (non proposizionale), diverse rappresentazioni del mondo, messaggi e funzioni che possono portare alla costruzione di nuove procedure.\\
Questo tipo di comunicazione si distingue da quello non proposizionale, proprio del primo livello descritto, che è molto più semplice, grezzo e antico dal punto di vista evolutivo.\\
I segnali non proposizionali, infatti, non posseggono una struttura simbolica e significativa per il sistema cognitivo. Il loro funzionamento è puramente casuale e immediato e servono per agire velocemente portando il sistema in una particolare "modalità di emozione" \parencite{Keith_JohnsonLaird}.\\
Tali modalità di emozioni, considerate primitive ed innate, vengono interpretate grazie alle valutazioni propositive, generando così specifiche emozioni. 

Questa teoria si differenzia da altre apparentemente molto simili poiché le emozioni complesse non vengono generate dall'insieme di più emozioni di base, in questo caso di modalità di emozioni, ma nascono dalla valutazione cognitiva, grazie ai segnali proposizionali.\clearpage
Le emozioni, quindi, nascono grazie al sistema comunicativo generato dai segnali delle stesse emozioni. Un primo livello, funzionante senza dati proposizionali, può dare vita immediatamente a modalità di emozione, successivamente valutate cognitivamente, su un livello più alto.\\
Per concludere le emozioni cono definite come «stati cognitivi che coordinano processi quasi autonomi nel sistema nervoso \parencite{Keith_JohnsonLaird}».

Questa teoria ci permette di introdurre la linguistica, disciplina, i cui argomenti ci serviranno per comprendere al meglio il lessico emotivo: protagonista assoluto di questo elaborato. 

\section{Introduzione alla linguistica}
Prima di porre il focus sul lessico emotivo e sull'approfondimento delle correnti psicologiche che danno voce ad esso, si ritiene importante spiegare brevemente le caratteristiche principali del linguaggio in senso stretto.

La disciplina che ci permette di analizzare accuratamente il linguaggio e i suoi risvolti nell'ambito psicologico e antropologico è la linguistica. Essa, infatti, è lo studio scientifico del linguaggio verbale umano e delle strutture che lo compongono; ma anche delle lingue in relazione ad un momento storico preciso ed alla loro cultura di riferimento \parencite{introduzione_linguistica}.\\
Data questa esplicitazione relativa alla linguistica, è bene anche chiarire la distinzione tra linguaggio e lingua. 

Il linguaggio è definito come «un sistema di comunicazione che consiste in suoni, parole e grammatica \parencite{langauge}».\\
Rappresenta, dunque, l'insieme di fenomeni della comunicazione che troviamo all'interno delle interazioni umane e al di fuori di esse (possiamo trovare forme di linguaggio anche negli animali o nelle macchine). \\
La lingua, invece, è un modo in cui si manifesta il linguaggio, determinato dal punto di vista storico e spaziale; un sistema attraverso il quale gli individui di una determinata comunità riescono a comunicare tra loro \parencite{lingua}.

Il linguaggio viene quindi considerato come un prerequisito per la lingua: è ciò che permette di creare i sistemi comunicativi, ovvero le lingue. É costituito anche da abilità e processi cognitivi molto complessi e difficilmente localizzabili nei circuiti cerebrali, che rendono particolarmente difficile il suo studio.\\
Ogni lingua, inoltre, ha una propria storia ed evoluzione e si distingue nettamente dai linguaggi artificiali (come, ad esempio, i segnali stradali).

La relazione tra lingua e linguaggio ci permette anche di fare chiarezza sulle componenti naturali e culturali della linguistica.\\
La parte considerata di natura biologica nell'essere umano è quella del linguaggio, mentre la lingua è considerata la componente culturale. Le lingue vengono studiate in relazione all'ambiente sociale e culturale in cui si trovano.

La predisposizione al linguaggio verrebbe dunque intesa come innata ed universale, trasmessa geneticamente, mentre la lingua come apprendibile, trasmessa per contatto, a seconda degli elementi ambientali e sociali che ci circondano \parencite{chomsky_predisposizione_linguaggio}. \\
In altre parole potremmo dire che il linguaggio è ciò che accomuna gli individui, mentre la lingua è ciò che li differenzia. Essendo la lingua relativa al contesto in cui viene utilizzata, essa presente diverse caratteristiche peculiari. \\
Innanzitutto possiede un ciclo di vita proprio: nasce, muta e muore. Nasce in una cultura specifica, attraverso le differenti influenze sociali. Si sviluppa come una proprietà mutevole: cambia a seconda dello spazio, del tempo, della formalità della situazione, dell’argomento, del rapporto che si ha con l’altro. Infine, se non viene più parlata, quindi se non viene più mantenuta in vita mediante un suo utilizzo, muore \parencite{lingue_e_linguaggio}.

\subsection{Funzioni linguistiche}
La linguistica, nel suo ampio ambito di studio, prende in considerazione tutti gli aspetti del linguaggio, che si possono dividere in interni ed esterni.\\
Quelli interni si riferiscono allo studio della morfologia, della fonologia, del lessico e della sintassi. Per quanto riguarda quelli esterni, invece, intendiamo lo studio degli aspetti comunicativi del linguaggio. 

La funzionalità principale della lingua, è proprio quella comunicativa, che permette agli individui di interagire con il mondo esterno e produrre, a livello interno, pensieri e ragionamenti.\\
Le funzioni della lingua, dunque, sono: comunicare con gli altri, parlare a noi stessi, pensare, formulare ragionamenti, produrre nuove idee e raccontarle \parencite{fondamenti_linguistica}.

Le funzioni linguistiche vengono definite con precisione dal linguista russo Roman Jakobson. Egli ne distingue sei: funzione emotiva o espressiva, conativa, referenziale, fàtica, metalinguistica e poetica.\\
Quella più rilevante per la comprensione del lessico emotivo è sicuramente la funzione emotiva o espressiva che si preoccupa di esprimere le emozioni, gli stati d'animo, gli atteggiamenti del mittente. Centrale è proprio il ruolo di quest'ultimo, che utilizza tutti gli elementi grammaticali declinati in prima persona, diventando il protagonista assoluto del racconto. In questo caso risulta quindi fondamentale la capacità di sapere esprimere le proprie emozioni e riuscire a parlare di sé \parencite{Jakobson}.

Ognuna delle funzioni linguistiche appena citate, risultano quindi fondamentali per la creazione di processi comunicativi e, infatti, ad ognuna di essere corrisponde ad una variabile comunicativa \parencite{Jakobson}. \\
Le variabili comunicative sono le parti di cui si compongono i processi comunicativi, che vanno analizzate per comprendere il funzionamento globale della comunicazione stessa \parencite{linguistica_comunicazione}.\\
I fattori comunicativi individuati da sono: il codice, per il quale si intende un «sistema di segni teso a trasmettere informazione tra un mittente e un ricevente, per il tramite di un messaggio \parencite{fondamenti_linguistica}», è ciò che  che permette effettivamente di formulare il messaggio, e può corrispondere, ad esempio, ad una lingua specifica. Il messaggio, ovvero ciò che si vuole comunicare all'altro attraverso il codice, è quindi il contenuto proprio dell'atto comunicativo. L'emittente, cioè colui che vuole comunicare il messaggio e il ricevente che, oppostamente, è colui a cui è rivolto il messaggio. Il canale, che rappresenta il mezzo che mette in relazione l'apparato utilizzato dell'emittente e quello del ricevente. Infine, il contesto, con il quale si intende l'insieme degli elementi di un testo messi in correlazione fra loro. Il contesto può anche essere considerato come lo sfondo della situazione di cui si sta parlando \parencite{fondamenti_linguistica}.

\paragraph{Semantica} 
\label{par: Semantica}
Una delle componenti principali che interessa lo studio della comunicazione, sulla quale è bene soffermarsi per comprendere molti aspetti del lessico emotivo, è la semantica.\\
Con semantica si intende «l'analisi e studio del linguaggio dal punto di vista del significato \parencite{semantica}». L'argomento di studio che concerne la semantica si occuperebbe quindi del rapporto tra significante e significato; considerando proprio la definizione di significante come: «elemento formale, fonico o grafico, del segno linguistico, a cui corrisponde l'elemento concettuale, detto significato \parencite{significante}».\\
La semantica, dunque, può riferirsi alla rappresentazione mentale che abbiamo di un’entità, oppure a quella stessa entità nella realtà.\\
L'insieme di significati che costituiscono le rappresentazioni mentali di ogni individuo si differenziano in base alla lingua parlata, a causa di diverse variabili, quali: la cultura in cui ci troviamo, il livello intellettuale degli interlocutori a cui ci rivolgiamo, l'ambito concettuale entro il quale ci stiamo esprimendo (per esempio se ci troviamo in ambito cinematografico, giuridico, ecc.) \parencite{lessico_semnatica}. 

Lo studio della semantica risulta quindi fondamentale per comprendere l'altro e rispondere adeguatamente in un processo di interazione.\\
Ci permette anche di andare ad indagare attentamente le differenze semantiche che troviamo nelle diverse lingue. Una stessa parola, infatti, può avere significati differenti in lingue diverse a causa dei fattori contestuali, ed è proprio per questo motivo che bisogna porre moltissima attenzione al processo di traduzione.  

Per riassumere e sottolineare nuovamente l'importanza del linguaggio e della semantica possiamo riflettere sul fatto che è proprio questo che permette all'individuo di costruire pensieri e ragionamenti: a seconda della vastità del vocabolario che una persona possiede, potrà generare pensieri più o meno articolati e avrà la possibilità di farsi capire in differente misura dalle persone che lo circondano. È dunque il linguaggio ciò che rende possibile comunicare con l'altro e creare legami interpersonali.\\
Il linguaggio ha il grande potere di unire o separare le persone: un significato diverso che si attribuisce ad una parola o frase può portare ad incomprensioni e divergenze, così come comprendere il linguaggio altrui può instaurare coesione ed intesa.

\subsection{Famiglie linguistiche}
Un altro elemento che è interessante da prendere in considerazione per comprendere successivamente le differenze culturali del lessico emotivo, è l'analisi delle lingue in base alla loro locazione spaziale e le relazioni tra esse.\\
Le molteplici analogie e differenze che troviamo tra le miriadi di lingue utilizzate nel mondo sono proprio dovute a vicinanza o lontananza rispetto al luogo in cui hanno avuto origine, o a cause storiche come, ad esempio, i processi di colonizzazione, che hanno portato allo spostamento di popoli da una parte all'altra del mondo, come pure all'espansione della propria lingua e cultura. 

Elementi comuni a diverse lingue hanno permesso di raggrupparle in diverse famiglie linguistiche, che ci permettono di comprendere al meglio le loro similitudini.\\
Le famiglie linguistiche sono un insieme di lingue che hanno un 'antenato comune', ovvero che sono nate da una lingua comune (ad esempio dal latino), chiamata 'protolingua' \parencite{Family_of_language}.\\
Ad oggi vengono identificate circa 150 famiglie linguistiche, alcune più grandi di altre, all'interno delle quali si localizzano molteplici lingue. Data la grande vastità di lingue che possiamo trovare, è possibile anche dedurre che alcune di esse saranno più simili di altre.\\
Bisogna tenere conto che esistono anche i gruppi, i sottogruppi, i rami: una grandissima vastità di suddivisioni. \\
La famiglia linguistica più diffusa di tutte è quella Indo-europea, che si compone di nove rami tra cui possiamo citare, tra quelli più diffusi in Europa, le lingue romanze, celtiche, germaniche e baltiche: i rami più comuni in Europa \parencite{fam_linguistiche}. 

Queste suddivisioni sono un argomento di indubbio interesse, che si può indagare nei minimi dettagli, ma  che, in questo contesto non sarà possibile approfondire, tuttavia risulta necessario fare riferimento ai ceppi linguistici per avere un’idea più chiara delle ragioni per cui alcune lingue si somigliano così tanto, del perché esista un lessico emotivo, uno stile di  vita, delle norme culturali che influiscono sulle similarità o peculiarità specifiche nel confronto tra le diverse lingue.\\
Per esempio, sapendo che dalla famiglia delle lingue romanze (o di matrice latina) si originano l’italiano, il francese, lo spagnolo, il portoghese, il rumeno e il catalano, possiamo fare caso ai molteplici termini simili, che andranno studiati più approfonditamente, considerando molti altri fattori sociali per capire il nesso tra lessico e cultura. 

\subsection{Rilevanza emotiva del linguaggio}
\paragraph{Suicidi per amore della lingua}
Un primo evento che ci può far comprendere quanto l'aspetto linguistico sia di grandissimo rilievo nella vita degli individui è la storia della questione linguistica indiana, dove si parla addirittura di "morte per amore della lingua" \parencite{language_emotion_politics}. 

Attualmente la Costituzione indiana riconosce l'inglese come seconda lingua ufficiale del Paese, ma le vicissitudini storiche che hanno portato a questo riconoscimento sono state tortuose e drammatiche.\\
Tutto nacque con l'affermarsi del colonialismo britannico in India, che impose, per legge, l'utilizzo dell'inglese in tutto il Paese.\\
Successivamente, negli anni Cinquanta, con l'indipendenza indiana si cercò di istituire una sola lingua nazionale che rappresentasse la cultura del Paese: l'Hindi. Nonostante questa lingua fosse parlata da quasi un terzo della popolazione, questa rimaneva propria solamente degli abitanti di regioni nordiche della nazione, trascurando tutta la popolazione del Sud.\\
Una soluzione proposta per non avvantaggiare solamente una parte del Paese ed escluderne un'altra fu l'utilizzo della lingua coloniale (l'inglese), ritenuta più neutra poiché non in grado di identificare e distinguere l'etnia, la religione o il rango sociale.\\
Questo conflitto sulla lingua più consona da utilizzare per una maggiore coesione del popolo si concluse poi con un compromesso tra le due lingue: pur mantenendo l'Hindi come lingua ufficiale, il governo indiano permise a ciascuno degli Stati indiani di cui è composta, di adottare l'inglese come lingua principale.\\
L'inglese ha visto in questi casi un suo uso pervasivo in tutti gli ambiti della vita quotidiana e, soprattutto nell'ambito scolastico: molti indiani cercano, infatti, di inserire i propri figli nelle scuole inglesi. Questo principalmente perché l'inglese è sempre stato un elemento discriminatorio delle classi sociali più abbienti, e questa concezione permane ancora oggi \parencite{indiamodi}. 

Prima di arrivare a questa soluzione di compromesso tra l'inglese e l'Hindi, ancora oggi messa in discussione, ci furono incredibili proteste da parte degli abitanti meridionali dello Stato, che portarono a quelli che vennero chiamatati «suicidi per amore della lingua \parencite{language_emotion_politics}».\\
Il primo fu quello di Potti Sriramulu, un leader politico indiano, che ha lottato affinché i Telugu di Andhra avessero un proprio stato e una propria lingua, il \textit{Telugu}. Per combattere per i suoi ideali intraprese uno sciopero della fame che lo condusse alla morte, da cui prese il via una protesta più estesa grazie alle moltissime persone dell'India meridionale che lo presero come esempio. 

Tante furono le spiegazioni che vennero date a questo movimento di ribellione, come la ricerca di fama e di importanza, nessuna però sufficientemente convincente per gli antropologi che studiarono questo caso.\\
Una di questi fu Lisa Mitchell che, nel suo libro \textit{"Language, Emotion and Politics in South India"}, illustra il pensiero del XIX secolo, sviluppato da J. G. Herder \footnote{J. G. Herder (1774-1803) fu un importantissimo filosofo tedesco, uno degli intellettuali più rilevanti nella corrente dello \textit{Sturm und Drang}, dove si celebra il sentimento, la natura e l'estetica. Nella sua filosofia del linguaggio egli afferma che «nella parola è l'anima stessa che si esprime e, viceversa, l'anima esiste solo in quanto si esprime nella parola \parencite{Herder}», portando alla luce un nuovo legame tra lingua e spiritualità.}, il quale si interroga sull'origine naturale del linguaggio, giungendo alla concezione della lingua come un essere vivente, attribuendogli le fasi dello sviluppo e la componente relazionale propria del rapporto tra individui.\\
Un altro punto fondamentale per interpretare al meglio questo pensiero è la concezione che l'uomo è in grado di svilupparsi solamente ricreando continuamente il suo linguaggio.\\
Partendo da questa corrente di pensiero si può dunque concepire anche la morte di una lingua, così come era stato fatto per le altre fasi vitali. 

In aggiunta, la lingua fu vista come divinità indiana, e dunque fu spostata anche sul piano spirituale, di culto, oltre che su quello biologico.\\
Infine, l'elemento che funse da "cassa di risonanza" furono i mass media, che permisero che l'amore che si era sviluppato per lingua si diffondesse in tutto il Paese \parencite{language_emotion_politics}. 

Da questa intensa e violenta rappresentazione di amore per la lingua possiamo comprendere quanto abbia rilevanza nella vita di un popolo la propria lingua, in quanto rappresenta ciò che permette di mantenere viva la cultura specifica di un popolo e che, se denigrata, porta alla cancellazione dell'identità etnica di una comunità.\\
La grande presenza di emozioni che legano l'uomo al suo linguaggio e la risonanza emotiva che hanno avuto le manifestazioni di protesta per proteggere la propria identità culturale mette in rilievo la vicinanza del piano emotivo a quello lessicale.

\paragraph{Svolta linguistica}
\textit{The linguistic turn}, o in italiano "svolta linguistica", è l'espressione che si usa per far riferimento ad un fenomeno culturale e intellettuale che ha avuto luogo durante il XX secolo.\\
Questa corrente si sviluppò principalmente nell'ambito della filosofia occidentale, dando successivamente vita alla filosofia analitica e linguistica \parencite{Rorty_Linguistic_Turn}. 

Il termine fu coniato dal filosofo Gustav Bergmann negli anni Cinquanta, e reso celebre da Richard Rorty con la sua antologia \textit{"The linguistic turn"} (1967), il quale si dissociò, in seguito, dal pensiero della filosofia linguistica.\\
La caratteristica più importante della svolta linguistica fu proprio l'attenzione della filosofia e delle altre discipline umanistiche principalmente alle relazioni tra il linguaggio, coloro che lo usano e il mondo esterno \parencite{linguistic_turn}. 

Un obbiettivo che si voleva perseguire era quello di superare il dualismo della dimensione soggettiva e oggettiva della mente, nata dal dualismo genitore di tutti: quello cartesiano, che divise mente e corpo.\\
Ciò che si intende con il raggiungimento di questo obbiettivo è quello di vedere l'oggettività della realtà a partire dal recupero della dimensione soggettiva dell'individuo, mettendo in luce la sua coscienza.\\
Il \textit{linguistic turn} costituisce una vera e propria svolta poiché, attraverso l'analisi della logica e della filosofia del linguaggio, sposta il focus dalla dimensione soggettiva della mente sul linguaggio stesso.\\
Si cercherà, quindi, di fare un'analisi molto approfondita del linguaggio utilizzato per dare un significato obbiettivo, partendo da un punto di vista soggettivo. 

La svolta linguistica, con il suo cambio di paradigma, ha interessato molte discipline.\\
Un effetto evidente è stato quello avvenuto nell'ambito della storia, più precisamente nella storiografia \parencite{Toews_linguistic} \footnote{Per storiografia si intende una «scienza e pratica dello scrivere opere relative a eventi storici del passato, in quanto si possano riconoscere in essa un’indagine critica e dei principi metodologici: i metodi della s.; storia della s., che ha per oggetto l’evolversi del metodo storico \parencite{storiografia}».}.\\
In questo caso viene posta al centro l'analisi del linguaggio utilizzato dagli storici, dal punto di vista della logica e della filosofia, che permetterebbe di sorpassare la dimensione soggettiva dell'autore, riuscendo a raggiungere l'oggettività del contenuto.\\
Con lo studio dei testi storici come artefatti linguistici, proprio come se fossero opere narrative e letterarie, è possibile comprendere la costruzione della realtà soggettiva dell'autore: ma è proprio la forma linguistica analizzata che determina l'"oggettività" del contenuto.\\
Questo concetto viene introdotto da H. White, con la sua pubblicazione \textit{"Metahistory"} \parencite{white}, nella quale viene sottolineata l'importanza di analizzare il discorso degli storici per comprendere la costruzione delle identità individuali e collettive. 

La presunta oggettività della storicità viene quindi messa in discussione, facendo presente che è sempre mediata dal pensiero etico e politico dello storico, che manipola la descrizione storica. É proprio dall’analisi semantica del linguaggio utilizzato che si può arrivare a riconoscere i limiti della realtà oggettiva del passato e l’enorme influenza che la percezione soggettiva dello storico ha esercitato sulla ricostruzione degli eventi storici.\\
Fino a quel momento, infatti, la soggettività dello storico era stata totalmente trascurata lasciando credere che fosse possibile ad una descrizione oggettiva dei fatti.\\
In sintesi dunque, con l’affermarsi del \textit{linguistic turn} gli studiosi cominciano a prendere coscienza del fatto che il passato non esiste al di fuori delle rappresentazioni testuali e che queste rappresentazioni, che sono rappresentazioni linguistiche, non possono essere separate dal bagaglio ideologico che gli storici portano con sé. 

Prendere coscienza di questi aspetti ha permesso innanzitutto di indagare alcuni aspetti fino ad allora trascurati nell’ambito storico come, ad esempio, la storia di genere, che si occupa di esaminare  il ruolo del genere e delle dinamiche di potere nella storia ed ha avuto una grande rilevanza  nella lotta per una maggiore parità tra generi \parencite{storia_di_genere}; o la memorialistica, ovvero la scritture delle memorie personali ed autobiografiche degli individui, che pur lasciando ampio spazio alla soggettività dell’autore permette di capire in modo esplicito come e in che misura la visione emotivo/soggettiva di chi descrive gli eventi permette di comprenderne i significati senza trascurare gli aspetti culturali, sociali e del contesto inteso sia come spazio che come tempo \parencite{memorialistica}.\\
In questa prospettiva diviene possibile afferrare quanto l’analisi semantica del linguaggio possa aprire nuovi mondi alla conoscenza dei fatti e permetta di sradicare concezioni assodate nel corso del tempo come oggettive, che invece vanno contestualizzate e relativizzate al contesto socio-culturale, al momento storico e alle ideologie e dimensioni emotive dell'autore. 

Lo studio analitico del linguaggio e della dimensione soggettiva degli autori ha permesso anche di dare più spazio all'ambito emotivo. Attraverso il linguaggio si possono individuare le emozioni provate, dando così la possibilità di ricostruire le norme sociali e culturali che veicolano quelle particolari emozioni intercettate.\\
\textit{The linguistic turn} fu quindi una grande "cassa di risonanza" per la dimensione emotiva, il linguaggio e, quindi, per lo studio del lessico emotivo.  

\paragraph{\textit{Affect labeling}}
\label{par: Affect labeling}
Raggiungendo tempi più recenti, possiamo trovare alcune indagini interessanti sull'\textit{affect labeling}: altro argomento che evidenzia la stretta correlazione tra emozioni e linguaggio.

Ciò che viene identificato come \textit{affect labeling}, letteralmente "etichettatura degli affetti", può essere spiegato semplicemente con l'espressione \textit{"putting feelings into words"} cioè "inserire i sentimenti all'interno delle parole".\\
Quello che vuole esplicitare quest'espressione è un metodo che consiste nell'etichettare esplicitamente le emozioni che si stanno provando \parencite{naural_basis_labeling}.\\
Alcune delle tecniche utilizzate per l'\textit{affect labeling} potrebbero essere, ad esempio: parlare con un terapeuta, scrivere le proprie esperienze interiori su un diario, sfogarsi con un amico; insomma, prendere coscienza dei propri sentimenti negativi attraverso l'esplicitazione a parole di essi \parencite{modalità_labeling}. 

Gli studi condotti sull'\textit{affect labeling} attraverso metodologie neuroscientifiche, si sono concertati in particolar modo sull'etichettatura delle emozioni negative, poiché è proprio durante questo processo che sono stati riscontrati alcuni effetti benefici psichici per l'individuo.\\
Tali ricerche hanno infatti dimostrato che parlare delle proprie emozioni negative, attraverso i vari modi sopracitati, aiuterebbe ad interfacciarci con esse, riuscendo a reagire meglio sia a livello interiore, mentale, prendendone coscienza; sia a livello esteriore, comportamentale.\\
Il processo che si mette in atto può essere descritto come una tecnica di regolazione emotiva molto potente: quando esprimiamo ciò che sentiamo a parole regoliamo automaticamente le nostre emozioni, ottenendo effetti positivi anche senza volerlo.

Grazie alla risonanza magnetica, una delle tecniche neuroscientifiche utilizzate per questi studi, è stato anche possibile comprendere i principali processi neurofisiologici messi in atto durante l'\textit{affect labeling} \parencite{fmri_affect_labeling}; riuscendo ad approfondire lo studio delle regioni cerebrali che coinvolgono il linguaggio e le esperienze emotive. 

Ciò che gli autori prendono in considerazione per rendere ragione dell’effetto benefico del dare parola alle emozioni sono gli aspetti di codifica delle informazioni e le aree cerebrali coinvolte \footnote{Possiamo partire dalle informazioni descritte nel \autoref{par: Neurofisiologia} e ampliarle.}.\\
Quello che è emerso da questo studio è che nell’etichettatura degli affetti, rispetto ad altri processi di codifica, si assiste ad una diminuzione significativa dell’attivazione dell’amigdala e di altre regioni limbiche quando vengono presentati ai soggetti sperimentali immagini di emozioni negative.\\
L'unica regione cerebrale che aumenta la sua attività nell'elaborazione linguistica delle emozioni negative è la corteccia prefrontale ventrolaterale destra (RVLPFC): questa è associata all'elaborazione simbolica delle informazioni emotive e ai processi inibitori top-down.

Questi risultati ci dicono che l'esplicitazione delle emozioni negative a parole, attraverso i processi linguistici, provocano un'attivazione della RVLPFC e una conseguente diminuzione responsiva dell'amigdala, che conduce ad un calo sostanziale dello stress emotivo \parencite{affect_labeling}.
Più nello specifico la conclusione a cui giungono i ricercatori è che ci sia una riduzione della reattività emotiva lungo il percorso che va dalla corteccia prefrontale ventrolaterale destra alla corteccia prefrontale mediale all’amigdala \parencite{affect_labeling}.\\
Nonostante i risultati ottenuti, rimangono molti altri processi cerebrali legati a questo fenomeno che necessitano di ulteriori indagini. 

L'\textit{affect labeling} viene ritenuto di grande importanza per l'impatto benefico che può avere in situazioni terapeutiche: invitando il paziente ad esprimere le proprie emozioni, infatti, possiamo aiutarlo a ridurre il suo malessere psichico. Inoltre questa tecnica può essere utilizzata da ciascuno di noi anche in situazioni quotidiane: si può prendere l'abitudine a scrivere un diario o a parlare più soventemente con qualcuno delle proprie emozioni negative.

Gli studi condotti su quest'argomento, inoltre, portano alla luce la grande connessione tra l'ambito linguistico e comunicativo e quello emotivo e psicologico.\\
L'aspetto del lessico emotivo acquisisce ampia rilevanza: è proprio grazie al lessico emotivo che si possiede che è possibile esprimere adeguatamente i propri stati emotivi, riconoscendone quelli positivi e negativi in base al contesto socio-culturale in cui ci si trova.
\chapter{Lessico emotivo} 
\label{chap: Lessico emotivo}

\section{Definire il lessico emotivo}
Seppure il senso comune non trovi una grande rilevanza nella relazione tra emozioni e linguaggio, anzi vi è più la concezione delle emozioni come qualcosa di fisico, distaccato dalle etichette linguistiche ad esse associate, la psicologia vede il linguaggio come fondamentale per i processi emotivi: sia per la percezione delle emozioni che per la loro sperimentazione concreta\footnote{Tema affrontato in modo approfondito dalla corrente di pensiero psicologico di tipo costruzionista, che verrà indagato nella \autoref{subsec: Socio-costruttivismo}.}.\\
Attualmente infatti, l'indagine delle emozioni umane richiede in maniera imprescindibile il contributo dello studio delle lingue, portando la psicologia ad interagire con altri ambiti di ricerca come l'antropologia e la sociologia, ma anche la linguistica e, più precisamente la psicolinguistica, disciplina nata negli anni Cinquanta, che studia i processi psicologici e neurobiologici che permettono l'acquisizione, la comprensione e l'utilizzo del linguaggio \parencite{psicolinguistica}. 

Il ruolo centrale del linguaggio nello studio delle emozioni risulta particolarmente rilevante quando si prendono in considerazione gli aspetti transculturali: a volte, ad esempio, un gruppo sembra concentrare maggiormente l'attenzione su stati emotivi per i quali altri gruppi non hanno nemmeno un nome \parencite{Wierzbicka}.\\
L'analisi approfondita del lessico emotivo, vale a dire dei significati degli stati emotivi nei diversi gruppi linguistici e culturali ha permesso di apportare elementi conoscitivi molto importanti sulle emozioni e sul contesto culturale all'interno del quale tali emozioni vengono espresse.

L'obbiettivo del presente elaborato è analizzare e definire il lessico emotivo e capire la sua funzione nella vita quotidiana degli individui.\\
Il lessico emotivo viene definito come: 

\begin{quote}
    «l'insieme di parole con le quali vengono identificate, isolate e distinte le differenti forme di vissuti emotivi e affettivi, nei diversi linguaggi umani \parencite{development_of_emotioinal_competencde}».
\end{quote}

In altri termini, per lessico emozionale si può intendere quell'insieme che raggruppa tutte le parole che usiamo nella vita quotidiana per parlare delle nostre emozioni e di quelle altrui: nello studio del lessico emotivo si prendono in considerazione gli aspetti riguardanti le parole di una lingua e le loro interazioni con le emozioni.\\
Il lessico emotivo ci dà la possibilità di trasmettere varie informazioni sull'emozione che stiamo provando, specificando di che emozione si tratta oppure spiegando l'evento che l'ha causata.

Nello studio del lessico emotivo, oltre ai contenuti semantici che possono essere considerati di tipo emotivo, bisogna esplorare anche le risorse utilizzate per trasmettere tali significati e riuscire a comunicare tutti i dati relativi ad uno stato uno emotivo.\\
Le risorse in questione, che si distinguono in diverse tipologie, vengono definite segmentali. Questo significa che sono considerate come segmenti, ovvero elementi linguistici, disposti in successione temporale, che in una frase formano una sequenza lineare \parencite{segmentale}; le risorse sono dunque segmenti del processo comunicativo.\\
Verranno di seguito elencati i tipi di risorse segmentali prese in considerazione nello studio della comunicazione emotiva attraverso il linguaggio.\\
Risorse lessicali: sono le più facilmente riconoscibili e costituiscono l'insieme dei nomi, aggettivi, verbi avverbi ed esclamazioni (ad esempio: felicità, rallegrarsi, gioiosamente, ecc.) che usiamo per denominare e descrivere approfonditamente gli stati emotivi, le particolari sensazioni provate da noi o da altri in un certo conteso, riuscendo a comunicarle facilmente. Risorse sintattiche: riguardano la struttura della frase, si riferiscono all'ordine specifico di ogni parola all'interno della frase, in modo che sia possibile esprimere le proprie idee. Tale risorsa permette di produrre uno specifico significato emotivo, riuscendo a portare l'enfasi si determinati elementi connotati emotivamente: ad esempio cambiando l'ordine delle parole o ripetendo la stessa più volte a inizio frase (anafora) è possibile enfatizzare quell'elemento specifico. 
Risorse morfologiche: sono costituite da diminutivi, vezzeggiativi o dispregiativi. Queste risorse permettono di dare un particolare connotato (negativo o positivo) alle parole che stiamo esprimendo, riuscendo così ad invocare emozioni coerenti con il connotato che si è dato nel ricevente del messaggio. 
Infine le risorse fonologiche sono riconducibili ad esempio ai fonosimboli, ossia delle manifestazioni foniche o onomatopeiche, che quindi riconducono a suoni o rumori (per esempio, nell'ambiente linguistico italiano, uffa, mah, bah) \parencite{fonosimbolo}.

Tali risorse, utilizzate congiuntamente in modo adeguato e coerente all'interno di un atto comunicativo, ci permettono di esprimere gli stati emotivi personali e, al contempo, capire quelli altrui.\\ 
Un altro mezzo che risulta fondamentale per adempire questo compito è, sicuramente, il contesto in cui ci si trova: analizzare attentamente la situazione in cui si è immersi ci permette di dare un senso a ciò che viene comunicato attraverso le risorse segmentali \parencite{parlato_emotivo}.  

Lo studio del lessico emotivo, inoltre, è considerato una via di accesso importante per studiare le emozioni stesse: vengono analizzati i vari modi in cui le emozioni sono identificate, isolate e distinte in base alla lingua parlata.

Ciò che viene particolarmenete indagato è la struttura semantica del lessico emotivo, ovvero «una sorta di linea guida con cui la nostra conoscenza dell'esperienza emotiva è strutturata e organizzata nella nostra mente \parencite{Sini}».\\
Questa si può mettere in luce attraverso l'analisi approfondita della considerazione di ogni individuo dei significati delle parole associate alle emozioni e, nello specifico, del modo in cui ognuno di noi mette in relazione tra loro queste parole. \\
La struttura semantica rappresenta quindi l'insieme dei significati che ognuno di noi esprime attraverso il linguaggio.\\
Un concetto basilare da comprendere è che il modo in cui gli individui utilizzano il linguaggio permette di accedere indirettamente alla modalità in cui sono organizzati i concetti internamente. \\
La grande rilevanza che viene data al linguaggio riguarda proprio questo punto cruciale: la disponibilità di vocaboli che abbiamo, la modalità in cui li organizziamo tra loro per formare dei significati, si traduce poi nei nostri pensieri e nella forma in cui riusciamo ad esprimerci e comunicare con gli altri. La struttura semantica  è la responsabile dei significati che ognuno di noi dà al mondo esteriore ed interiore. 

Il lessico emotivo sarà quindi un potentissimo strumento da utilizzare per studiare le emozioni e le differenze che possiamo trovare tra le diverse culture.\\
Per prima cosa, però, è importante comprendere quali siano le prospettive teoriche che lo riguardano e che ci permettono di comprendere al meglio i concetti che lo definiscono.

\section{Prospettive teoriche del lessico emotivo}
La concettualizzazione delle emozioni secondo una prospettiva universalista o socio-costruttivista ha acceso sempre un grande dibattito tra filosofi, psicologi, scienziati e antropologi nel corso della storia. L'ambito delle emozioni e, conseguentemente quello del lessico emotivo, sono stati indagati sotto l'influenza di epoche storiche ed intellettuali diverse, giungendo a prove a favore di entrambe le prospettive teoriche.

Per quanto riguarda la branca della linguistica \footnote{Come spiegato nel \autoref{par: Semantica}, questo ambito della linguistica si riferisce all'insieme di significati espressi attraverso il linguaggio, che permettono l'accesso alla nostra organizzazione mentale.}, si possono distinguere teorie che vengono suddivise in base a come vengono concepite le categorie linguistiche di ciascuna lingua.\\
Esse possono essere considerate come universali, comuni a tutte lingue: in questo caso i linguisti cercano di  osservare il comportamento e definire le caratteristiche universali che permetto di individuare una categoria universale per tutte le lingue prese in considerazione.\\
Se, al contrario, si pensa che ogni lingua abbia sue categorie specifiche, si parla di particolarismo categoriale e si ritiene che non sia possibile equiparare una categoria di una determinata lingua con la stessa di un altra \parencite{haspelmath}. 

Ciò che è stato messo in discussione nell'ambito del lessico emotivo è lo studio della semantica, all'interno del quale ci si chiede se l'organizzazione mentale del nostro lessico emotivo sia universale e innata, oppure dipendente dalla cultura specifica in cui viviamo \parencite{universal_structure}.

\subsection{Universalismo}
La prospettiva universalista sostiene che le emozioni abbiano un substrato innato, comune a tutti gli individui, indipendentemente dalla cultura a cui appartengono o al contesto sociale in cui sono immersi \parencite{izard_article}.\\
Le emozioni, quindi, sarebbero qualcosa di costante e biologicamente universale, e l'influenza culturale permetterebbe solamente di modificare la loro diversa manifestazione e concettualizzazione, senza intaccare il substrato universale che le compone.\\
Alcuni aspetti universali dell'esperienza emotiva possono essere individuati facilmente e vengono 
infatti ritenuti \textit{trivially true}, ossia banalmente veri: si tratta di automatismi fisiologici in risposta a stimoli emotivi che sono comuni a tutti gli esseri umani. Ad esempio, la paura può essere associata a un aumento del battito cardiaco, dell'afflusso di sangue ai muscoli e ad altre risposte fisiologiche simili in tutte persone, indipendentemente dalla loro cultura \parencite{storia_delle_emozioni}.

L'universalismo emotivo trova le sue origini per la prima volta nel  "darwinismo delle emozioni". Darwin, infatti, sosteneva che le emozioni umane avessero una base biologica comune e fossero il frutto dell'evoluzione, che potessero contribuire al successo riproduttivo e alla sopravvivenza della specie umana \parencite{darwin}.\\
Il pensiero Darwiniano fu ripreso poi da grandi psicologi che diedero vita alla prospettiva universalista delle emozioni. 

Gli studi di maggiore rilevanza, che hanno dato voce al pensiero universalistico delle emozioni sono sicuramente quelli di Paul Ekamn. Egli, attraverso molteplici esperimenti, afferma un principio di universalità delle espressioni facciali e l'esistenza di sei emozioni di base (gioia, tristezza, sorpresa, disgusto, paura e rabbia) che accomunano tutti gli individui, indipendentemente dal contesto sociale, dalla cultura d'origine o dalla lingua parlata.\\ Ekman voleva dimostrare che le differenze culturali non sono rilevanti per quanto riguarda le emozioni primarie di base: incondizionatamente dall'area geografica di provenienza, dallo status sociale o dal livello di sviluppo della popolazione presa in considerazione, tutti gli individui riuscivano a riconoscerle \parencite{ekman}.\\
Seppure il lavoro di Ekman fu successivamente  criticato da psicologi ed antropologi, il suo apporto teorico sulle espressioni facciali e l'esistenza delle emozioni di base ebbe un'incredibile rilevanza nell'ambito della psicologia delle emozioni. 

Molti autori del Ventesimo e Ventunesimo secolo hanno seguito le orme di Ekman, dando maggiore risonanza alla prospettiva universalista, rilevando diverse emozioni di base per spiegare l'esperienza emotiva umana.\\
Alcuni dei più conosciuti sono: Tomkins, il quale rileva la presenza di otto emozioni innate e universali, presenti fin dalla nascita, che fornirebbero una base comune per forgiare il comportamento e l'esperienza umana \parencite{tomkins}; così come Johnson Laird e Oatley, che affermano l'esistenza di cinque emozioni fondamentali, generate dal primo livello di risposta agli stimoli emotivi, predisposto all'adattamento dell'organismo all'ambiente \parencite{Keith_JohnsonLaird} \footnote{Tali livelli sono stati approfonditi nella \autoref{subsec: Teorie cognitive}.}.

Questa prospettiva viene approfondita anche nell'ambito delle neuroscienze, in quanto sono state individuate risposte cerebrali e fisiologiche agli stimoli emotivi comuni a tutti gli individui.\\
Uno degli neuroscienziati più rilevanti per la prospettiva universalista è stato sicuramente Panksepp\footnote{È bene sottolineare che nonostante alcuni suoi apporti teorici siano stati molto importanti per lo studio dell'universalità delle emozioni, le sue ricerche si sono concentrata principalmente sulle emozioni animali e la loro continuità con quelle umane \parencite{panksepp_libro}.}, che, attraverso i suoi studi nell'ambito delle "neuroscienze affettive" \footnote{Questo termine fu coniato dallo stesso Panksepp per indicare un'area di ricerca della scienza riguardante lo studio del cervello interspecie \parencite{panksepp_articolo}.}, ha individuato sette sistemi emozionali primari comuni (ricerca, cura, gioco e piacere, e i corrispettivi negativi di paura, malattia e angoscia). Varie ricerche oltre a dimostrare l'effettiva esistenza di tali emozioni, hanno confermato anche l'ipotesi secondo la quale i loro squilibri sarebbero una causa di disturbi psichiatrici \parencite{panksepp_articolo}.\\
Il sistema emotivo, ritenuto il più basilare, attraverso il quale tutti gli altri si sarebbero sviluppati, è quello della ricerca, ovvero la "voglia di fare", che corrisponderebbe a quella che Izard chiama "interesse" \parencite{panksepp_izard}. 

Carroll Izard è considerato un altro importante pioniere della prospettiva universalista delle emozioni, in quanto sostiene che le emozioni siano innate e che si sviluppino nel corso della crescita a partire dalle prime settimane di vita e per alcuni anni dello sviluppo.\\
Per far comprendere al meglio la non apprendibilità delle emozioni, le paragona a quattro gusti (salato, dolce, aspro e amaro) che siamo in grado di riconoscere in modo spontaneo, istintivo.\\
Un'importante aggiunta che egli propone rispetto ad altri autori, è però la funzione del linguaggio: questo viene ritenuto come qualcosa che può essere appreso, a differenza delle emozioni, ed è ciò che gli individui utilizzano per riferirsi agli stati emotivi. Egli, quindi, afferma che è possibile apprendere a nominare e gestire gli stati emotivi, ma questi ultimi rimangono di per sé un concetto universale ed innato \parencite{izard_intro}. 

Con questa teoria Izard differenzia quindi le emozioni di base, definite come strutture innate, e gli schemi emotivi, più complessi e appresi attraverso l'esperienza e il linguaggio \parencite{izard_schemi_emotivi}. Gli schemi emotivi, infatti, subiscono l'influenza della cultura in cui l'individuo è inserito, e vedono come elemento fondamentale quello della parola: il linguaggio appreso permette di comprendere a fondo l'emozione provata, esplicitandone le cause e le conseguenze, dando agli stati emotivi diversi significati in base ai costrutti culturali che influenzano il nostro modo di sentire.

Izard ci dà la possibilità di iniziare ad esplorare più approfonditamente il ruolo del linguaggio utilizzato nell'ambito emotivo, quindi del lessico emotivo, in relazione alle diverse prospettive teoriche che definiscono le emozioni.

\subsection{Verso il costruzionismo sociale}
Quando il linguaggio guadagna più importanza all'interno dello studio delle emozioni, bisogna iniziare a prendere in considerazione anche la prospettiva sociocostruttivista.\\
Di seguito verranno riportati alcuni autori che, nonostante rimangano legati ad alcuni elementi della prospettiva universalista delle emozioni, iniziano ad introdurre aspetti relativi al costruzionismo sociale, mettendo in evidenza le componenti culturali e sociali che influenzano l'ambito delle emozioni e del lessico emotivo.

Klaus R. Scherer, legandosi ai concetti delle emozioni di base proposti da Ekman, Tomnkins e Izard, introduce il suo pensiero riguardante le emozioni di base, definite da lui come "utilitaristiche", in quanto utili per l'adattamento individuale in risposta a situazioni frequenti \parencite{scherer_ekman}. Considerando gli aspetti di valutazione degli stimoli emotivi e, in particolare la prototipicità \footnote{La prototipicità è un concetto derivato dalla teoria dei prototipi di Eleanor Rosch e si riferisce alla misura della somiglianza di un oggetto ad un determinato prototipo di una categoria. Gli esempi che sono più simili al prototipo sono considerati più prototipici, mentre quelli che sono meno simili sono considerati meno prototipici \parencite{prototipicità}.} di queste emozioni, decide di nominarle "modali" e non "di base". 

Partendo da questa considerazione, Scherer, mette in evidenza la vasta gamma di emozioni presenti negli esseri viventi, che va al di là delle sole emozioni modali (facilmente rilevabili poiché più frequenti). Le molteplici emozioni che possono originare dai modelli di risposta agli stimoli emotivi \footnote {Scherer è di estrema rilevanza nell'ambito della psicologia per la sua teoria dell'\textit{appraisal}: egli considerava fondamentale la valutazione cognitiva degli stimoli emotivi per generare l'emozione. Nella sua teoria del "Modello dei Processi Componenti" vengono descritti i cinque componenti del processo di valutazione che, interagendo tra di loro, costituiscono l'emozione \parencite{teoria_componenti_scherer}.}, però, presentano un grande limite: sono difficilmente misurabili.\\
Una metodologia che lo psicologo svizzero propone, dunque, è quella di indagare le condotte popolari e il lessico emotivo utilizzato. In questo compito risulta fondamentale concentrarsi sulle distinzioni emozionali che il lessico emotivo descrive, cogliendo i processi a cui fanno riferimento i vari termini.\\
Per lo studio di questi aspetti, Scherer formula un esperimento in cui veniva chiesto ai partecipanti di cogliere le modalità in cui una persona tendenzialmente valuterebbe e risponderebbe ad uno stimolo emotivo relativo ad una determinata emozione.\\
Grazie ai riscontri ottenuti si riescono a formulare delle griglie semantiche per i diversi termini emozionali, le quali definiscono il loro campo di significato in ciascuna lingua analizzata. Questi dati permettono di studiare le sottili differenze di significato nei diversi termini emotivi in base alla lingua parlata e, conseguentemente, forniscono informazioni sulle potenziali differenze culturali e linguistiche nella codifica delle emozioni \parencite{scherer}.

Un libro fondamentale, che indaga ancora in maniera più approfondita il tema del lessico emotivo in relazione all'ambito transculturale e alla prospettiva universalista è sicuramente \textit{«Emotions in Crosslinguistic Perspective»} (2001) di Harkins e Wierzbicka. \\
Qui gli autori, parlando dell'universalità delle emozioni affermano che «è molto probabile che il numero di modelli fisiologici [legati alle emozioni] sia limitato e universale, ma che non ci sia universalità nell'esperienza soggettiva corrispondente \parencite{Wierzbicka}».\\
Le emozioni vengono quindi descritte come "universali esperienziali": esse sono legate al principio secondo cui tutti gli individui possono provare un insieme di esperienze soggettive simili tra loro. 

A differenza delle teorie universaliste, però, viene evidenziato che un'emozione non può essere semplicemente considerata universale, poiché la parola associata ad essa ha un significato specifico a seconda della cultura in cui è inserita e a seconda della lingua parlata. Le emozioni non possono dunque riferirsi a significati universali.\\
In sintesi le emozioni umane variano parecchio in base alla cultura e alla lingua, ma hanno anche molti aspetti esperienziali comuni tra loro. 

Per studiare tali somiglianze e differenze del lessico emotivo viene ritenuto necessario un metalinguaggio che possa arginare la grande variabilità data dalle lingue naturali, le quali posseggono una propria "immagine ingenua del mondo", che porta a una visione della realtà differente per ciascuna lingua.\\
Il metalinguaggio definito dagli autori è chiamato \textit{"Natural Semantic Metalanguage (NSM)"} \parencite{Wierzbicka}, attraverso il quale si cercano di analizzare i concetti emotivi universali, collegandoli ad una loro grammatica universale \footnote{Per grammatica universale si intendono le regole innate di combinazione universale dei concetti comuni a tutti gli individui \parencite{Chomsky}.}, partendo da studi linguistici realizzati nel corso degli anni. 

Lo studio degli aspetti universali del lessico emotivo secondo l'approccio del NSM si distanzia dal metodo precedente, proposto da Van Geert, il quale ritiene che solamente un esperto, grazie al linguaggio tecnico riesca a decifrare tutti i componenti dell'esperienza soggettiva emotiva \parencite{Van_Geertz}.

Secondo Harkins e Wierzbicka questa prospettiva sarebbe una forma di etnocentrismo \footnote{Per etnocentrismo si intende la «tendenza a giudicare i membri, la struttura, la cultura, la storia e il comportamento di altri gruppi etnici con riferimento ai valori, alle norme e ai costumi del gruppo a cui si appartiene, per acritica presunzione di una propria superiorità culturale \parencite{etnocentrismo}».} e di scientismo sviante \footnote{Scientismo è un termine che indica la «tendenza a considerare solo la scienza come unica fonte di conoscenza valida e a svalutare altri campi del sapere» e che, come in questo caso, può portare all'errore \parencite{scientismo}».}.
L'utilizzo di un metalinguaggio tecnico da parte di esperti, infatti, non può mettere in luce l'esperienza umana ordinaria e la sua concettualizzazione soggettiva.\\
Al contrario, per fare ciò, gli studiosi devono comprendere come gli individui, considerati "ordinari", pensano e parlano, provando a trovare elementi comuni tra il linguaggio utilizzato dagli studiosi e quello che si intende prendere in analisi. 

Con questo libro, gli autori mettono in discussione la prospettiva universalista, dando voce agli aspetti di differenziazione culturale, introducendone la loro grande importanza nello studio delle emozioni.\\
Viene anche profondamente contestata l'idea dell'universalista Ekman, secondo cui le parole emotive non hanno rilevanza per lo studio delle emozioni in sé, dato che queste ultime sarebbero frutto di processi biologici innati \parencite{Wierzbicka}.

Levy è un altro psicologo nel cui pensiero possiamo trovare un passaggio tra la prospettiva universalista e quella sociocostruttivista delle emozioni.\\
Nella sua teoria le emozioni sarebbero il risultato di processi cognitivi relativi all'elaborazione dello stimolo emotivo, l'interpretazione cognitiva di tale stimolo, la risposta fisiologica e l'espressione comportamentale. \\
Tra le varie emozioni generate dall'interazione di tutti questi processi, l'autore distingue una serie di emozioni, definite primarie e un'altra costituita da emozioni complesse \parencite{levy_culture}.\\
Le prime sarebbero emozioni universali ed innate, come la paura, la rabbia, la tristezza e la gioia; mentre le emozioni complesse risulterebbero da un'elaborazione cognitiva e culturale delle emozioni primarie. 

Levy si è poi principalmente occupato della regolazione emotiva, specialmente in relazione alla psicopatologia. In particolare, ha messo in evidenza come le emozioni complesse, generate attraverso le relazioni sociali, possano contribuire allo sviluppo di problemi psicopatologici \parencite{levy_emotion_regulation}. Egli, quindi, prende in considerazione le relazioni sociali e il contesto culturale per costruire elementi psicoterapeutici che possano aiutare nella regolazione delle emozioni complesse. \\
Nonostante la sua influenza sia stata più nell'ambito clinico e psicoterapeutico, nella sua teoria delle emozioni, possiamo osservare la compresenza di aspetti universalisti e sociocostruttivisti utili allo studio delle emozioni. 

La prospettiva universalista mette da parte gli aspetti culturali e sociali ma, come si è potuto già evincere dagli ultimi autori citati, questi sono fondamentali per studiare al meglio i processi emotivi e, soprattutto, il ruolo del lessico emotivo.\\
Le emozioni differiscono così tanto in base alla  cultura presa in considerazione, che il principio secondo cui tutti gli esseri umani sentono alla stessa maniera, viene fortemente messo in discussione. I principi culturali e sociali verranno infatti presi in considerazioni da molti psicologi, antropologi e linguisti, e costituiranno il nucleo della prospettiva socio-costruttivista.

\subsection{Socio-costruttivismo}
\label{subsec: Socio-costruttivismo}
La prospettiva socio-costruttivista delle emozioni viene concepita come opposta a quella universalista, poiché concettualizza le emozioni come un prodotto del contesto sociale e culturale e non come un qualcosa di innato e comune a tutti gli esseri umani. 

Nella prospettiva socio-costruttivista le emozioni sono completamente influenzate dal contesto in cui l'individuo vive e, dunque, dalle norme, aspettative e pratiche sociali di tale contesto.\\
Le emozioni vengono viste come qualcosa di socialmente appreso, costruite attraverso l'interazione sociale, seguendo regole ben precise definite socialmente e culturalmente \parencite{gergen}.\\
Un altro elemento fondamentale che evidenzia il socio-costruttivismo, che non veniva preso in considerazione nella prospettiva universalista, è la dinamicità delle emozioni: queste possono essere influenzate dalle risposte emotive degli altri e dalle dinamiche relazionali e, quindi, mutare nel tempo.\\
L'enfatizzazione delle dinamiche relazionali e delle interazioni tra individui ci porta a sottolineare il ruolo fondamentale del linguaggio nel socio-costruttivismo e quindi, per quanto riguarderà le emozioni, del lessico emotivo, il quale permette di etichettare e descrivere l'esperienza emotiva, dandole un significato preciso.\\
Per comprendere al meglio la prospettiva socio-costruttivista è bene conoscere il pensiero dei principali autori che descrivono la realtà come frutto della costruzione sociale.

Lo psicologico statunitense Kenneth J. Gergen è considerato uno dei padri di questa corrente di pensiero e mette in luce i suoi aspetti chiave.
Afferma che le esperienze individuali, comprese le emozioni, siano costruite attraverso le interazioni sociali e culturali \parencite{gergen}. \\
Le emozioni, che nell'universalismo erano concepite come interiori ed innate, ora vengono viste come una costruzione attraverso la comunicazione e le relazioni sociali. Esse sono influenzate dai significati condivisi dalla comunità a cui l'individuo appartiene: vengono, dunque, costruite dal linguaggio in base ai significati socialmente condivisi.\\
Di conseguenza le emozioni variano parecchio tra le diverse culture e i contesti sociali in base  all'insieme di tali significati.\\
Inoltre, proprio per questo motivo, le emozioni sarebbero costruite collettivamente (non sono più individuali come nell'universalismo) attraverso le interazioni sociali e le pratiche culturali che influenzano la gamma di emozioni che riconosciamo e le modalità attraverso cui le esprimiamo.

Più o meno negli stessi anni anche il filosofo e psicologo britannico Rom Harrè diede un grosso contributo alla prospettiva del costruzionismo sociale delle emozioni, soprattutto in relazione al linguaggio.\\
Harrè afferma che l'analisi dei differenti vocaboli, utilizzati in un determinato contesto per descrivere le emozioni, sia centrale \parencite{harre}. Studiare come viene usata una determinata parola associata ad un'emozione ci permette di indagare l'emozione stessa nella cultura presa in considerazione.\\  
Harré critica la visione psicologica del lessico emotivo come mera rappresentazione dell'emozione: esso ha la funzione di costruire le emozioni stesse.\\
La visione costruzionista, infatti, vede la realtà come il risultato di un lavoro di organizzazione, ordine ed interpretazione degli eventi esterni, che conduce alla possibilità di agire solamente in base alle risorse linguistiche e alle abilità sociali individuali.
 
Anche la psicologa e neuroscienziata britannica Lisa Felman Barrett sostiene che le emozioni non siano innate ed universali, ma costruite culturalmente.\\
Attraverso le sue ricerche sostiene che le emozioni non vengono innescate da stimoli esterni, ma che siano il risultato dell'interpretazione personale della propria risposta corporea, influenzata dal contesto sociale in cui ci si trova.\\
La neuroscienziata studia approfonditamente la relazione tra i processi cognitivi e le emozioni, dimostrando che le emozioni non sembrano essere localizzate in un area precisa del cervello, ma che smebrano essere generate dall'interazione di molteplici regioni cerebrali \parencite{barrett_neuroscienze}. \\
I processi cognitivi che si sviluppano in queste regioni, però, sono profondamente influenzati dalle norme sociali e culturali, dal contesto in cui ci si trova e dalle relazioni interpersonali presenti tra gli interlocutori \parencite{barrett_costruzionismo}.

Barrett, oltre a cogliere gli aspetti culturali che costituiscono l'esperienza emotiva, mette in luce anche l'importanza dell'etichettatura delle emozioni, ovvero l'assegnazione di determinate parole per riferirsi ad un'esperienza emotiva: l'importanza del lessico emotivo. L'utilizzo di un particolare lessico per descrivere uno stato emotivo influenza la percezione e la regolazione dell'emozione stessa. \\
La psicologa riesce quindi, attraverso le sue ricerche neuroscientifiche, a mettere in evidenza la rilevanza degli aspetti cognitivi relativi alle emozioni, l'influenza culturale sui processi interpretativi e l'importanza del lessico emotivo.

Il campo del costruzionismo sociale ha interessato oltre che psicologi e filosofi anche molti antropologi che, grazie ai loro studi sul campo, hanno apportato valido sostegno a questa prospettiva teorica. 

Verso la metà del Ventesimo secolo l'antropologo Clifford Geertz inizia a pubblicare i suoi studi in cui esprime il suo supporto alla prospettiva costruzionista della realtà. Il suo pensiero si centrava sull'antropologia simbolica, secondo la quale i simboli costruirebbero i significati socialmente condivisi dalla cultura.\\
La cultura, infatti, viene definita come «un sistema di concezioni ereditate, espresse in forme simboliche, attraverso le quali gli uomini comunicano, perpetuano e sviluppano le loro conoscenze e i loro atteggiamenti nei confronti della vita \parencite{geertz}».\\
Geerzt, riprendendo Max Weber, afferma che la cultura sia essenzialmente semiotica \footnote{La semiotica fa parte della scienza della comunicazione e viene definita come «scienza generale dei segni, della loro produzione, trasmissione e interpretazione, o dei modi in cui si comunica e si significa qualcosa, o si produce un oggetto comunque simbolico \parencite{semiotica}».}, e che quindi l'uomo si trovi all'interno dell'insieme di significati che egli stesso ha creato.\\
Questo concetto porta alla convinzione che l'analisi della cultura sia una scienza interpretativa alla ricerca del significato.\\
Conseguente al fatto che la cultura sia costituita da un insieme di simboli e di significati, infatti, viene messo in evidenza il principio guida dell'interpretazione, che permette di comprendere la vita degli individui in una particolare cultura \parencite{geertz}. 

Nella sua concettualizzazione della cultura egli sottolinea che anche le emozioni umane sono manufatti culturali, costituiti da significati condivisi; ma lo studio delle emozioni, in realtà, viene approfondito principalmente dalla moglie Hildred.\\
H. Geertz condusse vari studi antropologici nell'isola di Java, studiandone i processi di socializzazione. Nelle sue produzioni scientifiche riporta la concezione secondo la quale negli esseri umani sono presenti alcune emozioni culturalmente universali \parencite{hildred_geertz}.\\
Queste, però, sono soggette a diversi influssi e condizionamenti nel corso della socializzazione infantile, la quale è variabile a seconda delle diverse culture in cui si è inseriti. A seconda di tale cultura e dalle norme socialmente condivise in essa, infatti, verrebbero promosse alcune emozioni e smorzate altre e, proprio per questo motivo, le emozioni risulterebbero culturalmente diverse. 

Un'altra antropologa che ha trattato il tema delle emozioni a livello culturale e si è preoccupata di mettere in evidenza l'importanza del linguaggio in quest'ambito, è Catherine Lutz.\\
Nei suoi studi concernenti il lessico emotivo, Lutz afferma che le parole emotive non sono universali, ma vengono influenzate dalla cultura e dal contesto sociale in cui ci si trova. Essendo costruite socialmente e culturalmente, le parole che compongono il lessico emotivo, riflettono le norme e i valori condivisi della società in cui vengono utilizzate \parencite{lutz_politics_emotions}.\\
Il contesto culturale, quindi, assume una grandissima importanza, soprattutto per quanto riguarda la comprensione dei processi emotivi attraverso il lessico di riferimento. Quest'ultimo, infatti, può includere parole emotive che descrivono emozioni specifiche che non esistono in altre lingue, o che descrivono le emozioni in modo diverso a seconda del contesto culturale e situazionale differente.

L'antropologa mette in luce la grande importanza del lessico emotivo attraverso le sue molteplici funzioni che può avere nei confronti dei processi emotivi.\\
Il linguaggio utilizzato per descrivere una determinata esperienza emotiva, può anche influenzare la stessa: impiegare specifiche parole in un contesto può condizionare l'esperienza emotiva di chi partecipa a tale contesto sociale.\\
Un'altra funzione fondamentale del lessico emotivo che si riscontra è la regolazione delle emozioni, che può essere raggiunta attraverso l'etichettatura e la condivisione degli stati emotivi con gli altri, attraverso il linguaggio e la comunicazione\footnote{Questa argomentazione di Lutz può essere confermata dagli esperimenti condotti sull'etichettuatura delle emozioni, viste approfonditamente nel \autoref{par: Affect labeling}} \parencite{lutz_cultural_category}.\\
Tutti gli apporti di Lutz descritti finora hanno avuto un impatto significativo sulla comprensione delle emozioni e della loro relazione con il linguaggio e la cultura; fornendo una base teorica solida per lo studio delle parole emotive come un fenomeno culturale e sociale complesso.

L'ultimo ambito di studio che ha preso in carico il lessico emotivo in relazione alla prospettiva socio-costruttivista delle emozioni è sicuramente quello della linguistica. 

Un linguista e filosofo che ha dato un grande apporto allo studio del lessico emotivo è Reddy, il quale si concentra sulla semantica e sulla relazione di questa con la cultura, sviluppando la teoria degli \textit{"emotives"}.\\
Con questo termine, Reddy, rappresenta tutte le parole facenti parte del linguaggio emotivo che, attraverso la costruzione, indicano emozioni o valutazioni soggettive: vengono utilizzate per descrivere ciò che viene definito "indescrivibile", ovvero "come ci si sente" \parencite{reddy_no_costruzionismo}.\\
Importante, inoltre, è la specificazione della differenza tra un uso emotivo e uno descrittivo del linguaggio: si ha un'intenzionalità ben diversa. Quest'ultimo viene utilizzato con l'intenzione di comunicare significati descrittivi, mentre il linguaggio emotivo, proprio delle \textit{emotives}, vuole tradurre a parole un'emozione, con l'intenzione specifica di suscitare nell'altro uno stato emotivo preciso \parencite{reddy_articolo}. \\
Le \textit{emotives}, attraverso l'atto comunicativo, hanno anche la possibilità di costruire, modificare, nascondere o intensificare le emozioni stesse.\\
Da quest'affermazione si può comprendere che il lessico emotivo, attraverso la teoria di Reddy, acquisisce il ruolo di costruttore della realtà sociale.\\
Essendo il lessico costruito sulle basi di una specifica lingua socialmente e culturalmente influenzata, la sua costruzione delle emozioni varierà da cultura a cultura e genererà dunque emozioni non universali, ma socio-costruite. \\
Reddy, grazie ai suoi apporti teorici, ha contribuito allo studio approfondito del lessico emotivo, legato alle variabili culturali, sociali e contestuali che lo influenzano; permettendo uno studio approfondito delle emozioni attraverso il linguaggio. 

Un altro linguista del Ventesimo secolo che tratta il tema del lessico emotivo e, in particolar modo, delle metafore utilizzate nell'ambito emotivo è Zoltan Kovecses.\\
Il lessico emotivo, secondo la sua prospettiva, è un sistema complesso, costituito da metafore, schemi concettuali e processi cognitivi specifici che influenzano l'espressione delle emozioni; si differenzia quindi dagli studiosi secondo cui il lessico emotivo è definito solamente come l'insieme di parole che descrivono gli stati emotivi \parencite{kovecses_articolo}. \\
Egli si concentra molto sullo studio delle metafore, che portano alla comprensione delle emozioni. Per descrivere gli stati emotivi si possono utilizzare più metafore, che vengono identificate come subordinate di una metafora principale, la quale indica i concetti principali di quella determinata emozione.\\
Un esempio che potrebbe aiutare la comprensione di tale concetto sono le metafore legate all'emozione di rabbia. Si possono utilizzare molte metafore per descrivere il concetto di rabbia, come “sbuffare, esplodere, far ribollire il sangue, schiumare”. Tutte le immagini evocate da tali metafore possono ricondurre a sensazioni di calore, pericolo, forte intensità e perdita di controllo. Come si può osservare vi è una metafora principale che è quella della forza, energia\clearpage e violenza, dalle quale possono nascere diverse metafore ad esse subordinate \parencite{kovecses_articolo}. 

Kovecses, partendo dallo studio delle metafore, si interroga sulla dicotomia tra universalismo e socio-costruttivismo delle emozioni \footnote{Dicotomia presa in considerazione proprio in questi capitoli.}, ritenendola poco produttivo.\\
Proprio per questo motivo propone una mediazione tra i due poli opposti, prendendo in considerazione sia gli elementi universali delle emozioni, che quelli costruiti socialmente \parencite{kovecses_libro_metafore}. Secondo la sua teoria gli elementi universali riguarderebbero l'attivazione fisiologica scaturita dagli stimoli emotivi, dunque la dimensione corporea, sulla quale poi verrebbero formulate delle metafore.
La rabbia, ad esempio, viene associata ad un aumento della temperatura corporea, un aumento del battito cardiaco, eccetera. \\
Gli elementi fisiologici determinano l'ambito entro il quale vengono formulate le metafore centrali e universali di tale emozioni, nel caso quella rabbia questa sarebbe della del “contenitore sotto pressione”.

Questi concetti sono stati già menzionati nella descrizione del pensiero di Wierzbicka. Anche la linguista polacca, infatti, cerca di includere sia gli aspetti legati all'universalismo che quelli di matrice socio-costruttivista nella descrizione delle emozioni.\\
Nel suo libro afferma che tutti gli individui possono provare esperienze soggettive simili a livello esperienziale, corporeo, denominando le emozioni proprio "universali esperienziali". Dall'altra parte, però, proprio come Kovecses, dà all'ambito linguistico e semantico il ruolo socio-costruttivista, andando così a contraddire la concezione universalista delle emozioni.\\
I pensieri dei due linguisti trovano grandi analogie generali, ma si può notare come Kovecses si concentri molto più profondamente sugli aspetti corporei, legandoli anche alle metafore linguistiche, mentre Wierzbicka lasci più spazio allo studio semantico e grammaticale del linguaggio trovandone somiglianza e differenze nelle varie culture. 

Tornando allo studio del lessico emotivo di Kovecses, possiamo incontrare molto facilmente l'aspetto del costruzionismo sociale. \\
Secondo il linguista ungherese, infatti, il lessico emotivo viene  utilizzato principalmente in modo figurato e costruito culturalmente: le parole utilizzate per descrivere le emozioni sono sottoposte all'influenza delle esperienze culturali personali e delle norme socialmente condivise, che possono condurre ad una diversa quantità di vocaboli per descrivere la stessa emozione in base alla cultura in cui ci si trova \parencite{kovecses_libro_metafore}.\\
Di grande importanze viene anche ritenuto il contesto situazionale e relazionale in cui le parole vengono utilizzate, che può cambiare il loro significato.

L'importanza delle metafore come strumento per comunicare gli stati emotivi veniva già citata dai linguisti Lakoff e Johnson, negli anni Ottanta. 
Data la grande difficoltà nella descrizione degli stati emotivi, gli autori affermano che l'individuo si serve delle metafore per riuscire ad esprimersi  più facilmente ed in modo indiretto, facendo riferimento ad analogie con altre esperienze.\\
Identificano diversi tipi di metafore tra cui: corporee, di azione, di sostanza e di spazio, facendo riferimento proprio ai fenomeni a cui l'individuo fa riferimento per esprimere le emozioni attraverso le analogie \footnote{Ad esempio, le metafore corporee si riferiscono alle emozioni attraverso i termini relativi alle sensazioni fisiche, come "sentirsi pesanti" \parencite{lakoff_johnson}.} \parencite{lakoff_johnson}.\\
Le metafore sono profondamente influenzate dalla cultura e dal contesto sociale; pertanto si può ricorrere all'utilizzo di diverse metafore per descrivere le stesse emozioni. 

Sempre nell'ambito della linguistica, il tedesco Martin Haspelmath, si occupa dello studio della semantica del lessico emotivo e delle differenze tra le diverse lingue.\\
Il suo pensiero si pone a metà tra la prospettiva universalista e quella sociocostruttivista.\\
Egli, infatti, ritiene che esistano parole emotive universali, presenti in tutte le lingue, ma che la semantica di tali parole e la loro distribuzione (ovvero la quantità di vocaboli disponibili per parlare delle emozioni) subiscano l'influenza culturale che porta ad una loro grande differenziazione in base alla lingua parlata.\\
Viene anche evidenziata la presenza di alcune parole specifiche esistenti solo in determinate lingue, e non in altre \parencite{haspelmath}.\\
Per riuscire a comprendere al meglio l'espressione delle emozioni attraverso il linguaggio, risulta quindi fondamentale analizzare il contesto culturale e situazionale in cui si è inseriti. \\
Seguendo questa corrente di pensiero, Haspelmath, studia la semantica del lessico emotivo in relazione al lessico generale della lingua in questione \parencite{Haspelmath_articolo}.\\
Egli sostiene che le parole utilizzate nella descrizione emotiva possiedono una base semantica simile ad altre parole nella lingua. Tale relazione semantica tra le parole emotive e il lessico generale può portare alla comprensione ed espressione delle emozioni in quella lingua. 

Haspelmath ha dunque contribuito in modo considerevole alla comprensione delle emozioni attraverso il linguaggio, alla semantica di esso e alle differenze del lessico emotivo in relazione al contesto sociale e culturale. 

Attraverso l’esposizione delle diverse teorie sulle emozioni e sul linguaggio emotivo sistematizzate dai diversi psicologi, linguisti, antropologi e filosofi che hanno focalizzato l’attenzione su tale oggetto di indagine, si possono trarre alcune considerazioni generali. 

Le due prospettive teoriche prese in considerazione vengono facilmente viste come due poli teorici opposti: le emozioni possono nascere in concomitanza con l'individuo, come meccanismi cerebrali innati; oppure in un ambiente sociale e culturale, divenendo un prodotto della costruzione socio-culturale.\\
Ciò che si è visto prendendo in considerazione il pensiero dei vari autori, però, è che spesso tali prospettive sono complementari e non contrapposte, risultando così entrambe necessarie per comprendere le emozioni e la loro genesi.\\
La dicotomia tra universalismo e socio-costruttivismo è infatti ad oggi superata, ma ciò che è importante comprendere è in che maniera e misura gli elementi universali e quelli socio-costruiti influenzano i processi emotivi. Si è visto, ad esempio, che gli aspetti legati all'innatismo e comuni a tutti gli individui sembrano appartenere più alla sfera di attivazione fisiologica, mentre quelli che si differenziano da cultura a cultura sembrano essere legati alla sfera semantica delle emozioni. Più approfonditamente il processo di costruzione sociale e culturale è stato fatto proprio del lessico emotivo.\\
Quest'ultimo pare essere di fondamentale rilevanza nell'ambito semantico dei processi emotivi, risultando esso stesso costruito culturalmente e socialmente, ma anche in grado di modificare e costruire le emozioni attraverso il suo utilizzo. \\
Questo è uno dei focus principali che danno origine a vari dibattiti in questo campo: quanto le emozioni siano in grado di essere costruite dal lessico, e quindi dalla cultura e dalla società e quanto siano invece innate ed universali.

Ciò che si è potuto sicuramente comprendere è che le emozioni sono caratterizzate da un'estrema complessità, data anche dalla loro influenza da innumerevoli variabili di tipo biologico, culturale e sociale e, soprattutto, dalla loro difficile misurazione, essendo processi non direttamente osservabili e difficilmente descrivibili verbalmente. 

Partendo da tali considerazioni, il prossimo capitolo cercherà di approfondire gli aspetti socio-culturali legati al lessico emotivo, andando ad indagarne le differenze nelle varie lingue, e analizzandolo come strumento per esplorare per i processi emotivi.


\chapter{Differenze culturali del lessico emotivo} 
\label{chap: Capitolo 3}
\section{Studi antropologici}
Ciò che è stato osservato negli studi di matrice socio-costruttivista sul lessico emotivo è che questo si sviluppa all'interno di un contesto sociale e culturale specifico. Di conseguenza, le parole utilizzate per descrivere le emozioni sarebbero legate alle credenze, i valori, le norme e le pratiche di  una specifica comunità e differirebbero notevolmente in base alla lingua parlata.\\
Ad esempio, alcune culture potrebbero avere parole specifiche per descrivere emozioni non presenti in altre, oppure possedere vocaboli che si riferiscono a certe emozioni in modo più preciso e dettagliato o, ancora, le differenze culturali possono influire anche la frequenza con cui le persone esprimono determinate emozioni e sulla loro disposizione a mostrare apertamente le loro emozioni in pubblico.\\
Le norme socio-culturali, quindi, possono influenzare sia il tipo di emozioni che vengono promosse, sia l'intensità con la quale vengono espresse pubblicamente.

Le norme sociali e le differenze culturali che condizionano gli stati emotivi sono state studiate approfonditamente da parte dell'antropologia, che ha dato un grande apporto anche agli studi sul lessico emotivo.\\
Molti antropologi, alcuni dei quali già menzionati nel \autoref{chap: Lessico emotivo}, hanno condotto studi sul campo, immergendosi nella realtà di una specifica comunità per svariato tempo hanno cercato di comprendere al meglio come gli individui vivessero le proprie emozioni, quali fossero quelle prevalenti e i termini utilizzati per descriverle. 

Una di queste è sicuramente l'antropologa statunitense Catherine Lutz, la quale nel 1977 condusse uno studio sui popoli della Micronesia, in particolare sul popolo Ifaluk, abitanti le isole Caroline nel Pacifico. \\
Nella cultura di questo popolo le emozioni risultano essere centrali e vengono integrate alla sfera spirituale e religiosa, quindi sono parte di un sistema di valori più ampio. \\
Per quanto riguarda le emozioni specifiche riscontrate, l'antropologa afferma la forte rilevanza delle emozioni negative, che non vengono controllate o represse come nella cultura occidentale, ma espresse ed accettate. \\
Le emozioni principali del popolo Ifaluk sembrano essere due, denominate \textit{fago} e \textit{song}: la prima corrisponde ad una combinazione di amore, compassione e tristezza e si prova principalmente nel confronti di un individuo che manifesta uno stato di bisogno, è quindi legata al prendersi cura dell'altro, un interesse altruistico verso la comunità. L'emozione \textit{song}, invece, rappresenta una rabbia ingiustificata o indignazione morale contro qualcuno che trasgredisce valori fondamentali condivisi nella comunità. \\
Tali emozioni vengono considerati complementari l'una dell'altra: è necessario provare compassione e interesse nei confronti dell'altro per poi provare indignazione quando tali atteggiamenti di cura altrui vengono trasgrediti o non condivisi dall’intera comunità \parencite{lutz_micronesia}. \\
Attraverso questa ricerca si può notare, oltre che la presenza di diverse parole emotive che rappresentano le emozioni assenti nel mondo occidentale, anche il fatto di trovarvi riflesse le norme di una società all'interno di questi vocaboli, che divengono espressione della diversa esperienza emotiva espressa dal popolo in questione. 

Abu-Lughod, antropologa che ha lavorato con Lutz, ha condotto uno studio sulle donne beduine e sulla loro esperienza emotiva. \\
Il primo elemento che pone in evidenza è che in presenza di uomini, le donne beduine dovevano attenersi a rigide regole di pudore e sottomissione impartite dal codice morale \textit{Hasham}. Questo implicava il divieto di esprimere o la necessità di camuffare le proprie emozioni come ad esempio la gelosia, che però potevano poi esprimere nell'ambito privato, tramite canti carichi emotivamente denominati \textit{Ghinnawa} \parencite{abu_lughod}. \\
Questo aspetto di camuffamento delle emozioni dovuto a norme imposte dalla società in una specifica cultura si rifà al concetto di "lavoro emotivo", ovvero il mascheramento delle proprie emozioni in pubblico per adeguarsi alle regole emanate dal contesto culturale, e la possibilità di esprimerle nel privato. Il lavoro emotivo può essere superficiale, quindi una modificazione a livello espressivo delle emozioni, oppure profondo, che si riferisce ad una modificazione e gestione interna delle emozioni che vengono provate \parencite{hochschild}.\\
Questo ci fa apprendere quanto le norme culturali siano in grado di influenzare sia l'espressione esterna sia la capacità di modificare gli stati emotivi che si provano internamente.

Oltre alla diversa manifestazione delle emozioni dovuta alle norme sociali e culturali si possono notare differenze culturali interlinguistiche per quanto riguarda vocaboli che indicano emozioni specifiche di una comunità, non esistenti in nessun'altra. \\
Per fare alcuni esempi possiamo citare un'emozione identificata in studi antropologici della Papua Nuova Guinea chiamata \textit{awumbuk}, che descrive un senso di stanchezza ed esaurimento dovuto al commiato di un ospite dopo aver trascorso una notte nella propria casa \parencite{fajans}. \clearpage
Levy, nel suo studio sul popolo tahitiano, notò la presenza di alcune parole specifiche per descrivere emozioni di rabbia e odio, che sembrano emozioni molto rilevanti nella loro cultura, ma anche l'assenza di vocaboli per descrivere altre emozioni come quelle di affetto, sentimento o passione \parencite{levy_thaitians}. \\
Oppure, ancora, nella lingua tedesca possiamo trovare la parola \textit{Weltschmerz}, spesso usata in ambito letterario, che significa "dolore del mondo" e sta ad indicare un senso profondo di tristezza e insoddisfazione esistenziale nei confronti della condizione umana e delle sofferenze condivise all'interno di un contesto \parencite{goethe}. \\
L'elenco di questi vocaboli è ampissimo e la loro esistenza sembra dimostrare quanto le emozioni si differenzino in base alla cultura in cui si vive e quanto le specificità culturali e sociali si riflettano nella lingua parlata, quindi nel lessico emotivo. 

Il linguaggio, quindi, diventa un importantissimo strumento per andare ad indagare come gli stati emotivi interni vengano concettualizzati nelle varie culture e per metterli a confronto tra loro. \\
Lo studio del lessico emotivo risulta assai complesso e, per poterlo analizzare al meglio e permettere il confronto tra più lingue diverse, sono state individuate dimensioni affettive specifiche in ciascuna lingua presa in considerazione. 

\section{Modelli dimensionali}
Linguisti e psicologi cercano di analizzare le somiglianze e le differenze delle parole utilizzate per parlare delle emozioni, andando ad indagare se fattori come la vicinanza geografica o la famiglia linguistica possano influenzare tali relazioni. \\
Un metodo utilizzato che permette una dettagliata indagine del lessico emotivo e il confronto tra più lingue risulta essere quello di analizzare le varie dimensioni affettive delle emozioni, ovvero proprietà che forniscono significato e che vengono ritrovate nei termini usati per descrivere le emozioni. Risultano quindi identificabili nell'organizzazione semantica del lessico emotivo e potrebbero essere indice della manifestazione e dell'esperienza stessa delle emozioni.\\
Tali dimensioni, proprie della valutazione affettiva, hanno dato vita a diversi modelli, tra cui i principali definiti tridimensionali o bidimensionali, in base al numero di dimensioni rilevate nell'analisi del lessico emotivo delle lingue prese in esame \parencite{model}. 

Uno dei modelli principali è quello di Russell, noto come teoria circomplessa delle emozioni, che si concentra sulla descrizione organizzativa delle emozioni e sulla relazione tra esse.\\
Il modello circomplesso è bidimensionale: in esso le emozioni prese in considerazione si possono rappresentare in uno spazio circolare, caratterizzato da due assi, che rappresentano due dimensioni bipolari indipendenti. L'asse orizzontale si riferisce alla dimensione edonica, ovvero un continuum tra piacere e dispiacere; mentre l'asse verticale delinea il grado di attivazione, quindi un continuum che va dalla calma all'eccitazione fisiologica o cognitiva.\\
Attraverso questi due assi vengono introdotti due concetti importanti: quello di valenza, che coincide con l'asse orizzontale proprio del grado di piacevolezza e quello di \textit{arousal}, corrispondente all'asse verticale dell'attivazione fisica o mentale \parencite{russell_circumplex}. \\
Secondo Russell gran parte delle emozioni possono essere descritte utilizzando solamente questi due fattori, in quanto la combinazione tra essi sembra dare luogo alle diverse sfumature di significato dell’esperienza emozionale, rappresentabile mediante parole che si collocano nello spazio cartesiano che si struttura attorno a tali assi dimensionali. 

Il modello circomplesso può essere un valido strumento per studiare l'esperienza emotiva umana, analizzata attraverso \textit{self-report} \footnote{Per riuscire a studiare le emozioni umane, lo strumento più facile che possiamo utilizzare è l'intervista diretta. Viene chiesto ai partecipanti come ci si sente e la dichiarazioni fatte, ossia i \textit{self-report}, vengono trattate come comportamenti verbali che possono essere studiati e osservati. Questo processo, però, è molto complesso a causa delle diverse difficoltà comunicative: una domanda chiave è se le differenze nei \textit{self-report} riflettano i sentimenti reali o semplicemente differenze nel modo in cui le persone comprendono le parole usate nel processo di valutazione \parencite{barrett}.}. Nell'analisi di questi ultimi si va ad osservare quali sono le dimensioni che più vengono enfatizzate dall'individuo, si parla quindi di \textit{valence focus} per quanto riguarda la concentrazione sulla dimensione valutativa o \textit{arousal focus} relativa alla dimensione di eccitazione \parencite{barrett}.

Vari autori hanno proposto modelli bidimensionali simili, che includono sempre dimensioni relative all'\textit{arousal} e alla valenza \parencite{Larsen}. Altri, invece, pur prendendo in considerazione due dimensioni, si distinguono da quello di Russell.\\
Per esempio, il modello bidimensionale di Watson e Tellegen individua come dimensioni l'affetto positivo e l'affetto negativo. Tale modello, però, descrive la presenza di due scale unipolari, che sono costituite da una "miscela" degli aspetti di valenza e attivazione \footnote{I punteggi elevati rilevati nella scala dell'affetto positivo rappresentavano stati emotivi positivi con elevato \textit{arousal}; quelli elevati nell'asse dell'affetto negativo invece, corrispondevano a stati emotivi negativi, sempre con elevata attivazione fisiologica \parencite{Watson}.}.\\
Qui, infatti, l'elemento distintivo e piuttosto rilevante è la presenza di due dimensioni ortogonali e non di un'unica scala in cui l'affetto negativo e positivo ne rappresentano i poli opposti. Questo dà la possibilità di rappresentare stati emotivi con una valenza mista, sia negativa che positiva \parencite{Watson}.

Ciascuno di questi modelli bidimensionali porta con sé argomentazioni valide, le quali mantengono vivo il dibattito sul grado di applicazione di ciascun modello alle diverse lingue e culture, che riesce ad ampliarsi ancora di più con i modelli tridimensionali, ovvero quelli che prendono in considerazione tre dimensioni diverse.

La loro storia inizia con Wundt, il quale descrive le dimensioni individuate con lo studio dei termini emotivi: parla di "piacere-dispiacere", che si traduce nella valenza per i modelli bidimensionali; "eccitazione-calma" e "tensione-rilassamento", che rappresentano l'attivazione fisiologica \parencite{wundt}.\\
Come fa notare Thayer, però, c'è una grande differenza tra i due tipi di attivazione fisiologica poiché mentre l'eccitazione energetica può scaturire da uno stimolo molto positivo, quella tensiva può essere frutto di una situazione di stress e tensione molto forte \parencite{Thayer}.\\
Successivamente, uno dei modelli tridimensionali di maggiore rilevanza è stato quello proposto da Schimmack e Grob, che vede come dimensioni principali per descrivere le emozioni: piacere-dispiacere, veglia-stanchezza e tensione-rilassamento, che lo fanno risultare molto simile al modello originale di Wundt \parencite{Schimmack}. 

Modelli storicamente anche precedenti da quello descritto, ma che si concentrano più approfonditamente sull'aspetto della semantica degli stimoli emotivi, vengono presi in considerazione soprattutto nello studio del lessico emotivo.\\
Il modello principale è stato quello di Mehrabian, che presenta tre dimensioni: quella di piacere, di eccitazione e di dominanza \parencite{Mehrabian}. Tale modello, però, deriva da quello proposto da Osgood, che descriveva precisamente il significato affettivo delle parole.

Il modello tridimensionale di Osgood rileva l'esistenza di tre diverse dimensioni che le parole possono esprimere implicitamente o esplicitamente: la valutazione, ovvero il senso di piacere o dispiacere, l'attività, che si riferisce al livello di attivazione fisiologica, e la potenza, che corrisponde alla capacità di affrontare le sfide ambientali da parte dell'individuo \parencite{Osgood}. \\
Tali dimensioni vengono riprese anche da altri autori, tra cui Davitz, il quale parla di tono edonico, corrispondente alla dimensione di piacere o dispiacere \parencite{Davitz}; \textit{arousal} o attivazione per quanto riguarda l'attivazione fisiologica e competenza, che corrisponde alla potenza di Osgood. Egli, inoltre, aggiunge una quarta dimensione: quella della relazionalità, che si riferisce alla relazione con le persone e con l'ambiente, la quale influenza fortemente l'esperienza emotiva. 

Un altro modello utilizzato in ricerche psicologiche più recenti è quello di \textit{scaling} multidimensionale (MDS). In alcuni studi condotti sulla lingua inglese però, si è notato che questo modello ha praticamente replicato le tre dimensioni del modello di Osgood, non aggiungendone altre \parencite{russell_multidimensional}. \\
Attraverso diversi studi si è quindi compreso che le dimensioni strutturali del lessico emotivo che si possono incontrare attraverso la sua analisi sembrano essere principalmente due: la valenza e l'attivazione fisiologica. 

Per concludere, i modelli principali strutturano gli stati emotivi e la loro descrizione attraverso il lessico emotivo includono principalmente due o tre dimensioni. Si è vista inoltre la tendenza per gli autori americani alla creazione di modelli bidimensionali, mentre per gli autori europei di quelli tridimensionali \parencite{model}.\\
Tali modelli vengono realizzati solitamente attraverso self-report, auto dichiarazioni che utilizzano tendenzialmente la lingua inglese e, quindi, fanno riferimento solo a termini inglesi o termini traducibili in inglese per descrivere le emozioni, escludendo l'enorme possibilità di variabilità linguistica legata all'ambito emotivo. Questo conduce ad un dubbio attuale sulla possibilità di un'applicazione di tali modelli a diverse lingue e culture per studiarne le emozioni.\\
Questo dovuto anche alla supposizione di un'influenza da parte di fattori culturali e sociali molto ampia nella struttura dimensionale degli affetti, che deve essere esaminata al meglio.

\section{Variazioni interlinguistiche}
Partendo da queste premesse teoriche si possono quindi andare ad analizzare i risultati di diversi studi che vogliono indagare le differenze dell'ambito emotivo in diverse culture, attraverso l'analisi semantica del lessico emotivo.\\
Verranno prese in considerazione le differenti dimensioni esplicitate dai modelli sopracitati e verranno comparate nelle varie lingue, cercando di comprendere quali siano le analogie che accomunano certi gruppi linguistici e quali le differenze che li distanziano.

Come già accennato, gran parte delle ricerche empiriche condotte in questo campo si riferiscono alla lingua angloamericana o su termini traducibili in inglese, senza prendere in considerazione termini peculiari propri di ciascuna lingua \parencite{russell_multidimensional}. In questi casi i gruppi semantici identificati maggiormente come categorie concettuali discrete, comuni a gran parte delle lingue studiate, si riferiscono a quelle che Ekman chiama "emozioni di base" quindi: rabbia, gioia, sorpresa, paura, disgusto e tristezza \parencite{ekman}. Data la forte influenza della lingua inglese, però, tali studi non possono essere considerati pienamente validi a livello interlinguistico, si ritiene pertanto necessario approfondire la questione mediante  studi transculturali.  

Gli studi transculturali del lessico emotivo danno vita a diversi modelli dimensionali, mettendo in luce le dimensioni più rilevanti nelle diverse lingue prese in esame; quindi le dimensioni che ciascuna lingua permette di identificare. \\
I riscontri ottenuti attraverso diversi studi hanno portato a galla interessanti considerazioni. \\
Innanzitutto si è potuto notare che non prendendo in considerazione la lingua inglese le dimensioni individuate differiscono dalle sole valenza ed attivazione fisiologica, considerate generalmente le più rilevanti \parencite{Galati}. \\ 
Ciò che si è visto, infatti, è che la dimensione edonica è l'unica ad essere presente in tutte le lingue studiate: piacere e dispiacere sembrano essere imprescindibili per la descrizione delle emozioni, in alcuni casi è presente solamente quest'asse, ad esempio nella lingua ebraica \parencite{Fillenbaum}. 
La dimensione di potenza, invece, è stata identificata come dimensione rilevante solamente in alcuni casi, per esempio, studiando la lingua italiana si è visto che questa dimensione risulta più importante rispetto a quella dell'eccitazione fisiologica \parencite{Sini}. \\
Ciò non toglie la possibile compresenza di tutte e tre le dimensioni di potenza, attivazione e valenza, come è stato riscontrato in alcuni studi della lingua spagnola.

Uno studio specifico sulla famiglia linguistica delle lingue neolatine, ovvero: italiano, francese, castigliano, catalano, portoghese e rumeno \parencite{Sini}, ci permette di approfondire ancora di più l'argomento, dandoci così la possibilità di afferrare meglio la misura delle variazioni del lessico emotivo. \\
Lo studio di trentadue termini emotivi per ciascuna lingua neolatina ha permesso di rilevare diverse dimensioni emotive. Le dimensioni riscontrate come comuni a tutte le lingue analizzate sono la valenza edonica, quindi la presenza di termini linguistici che indicano una distinzione tra emozioni negative e positive, che danno dunque un senso di piacere o dispiacere; e l'attivazione fisiologica, data la presenza di parole che indicano una bassa o alta eccitazione comportamentale. \\
Alcuni esempi, per quanto riguarda la valenza possono essere i termini italiani felice/infelice, quelli in castigliano \textit{euforico/triste}, in catalano \textit{alegre/frustrat} e così via. Per quanto riguarda la dimensione dell'attivazione, invece, si possono incontrare diversi vocaboli che indicano stati di agitazione, ossia una forte attivazione fisiologica e, in contrapposizione, altri che si riferiscono a stati di calma, quindi a basse attivazioni fisiologiche, ad esempio i termini francesi \textit{agité/ désolé}, oppure quelli portoghesi \textit{agitado/acanhado} o ancora in rumeno \textit{infuriat/melancolic}. \\
Sugli assi delle dimensioni della valenza edonica e dell'attivazione fisiologica i termini sopracitati si pongono ai rispettivi poli opposti: per esempio, in castigliano \textit{euforico} si colloca al polo che rappresenta il punteggio più alto della valenza edonica, mentre \textit{triste} all'esatto polo opposto di tale asse.

L'ultima dimensione del modello tridimensionale esaminata è quella che si riferisce alla potenza, che, a differenza delle due precedenti, non è stata ritrovata in tutte e sei le lingue neolatine \parencite{Sini}.\\
La dimensione di potenza, infatti, può essere chiaramente individuata in italiano, rumeno e francese attraverso la presenza di termini che indicano irritazione, rabbia o disgusto, come il francese \textit{fâché}, l'italiano "irritato" o il rumeno \textit{aprins}. Questi termini si trovano in opposizione alle emozioni di significato contrario, che possono indicare spavento o paura come \textit{effrayé} in francese oppure "pauroso" in italiano.\\
In castigliano e catalano questa dimensione viene riconosciuta come un insieme delle dimensioni di potenza e attivazione fisiologica: mentre in castigliano troviamo parole come \textit{abochornado} o \textit{asustado}, in catalano incontriamo \textit{atemorit} o \textit{horrorit}, termini che indicano stati emotivi di spavento e terrore. Infine, in portoghese, la dimensione di potenza sembra ricondurre alla valutazione dell'«adeguatezza del comportamento alle proprie norme interne \parencite{Sini}», ad esempio, attraverso il vocabolo \textit{orgulhoso} o \textit{envergonhado}, che indicano orgoglio o vergogna. 

Questa dimensione e quella dell'attivazione fisiologica variano parecchio nelle diverse lingue, con alcune di esse in cui risulta più rilevante la dimensione dell'eccitazione rispetto a quella delle potenza, e viceversa. Tali dimensioni, inoltre, sembrano fare riferimento a diverse strategie di adattamento dell'individuo all'ambiente: modalità con cui vengono affrontate le sfide ambientali e i cambiamenti fisiologici e psichici. \\
Partendo dalle differenze sopraelencate riguardanti queste differenze, quindi, è possibile fare delle inferenze su come gli individui che utilizzano una specifica lingua affrontino le sfide ambientali a loro sottoposte. \\
Considerando il catalano e il castigliano, ad esempio, si può osservare che i termini che si riferiscono alle dimensioni di attivazione fisiologica indicano i meccanismi di \textit{coping}, andando a distinguere reazioni aggressive e comportamenti di fuga, contrapposte ai comportamenti di riposo \parencite{Sini}. 

Attraverso quest'analisi dettagliata delle sei lingue neolatine e delle dimensioni emotive presenti in esse, si è concluso che per tale studio il modello tridimensionale di Osgood risulta il più adeguato. In questo caso, infatti, sono fondamentali le dimensioni di valenza ed attivazione, ma anche quella riferita alla potenza, che gioca un ruolo di rilievo. \\
Ciò che si è anche notato è che mentre la dimensione di valutazione edonica risulta sempre la più importante, le altre due presentano diversa salienza nelle sei lingue. \\
In italiano, francese, catalano e castigliano, la seconda dimensione più rilevante è la potenza, seguita dall'attivazione; mentre in rumeno e in portoghese la situazione è opposta, vedendo come seconda l'attivazione fisiologica \parencite{Sini}. 

Ciò che è interessante indagare sono le cause di tali somiglianze nelle lingue indicate: una probabile spiegazione è quella della distanza geografica, vedendo che le lingue più simili sono anche le più vicine geograficamente. \\
Inoltre, le lingue neolatine che presentano una predilezione per la dimensione della potenza hanno un'origine latina e greca, ovvero lingue nate da culture che pongono al centro l'attività cognitiva e quindi meno l'attivazione fisiologica. Quest'ultima considerazione andrebbe quindi a sostenere una causa di tipo storico, sociale e culturale: secondo questa spiegazione sarebbero la cultura greca e latina con le loro norme sociali ad influenzare il lessico emotivo. 

Dopo aver visto un'indagine su una famiglia linguistica specifica, si andrà ora ad approfondire uno degli studi interculturali più ambiziosi, che analizza il lessico emotivo di 20 famiglie linguistiche.\\
Si tratta dello studio è di Jackson e altri psicologici, i quali presentano una ricerca interlinguistica che prende in considerazione le reti semantiche di oltre un terzo delle lingue del mondo, mostrando la grande diversità con cui i concetti di emozione vengono espressi nelle diverse culture \parencite{majid_jackson}.\\
Per analizzare il modo in cui i concetti di emozione sono collegati tra loro, gli autori utilizzano quelle che vengono chiamate "colessificazioni", ovvero i casi in cui una singola parola si riferisce a più concetti: nella lingua persiana, ad esempio, non esistono due parole distinte per indicare "dolore" e "rimpianto", ma una soltanto \textit{ænduh}, che li indica entrambi \parencite{jackson_joshua}. Il metodo che implica lo studio delle colessificazioni risulta parecchio favorevole per ricavare informazioni sulla struttura semantica del lessico emotivo, poiché queste vengono spesso riscontrare quando due parole vengono percepite come concettualmente simili. 

Andando ad analizzare le reti semantiche di tali colessificazioni in 2.474 lingue, gli autori hanno riportato la presenza di una grandissima variazione semantica nei concetti di emozione. I vocaboli riferiti alle emozioni sembrano avere diversi modelli di associazione a seconda delle famiglie linguistiche indagate. Ad esempio, prendendo in considerazione le lingue austroasiatiche \footnote{La famiglia delle lingue austroasiatiche comprende 169 lingue, parlate nel sud-est asiatico e India \parencite{austroasiatiche}.} e quelle tai-kadai \footnote{Si tratta di una famiglia linguistica, composta da circa 90 lingue utilizzate nell'indocina, ovvero nell'Asia sud-orientale.}, viene riportato che mentre nelle prime il concetto di "ansia" è strettamente correlato al "dolore" e al "rimpianto"; nelle seconde questo viene maggiormente legato alla "paura". 

Le variazioni riscontrare riguardanti le dimensioni delle emozioni appartengono soprattutto a due: la valenza e l'eccitazione; queste due dimensioni, infatti, sono considerate «i vincoli universali alla variabilità della semantica delle emozioni \parencite{jackson_joshua}», ciò che più differenzia il lessico emotivo nelle diverse lingue parlate. \\
Ciò che viene osservato in questo studio è la presenza di una maggiore eccitazione fisiologica e mentale nelle culture individualiste, quindi principalmente nel mondo occidentale; in contrapposizione, le culture orientali, culturalmente collettiviste, prediligono una bassa eccitazione. \\ 
Prendendo come esempio la felicità si è potuto rilevare che, mentre nelle culture occidentali questa viene associata all'essere ottimisti, presentando la dimensione dell'eccitazione in grande misura; nelle culture orientali alla felicità viene assegnato il significato di solennità e riservatezza, che è associata ad una minore eccitazione fisiologica. \\
La dimensione della valenza presenta i due poli opposti di valenza positiva e negativa: ciò che si è riscontrato nello studio di colessificazione è la quasi totale assenza di emozioni con valenza positiva e negativa nella stessa comunità di colessificazione; con alcune eccezioni peculiari, che possono far comprendere ancora meglio la vastità di differenze del lessico emotivo che si possono trovare. \\
Ad esempio, in alcune lingue austronesiane vengono uniti i concetti di "pietà" e "amore": questo significa che, rispetto ad altre lingue, queste possono presentare una concettualizzazione di "amore" più negativa oppure di "pietà" più positiva \parencite{jackson_joshua}.

La globale differenza delle dimensioni di attivazione e valenza nella maggior parte delle culture occidentali ed orientali, confermerebbe la presenza di pattern culturali simili in base alla vicinanza geografica delle lingue.\\ Famiglie linguistiche più vicine, infatti, tendono a colessificare i concetti di emozione in modi più simili rispetto a quelle geograficamente più lontane.

Questa deduzione apre diverse questioni sulle cause di tale somiglianza basata sulla posizione geografica. Alcune ipotesi potrebbero essere la presa in prestito di diversi concetti tra lingue geograficamente vicine, o la presenza di un antenato linguistico comune. Questa seconda ipotesi interpretativa parte dall'assunto che lingue molto simili tra loro, probabilmente posseggono una lingua primordiale comune, che condurrebbe alla presenza di un lessico emotivo affine. Ad esempio, l'inglese e il tedesco sono entrambi lingue germaniche, quindi tutte e due originate dalla lingua proto-germanica; è possibile, quindi, trovare molte somiglianze nel lessico utilizzato per descrivere concetti emotivi e, conseguentemente, nella rilevanza delle diverse dimensioni. 

Altri fattori che possono aver influenzato tali analogie linguistiche potrebbero anche riguardare il commercio, le conquiste territoriali, i flussi migratori. Gli aspetti storici e sociali avrebbero quindi influito parecchio sulla concettualizzazione delle emozioni: questi aspetti incoraggiano la ricerca sull'analisi dei processi di trasmissione verticale ed orizzontale, che conducono alle variazioni nella semantica delle emozioni \parencite{majid_jackson}. \\
Tale spiegazione sosterrebbe il pensiero costruzionista, che propende per una spiegazione della natura delle emozioni come qualcosa di appreso, che conduce alla loro differenziazione interculturale. 

Lo studio di Jackson et al., oltre a mettere in luce l'aspetto dell'influenza geografica sul lessico emotivo utilizzato, pone al centro della discussione la presenza di concetti preesistenti a cui le parole fanno riferimento a cui si giunge mediante studi linguistici che analizzano termini polisemici o monosemici\footnote{In linguistica è presente un forte dibattito in cui alcuni linguisti si trovano a favore dell'analisi di significato in termini di polisemia, quindi prendendo in considerazione concetti multipli, oppure di monosemia, quindi un'analisi effettuata solamente attraverso concetti unitari \parencite{majid_jackson}.}. \\
Ciò che viene messo in discussione è l'esistenza di concetti universali ed innati data la grande variazione culturale che si è riscontrata studiando le molteplici lingue e la presenza di molte colessificazioni.\\
Un punto cruciale di questo dibattito riguarda la presenza di singole parole che si riferiscono a più concetti: ci si domanda, infatti, se nelle lingue che possiedono un solo vocabolo per indicare due concetti esistano effettivamente due concetti distinti o uno soltanto. Per riprendere l'esempio fatto in precedenza, sul vocabolo persiano \textit{ænduh}, ci si domanda se effettivamente gli individui che parlano questa lingua conoscano i concetti di dolore e rimpianto a cui la parola fa riferimento o se, al contrario, per loro esista solo l'insieme di questi due concetti, racchiusi in \textit{ænduh} \parencite{jackson_joshua}. \\
Ci si domanda, quindi, se diversi modi di parlare di emozioni cambino effettivamente il modo in cui le persone vivono tali emozioni.\\
Lo studio di Jackson et al., attraverso l'analisi di colessificazione, ha potuto creare delle reti che descrivono come gli individui utilizzano il lessico emotivo, cercando di esplicitare i meccanismi evolutivi, biologici e culturali che permettono di attribuire significati alle parole che si riferiscono alle emozioni in ciascuna lingua. 

Ciò che bisogna andare ad indagare sono proprio i fattori sociali, culturali, evolutivi, per andare a comprendere quanto effivamente influiscano sulle emozioni e sulla creazione del lessico emotivo. Inoltre, bisogna comprendere quanto quest'ultimo sia in grado di modificare il sentire umano interno. \\
La domanda più difficile che gli studiosi in questi ambiti si pongono è la possibilità dell'esistenza di un'emozione che non possiede un vocabolo che la rappresenta. \\
Insomma, la semantica delle emozioni, il significato che posseggono le parole che le descrivono, e la loro correlazione con l'esperienza emotiva, è un dibattito antico, che rimane ancora oggi attuale nella letteratura scientifica. 

Grazie a questi importantissime ricerche, però, si può afferrare al meglio la vastità di variazioni interlinguistche per quanto riguarda il lessico emotivo e approfondire la relazione tra i vocaboli utilizzati per descrivere gli stati emotivi, la manifestazione delle emozioni e il sentire emotivo interno a ciascun individuo.



\backmatter

\clearpage
\chapter{Conclusione}
Il presente elaborato permette di andare a scoprire ed esaminare un ambito molto importante della psicologia delle emozioni, che per svariato tempo è stato trascurato: il lessico emotivo. Questo argomento si ritiene molto interessante per diversi motivi: la descrizione delle emozioni attraverso le parole dà loro significato e permette la loro comunicazione; inoltre risulta fondamentale per un'indagine approfondita delle emozioni e la possibilità di portare a termine ricerche interculturali, che mettono a confronto le emozioni in diverse culture. 

Per addentrarsi negli aspetti culturali del lessico emotivo è essenziale conoscere le prospettive teoriche che lo circondano, andandone a cogliere gli elementi fondamentali. \\
L'universalismo permette di individuare emozioni di base, comuni in tutte le culture, che ogni individuo sembra essere in grado di riconoscere, indipendentemente dal luogo in cui vive, la lingua che parla e la società in cui è inserito. \\
La prospettiva socio-costruttivista, invece, ha idee completamente diverse e sostiene che le emozioni vengano socialmente costruite, attraverso norme e valori della cultura in cui si nasce e cresce, rendendole un qualcosa che può essere appreso e normato dalla società. \\
Un elemento che aggiungono alcuni autori appartenenti alla corrente socio-costruttivista è il ruolo del lessico emotivo: le parole che si usano per descrivere le emozioni sembrano essere il mezzo capace di dare loro significato, e quindi di costruirle. Data la costruzione e l'apprendimento delle emozioni in uno specifico contesto culturale, le norme socio-culturali presenterebbero un ruolo importantissimo: influenzerebbero completamente la costruzione delle emozioni in ciascuna cultura.\\
Dopo aver visto alcuni dei principali autori che contribuiscono alle teorie universaliste e socio-costruttiviste, si sono cercate di analizzare varie differenze e analogie del lessico emotivo in diverse culture, che possiamo trovare attraverso il confronto di varie lingue e famiglie linguistiche. Per fare ciò va indagato l'aspetto semantico del lessico emotivo, che dà significato alle parole, andando a rappresentare stati emotivi specifici. \\
Attraverso i risultati degli studi condotti, si sono andati a delineare dei modelli dimensionali, a seconda del numero di dimensioni riscontrate nei termini emotivi di ciascuna lingua. 

Le conclusioni ottenute dalle molteplici ricerche interlinguistiche sono assai varie e possono dare luogo ad interessanti discussioni. \\
Sono state riscontrate moltissime variazioni sia per quanto riguarda le dimensioni del lessico emotivo, che per la presenza di termini emotivi unici, esistenti in una sola lingua; ma anche parecchie somiglianze. Si è visto, infatti, che nella maggior parte delle lingue analizzate è possibile identificare le dimensioni di attivazione fisiologica e valenza: questo indicherebbe la possibile presenza di elementi innati delle emozioni, presenti in tutte le culture e rappresentati in tutte le lingue. \\
Si è notato, inoltre, che le molte lingue risultanti simili tra loro sono geograficamente vicine: tali analogie, quindi, potrebbero essere date da antenati linguistici comuni, eventi storici e quindi norme sociali e culturali simili. 

Analizzando, invece, le differenze emerse nel lessico di culture distanti si possono fare diverse ipotesi: tali diversità linguistiche potrebbero essere la rappresentazione delle regole e valori di una determinata cultura, le quali influenzerebbero fortemente la manifestazione emotiva. Ad esempio, se in una determinata lingua si hanno molti termini per descrivere emozioni di rabbia si può dedurre che in quella cultura la rabbia sia un'emozione importante, la cui manifestazione è accettata e promossa; al contrario, la mancanza di termini che indicano altre emozioni potrebbero rappresentare una non tolleranza sociale di esse. \\
Le regole e i valori che normerebbero la manifestazione o la soppressione di certe emozioni potrebbero essere dovuti a fattori storici, geografici o religiosi.

Un'altra ipotesi che è possibile effettuare riguardo le variazioni interlinguistiche sarebbe quella proposta dai socio-costruttivisti, quindi l'idea secondo la quale è proprio il lessico emotivo il mezzo con il quale si costruiscono le emozioni. \\
In questo caso, quindi, sarebbero proprio le parole, con la loro funzione semantica, a far nascere emozioni diverse e a condizionare anche il modo di sentire interno agli individui. Ricorrendo all'esempio precedente, secondo questa prospettiva, sarebbe proprio una parola specifica rappresentante un sentimento di rabbia a farla provare all'individuo. 

Il dibattito, dunque, è molto vasto e può prendere direzioni diverse in base ai dati conseguiti dalle molteplici ricerche. Ciò che viene dimostrato è sia la presenza di elementi universali, che di un'incredibile variazione interlinguistica: sono le cause di tali fattori ad essere messe in continua discussione. \\
Ricerche future saranno utili per mantenere tale dibattito aperto: con una società come quella odierna in continuo cambiamento, gli aspetti culturali, linguistici ed emotivi andranno indagati sempre di più. Elementi sociali rivoluzionari come la digitalizzazione potrebbero portare a risultati nuovi, che potranno favorire una certa prospettiva piuttosto che un'altra. 
\clearpage 

\nocite{*}

\printbibliography[notkeyword={site}, heading=bibintoc]
\printbibliography[keyword={site}, title=Sitografia, heading=bibintoc]

\end{document}
% ----------------------------------------------------------
